\documentclass{article}
\usepackage[utf8]{inputenc}
\usepackage[english]{babel}
\usepackage{amsfonts}
\usepackage{amsthm}
\usepackage{amsmath}
\usepackage{amssymb}

\newtheorem{theorem}{Theorem}
\newtheorem{es}{Examples}

\newcommand{\inter}[1]{int(#1)}

\title{MATC27 Assignment 4}
\author{Anmol Bhullar - 1002678140}

\begin{document}
    \textbf{Question 5.}

    Let $(X,\tau)$ be a topological space induced by a metric $d$ on $X$. Let $A\subseteq X$ be non-empty and define $\tau_A$ to be the subspace topology 
    on $A$. Furthermore, define the metric $d_A$ by the rule $d_A(x,y) = d(x,y)$ for all $x,y\in A$ (note it is straightforward to see that $d_A$ is a metric
    since $d$ is a metric and $A\subseteq X$) and define the topology $\tau_{d_A}$ to be the topology induced by the metric $d_A$ on $A$. We want to show that
    $\tau_{d_A} = \tau_A$.\\

    To do this, we attempt to find a basis $\beta$ of $\tau_A$ and a basis $\Gamma$ of $\tau_{d_A}$ and show the two bases are equivalent. By definition, this
    would imply that they generate the same topology which is enough to imply that $\tau_A = \tau_{d_A}$.\\

    We know for a topology induced by a metric, the collection of all $\epsilon$-balls forms a basis for that space. Thus, define $\beta_X$ to be the collection
    of all $\epsilon$-balls given by the metric $d$, then $\beta_X$ is a basis for $\tau$. Furthermore, define $\beta = \{B' \cap A: B'\in\beta_X\}$. Since
    $\beta_X$ is a basis for $\tau$ and $\tau_A$ is a subspace topology of it, we obtain that $\beta$ is a basis for the subspace topology $\tau_A$. Similarly to before,
    define $\triangledown$ to be a basis for $\tau_{d_A}$ given by the collection of all $\epsilon$-balls given by the metric $d_A$. We claim these two bases are 
    equivalent and prove it below.\\

    First, we show that for all $B\in\beta$, and for all $x\in B$, there exists $B'\in\triangledown$ such that $x\in B'\subseteq B$. Thus, choose an arbitrary
    $B\in\beta$ and from this, choose an arbitrary $x\in B$. From the definition of $\beta$, we can write $B$ as $H \cap A$ where $H$ is some $\epsilon$-ball.
    Suppose $H\subseteq A$. Then $B = H \cap A = H$ (note, this also implies $x\in H$). 
    Note that for all $z,w\in H$, we have $z,w\in A$ so that $\epsilon > d(x,y) = d_A(x,y)$ so that
    $H$ is an $\epsilon$-ball in $\beta_X$ and $\triangledown$. Thus, we have the existence of a set $H\in\triangledown$ such that $x\in H\subseteq H \cap A = B$ 
    as wanted. Now, suppose $H\subseteq A$ is not true. For any $\epsilon > 0$, we know $x\in B_{d_A}(x,\epsilon)\subseteq A$ (since $d_A$ is a metric on $A$).
    Also, since $H\in\tau$, we can find an $\epsilon$-ball around any point in $H$ such that the ball is contained in $H$ i.e. there exists $\epsilon_1>0$ such that
    $x\in B_d(x,\epsilon_1)\subseteq H$. So, now we have two $\epsilon$-balls centered at $x$. If $\epsilon_1 < \epsilon$, we get a similar case to if $H\subseteq A$,
    thus assume that $\epsilon_1 < \epsilon$. It follows $B_{d_A}(x,\epsilon_1)\subseteq H$ and $B_{d_A}(x,\epsilon_1)\subseteq A$ so that $x\in B_{d_A}(x,\epsilon_1)
    \subseteq H \cap A = B$ as wanted. Since, we covered all cases (either $H\subseteq A$ or not), we are done for this direction.\\
    
    Now, we show that for all $B'\in\triangledown$, and for all $x\in B'$, there exists $B\in\beta$ such that $x\in B\subseteq B'$. Since $B'\in\triangledown$,
    we can write $B'$ as an $\epsilon$-ball i.e. $B' = B_{d_A}(y,\epsilon)$ for some $\epsilon>0$ and $y\in X$. Note, since $\epsilon > d_A(x,y) = d(x,y)$, we have
    that $x\in B_d(y,\epsilon)$. Note, $B_d(y,\epsilon) \cap A = \{x: d(x,y) < \epsilon\} \cap A = 
    \{ x\in A: d(x,y) < \epsilon\} = \{ x\in A: d_A(x,y) < \epsilon\} = B_{d_A}(y,\epsilon)$. Since $B_d(y,\epsilon)\cap A\in\beta$, it follows that:
    $x\in B_d(y,\epsilon) \cap A \subseteq B_{d_A}(y,\epsilon)$ as wanted.\hfill$\blacksquare$\\
\end{document}
