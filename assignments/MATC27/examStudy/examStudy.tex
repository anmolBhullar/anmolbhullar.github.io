\documentclass{article}
\usepackage[utf8]{inputenc}
\usepackage[english]{babel}
\usepackage{amsfonts}
\usepackage{amsthm}
\usepackage{amsmath}
\usepackage{amssymb}
\usepackage{nicefrac}

\newtheorem{theorem}{Theorem}
\newtheorem{es}{Examples}
\newtheorem*{example}{Example}

\newcommand\restr[2]{{% we make the whole thing an ordinary symbol
    \left.\kern-\nulldelimiterspace % automatically resize the bar with \right
    #1 % the function
    \vphantom{\big|} % pretend it's a little taller at normal size
    \right|_{#2} % this is the delimiter
}}

\theoremstyle{definition}
\newtheorem{definition}{Definition}[section]
 
\theoremstyle{remark}
\newtheorem*{remark}{Remark}

\newtheorem{corollary}{Corollary}[theorem]
\newtheorem{lemma}[theorem]{Lemma}

\newcommand{\inter}[1]{int(#1)}

\title{MATC27 Exam Study}
\author{Anmol Bhullar}

\begin{document}
    \maketitle

    \textbf{Chapter 2: Section 12.}
    \begin{definition}
        A \textbf{topology} on a set $X$ is a collection $\tau$ of subsets of $X$ having the following properties:
        \begin{enumerate}
            \item $\emptyset$ and $X$ are in $\tau$
            \item The union of the elements of any subcollection of $\tau$ is in $\tau$
            \item The intersection of the elements of any finite subcolelction of $\tau$ is in $\tau$
        \end{enumerate}
        A set $X$ for which a topology $\tau$ has been specified is called a \textbf{topological space}. We might write this
        as $(X,\tau)$ or as $X$ if the topology put on $X$ is unambigious.
    \end{definition}

    \begin{definition}
        If $X$ is a topological space with topology $\tau$, we say that a subset $U$ of $X$ is an \textbf{open set} of $X$ if $U$
        belongs to the collection $\tau$.
    \end{definition}

    \begin{es}
        If $X$ is any set, the collection of \textit{all} subsets of $X$ is a topology on $X$; it is called the \textbf{discrete
            topology}. The collection consisting of $X$ and $\emptyset$ only is also a topology on $X$; we shall call it the
        \textbf{indiscrete topology}, or the \textbf{trivial topology}. Let $X$ be a set; let $\tau_f$ be the collection of all
        subsets $U$ of $X$ such that $X-U$ either is finite or is all of $X$. Then $\tau_f$ is a topology on $X$, called the
        \textbf{finite complement topology}.
    \end{es}
    
    \begin{definition}
        Suppose that $\tau$ and $\tau'$ are two topologies on a given set $X$. If $\tau'\supseteq\tau$, we say that $\tau'$ is
        \textbf{finer} than $\tau$; if $\tau'$ \textit{properly} contains $\tau$, we say that $\tau'$ is \textbf{strictly finer}
        than $\tau$. We also say that $\tau$ is \textbf{coarser} than $\tau'$, or \textbf{strictly coarser}, in these two respective
        situations. We say $\tau$ is \textbf{comparable} with $\tau'$ if either $\tau'\supseteq\tau$ or $\tau\supseteq\tau'$.
    \end{definition}

    \newpage

    \textbf{Chapter 2: Section 13.}
    \begin{definition}
        If $X$ is a set, a \textbf{basis} for a topology on $X$ is a collection $\mathcal{B}$ of subsets of $X$ 
        (called \textbf{basis elements}) such that
        \begin{enumerate}
            \item For each $x\in X$, there is at least one basis element $B$ containing $x$.
            \item If $x$ belongs to the intersection of two basis elements $B_1$ and $B_2$, then there is a basis element $B_3$
                containing $x$ such that $B_3\subseteq B_1\cap B_2$.
        \end{enumerate}
        If $\mathcal{B}$ satisfies these two conditions, then we define the \textbf{topology $\tau$ generated by} $\mathcal{B}$
        as follows: A subset $U$ of $X$ is said to be open in $X$ (that is, to be an element of $\tau$) if for
        each $x\in U$, there is a basis element $B\in\mathcal{B}$ such that $x\in B$ and $B\subseteq U$. Note that each
        basis element is itself an element of $\tau$.
    \end{definition}

    \begin{lemma}[Lemma 13.1]
        Let $X$ be a set; let $\mathcal{B}$ be a basis for a topology on $X$. Then $\tau$ equals the collection of all unions of
        of elements of $\mathcal{B}$.
    \end{lemma}

    \begin{lemma}[Lemma 13.2]
        Let $X$ be a topological space. Suppose that $\mathcal{C}$ is a collection of open sets of $X$ such that for each open set $U$ 
        of $X$ and each $x$ in $U$, there is an element $C$ of $\mathcal{C}$ such that $x\in C\subseteq U$. Then $\mathcal{C}$ is a
        basis for the topology of $X$.
    \end{lemma}

    \begin{lemma}[Lemma 13.3]
        Let $\mathcal{B}$ and $\mathcal{B}'$ be bases for the topologies $\tau$ and $\tau'$, respectively, on $X$. Then the following
        are equivalent:
        \begin{enumerate}
            \item $\tau'$ is finer than $\tau$
            \item For each $x\in X$ and each basis element $B\in\mathcal{B}$ containing $x$, there is a basis element $B'\in\mathcal{B}'$
                such that $x\in B'\subseteq B$.
        \end{enumerate}
    \end{lemma}

    \begin{definition}
        If $\mathcal{B}$ is the collection of all open intervals in the real line,
        \[ (a,b) = \{x: a < x < b\} \]
        the topology generated by $\mathcal{B}$ is called the \textbf{standard topology} on the real line. Whenever we consider
        $\mathbb{R}$, we shall suppose it is given this topology unless we specifically state otherwise. If $\mathcal{B}'$ is the
        collection of all half open intervals of the form
        \[ [a,b) = \{x: a\leq x < b\} \]
        where $a < b$, the topology generated by $\mathcal{B}'$ is called the \textbf{lower limit topology} on $\mathbb{R}$.
        When $\mathbb{R}$ is given the lower limit topology, we denote it by $\mathbb{R}_l$. Finally let $K$ denote the
        set of all numbers of the form $\frac{1}{n}$, for $n\in\mathbb{Z}_+$, and let $\mathcal{B}''$ be the collection of
        all open intervals $(a,b)$, along with the sets of the form $(a,b) - K$. The topology generated by $\mathcal{B}''$ will
        be called the \textbf{K-topology} on $\mathbb{R}$.
    \end{definition}

    \begin{lemma}[Lemma 13.4]
        The topologies of $\mathbb{R}_l$ and $\mathbb{R}_K$ are strictly finer than the standard topology on $\mathbb{R}$, but are
        not comparable with one another.
    \end{lemma}

    \begin{definition}
        A \textbf{subbasis} $\mathcal{S}$ for a topology on $X$ is a collection of subsets of $X$ whose union equals $X$. The
        \textbf{topology generated by the subbasis} $\mathcal{S}$ is defined to be the collection $\tau$ of all unions of finite
        intersections of elements of $\mathcal{S}$.
    \end{definition}

    \newpage

    \textbf{Chapter 2: Section 15.}
    \begin{definition}
        Let $X$ and $Y$ be topological spaces. The \textbf{product topology} on $X\times Y$ is the topology having as basis the
        collection $\mathcal{B}$ of all sets of the form $U\times V$, where $U$ is an open subset of $X$ and $V$ is an open subset
        of $Y$.
    \end{definition}

    \begin{theorem}[Theorem 15.1]
        If $\mathcal{B}$ is a basis for the topology of $X$ and $\mathcal{C}$ is basis for the topology of $Y$, then the collection
        \[ \mathcal{D} = \{\mathcal{B}\times \mathcal{C}: B\in\mathcal{B}\:\text{and}\: C\in\mathcal{C}\} \]
        is a basis for the topology of $X\times Y$.
    \end{theorem}

    \begin{definition}
        Let $\pi_1: X\times Y\to X$ be defined by the equation
        \[ \pi_1(x,y) = x; \]
        let $\pi_2: X\times Y\to Y$ be defined by the equation
        \[ \pi_2(x,y) = y. \]
        The maps $\pi_1$ and $\pi_2$ are called \textbf{projections} of $X\times Y$ onto its first and second factors, respectively.
    \end{definition}

    \begin{theorem}[Theorem 15.2]
        The collection
        \[ \mathcal{S} = \{\pi_1^{-1}(U): U\:\text{open in}\: X\} \cup \{\pi_2^{-1}(V): V\:\text{open in}\:Y\} \]
        is a subbasis for the product topology on $X\times Y$.
    \end{theorem}

    \newpage

    \textbf{Chapter 2: Section 16.}
    \begin{definition}
        Let $X$ be a topological space with topology $\tau$. If $Y$ is a subset of $X$, the collection
        \[ \tau_Y = \{Y\cap U: U\in \tau\} \]
        is a topology on $Y$, called the \textbf{subspace topology}. With this topology, $Y$ is called a \textbf{subspace} of $X$;
        its open sets consist of all intersections of open sets of $X$ with $Y$.
    \end{definition}

    \begin{lemma}[Lemma 16.1]
        If $\mathcal{B}$ is a basis for the topology of $X$ then the collection
        \[ \mathcal{B}_Y = \{B\cap Y: B\in\mathcal{B}\} \]
        is a basis for the subspace topology on $Y$.
    \end{lemma}

    If $Y$ is a subspace of $X$, we say that a set $U$ is \textbf{open in $Y$} (or open \textit{relative} to $Y$) if it belongs
    to the topology on $Y$; this implies in particular that it is a subset of $Y$. We say that $U$ is \textbf{open in $X$} if it
    belongs to the topology of $X$.

    \begin{lemma}[Lemma 16.2]
        Let $Y$ be a subspace of $X$. if $U$ is open in $Y$ and $Y$ in $X$, then $U$ is open in $X$.
    \end{lemma}

    \begin{theorem}[Theorem 16.3]
        If $A$ is a subspace of $X$ and $B$ is a subspace of $Y$, then the product topology on $A\times B$ is the same as the topology
        $A\times B$ inherits as a subspace of $X\times Y$.
    \end{theorem}

    \newpage

    \textbf{Chapter 2: Section 17.}
    \begin{definition}
        A subset $A$ of a topological space $X$ is said to be \textbf{closed} if the set $X-A$ is open.
    \end{definition}

    \begin{theorem}[Theorem 17.1]
        Let $X$ be a topological space. Then the following conditions hold:
        \begin{enumerate}
            \item $\emptyset$ and $X$ are closed
            \item Arbitrary intersections of closed sets are closed
            \item Finite unions of closed sets are closed
        \end{enumerate}
    \end{theorem}

    If $Y$ is a subspace of $X$, we say that a set $A$ is \textbf{closed in $Y$} if $A$ is a subset of $Y$ and if $A$ is closed 
    in the subspace topology of $Y$ (that is, if $Y-A$ is open in $Y$).

    \begin{theorem}[Theorem 17.2]
        Let $Y$ be a subspace of $X$. Then a set $A$ is closed in $Y$ if and only if it equals the intersection of a closed set of
        $X$ with $Y$.
    \end{theorem}

    \begin{theorem}[Theorem 17.3]
        Let $Y$ be a subspace of $X$. If $A$ is closed in $Y$ and $Y$ is closed in $X$, then $A$ is closed in $X$.
    \end{theorem}

    \begin{definition}
        Given a subset $A$ of a topological space $X$, the \textbf{interior} of $A$ is defined as the union of all open sets
        contained in $A$ i.e. the largest open set contained in $A$, and the \textbf{closure} of $A$ is defined as the intersection
        of all closed sets contained in $A$ i.e. the smallest closed set which contains $A$.
    \end{definition}

    The interior of $A$ is denoted by \inter{A} or $A^{\circ}$ and the closure of $A$ is denoted Cl $A$ or by $\overline{A}$.

    \begin{theorem}[Theorem 17.4]
        Let $Y$ be a subspace of $X$; let $A$ be a subset of $Y$; let $\overline{A}$ denote the closure of $A$ in $X$. Then the closure
        of $A$ in $Y$ equals $\overline{A}\cap Y$.
    \end{theorem}

    Say that a set $A$ \textbf{intersects} a set $B$ if the intersection $A\cap B$ is not empty.

    \begin{theorem}[Theorem 17.5]
        Let $A$ be a subset of the topological space $X$.
        \begin{enumerate}
            \item Then $x\in\overline{A}$ if and only if every open set $U$ containing $x$ intersects $A$.
            \item Supposing the topology of $X$ is given by a basis, then $x\in\overline{A}$ if and only if every basis element
                $B$ containing $x$ intersects $A$.
        \end{enumerate}
    \end{theorem}
    \begin{proof}
        Prove (a). We prove the contrapositive of (a):
        \[ x\not\in\overline{A} \Leftrightarrow \text{there exists an open set $U$ containing $x$ that does not intersect $A$} \]
        If $x$ is not in $\overline{A}$, the set $U = X - \overline{A}$ is an open set containing $x$ that does not intersect
        $A$, as desired. Conversely, if there exists an open set $U$ containing $x$ which does not intersect $A$, then $X-U$ is a
        closed set containing $A$. By definition of closure $\overline{A}$, the set $X-U$ must contain $\overline{A}$; therefore,
        $x$ cannot be in $A$.
    \end{proof}
    
    Say "$U$ is a \textbf{neighbourhood} of $x$" if $U$ is some open set containing $x$.

    \begin{definition}
        If $A$ is a subset of the topological space $X$ and if $x$ is a point of $X$, we say that $x$ is a \textbf{limit point}
        (or "cluster point," or "point of accumulation") of $A$ if every neighborhood of $x$ intersects $A$ in some point
        \textit{other than $x$ itself} i.e. $x$ is a limit point of $A$ if it belongs to the closure of $A - \{x\}$
    \end{definition}

    \begin{theorem}[Theorem 17.6]
        Let $A$ be a subset of the topological space $X$; let $A'$ be the set of all limit points of $A$. Then
        \[ \overline{A} = A \cup A' \]
        We may also denote $A'$ by $\partial A$.
    \end{theorem}

    \begin{corollary}[Corollary 17.7]
        A subset of a topological space is closed if and only if it contains all limit points.
    \end{corollary}

    In an arbitrary topological space, one says that a sequence $x_1,x_2,\hdots$ of points of the space $X$ \textbf{converges} to
    the point $x$ of $X$ provided that, corresponding to each neighborhood $U$ of $x$, there is a positive integer $N$ such that
    $x_n\in U$ for all $n\geq N$.

    \begin{definition}
        A topological space $X$ is called a \textbf{Hausdorff space} if for each pair $x_1,x_2$ of distinct points of $X$,
        there exists a neighborhoods $U_1$, and $U_2$ of $x_1$ and $x_2$, respectively, that are disjoint.
    \end{definition}

    \begin{theorem}[Theorem 17.8]
        Every finite point set in a Hausdorff space $X$ is closed.
    \end{theorem}

    The condition that finite point sets are closed is weaker than Hausdorff, if a topological space $X$ has the condition
    that all finite point sets are closed, then it is said to be \textbf{T$_1$}. The property itself is called the T$_1$ axiom.

    \begin{theorem}[Theorem 17.9]
        Let $X$ be a space satisfying the $T_1$ axiom; let $A$ be a subset of $X$. Then the point $x$ is a limit point of $A$ if and
        only if every neighborhood of $x$ contains infinitely many points of $A$.
    \end{theorem}

    \begin{theorem}[Theorem 17.10]
        If $X$ is a Hausdorff space, then a sequence of points of $X$ converges to at most one point of $X$.
    \end{theorem}

    \newpage

    \textbf{Chapter 2: Section 18.}
    \begin{definition}
        Let $X$ and $Y$ be topological spaces. A function $f: X\to Y$ is said to be \textbf{continuous} if for each open subset
        $V$ of $Y$, the set $f^{-1}(V)$ is an open subset of $X$. Since the continuity on the definition of $f$ \textit{and} on the
        topologies on $X$ and $Y$, we can also say $f$ is continous \textit{relative} to specific topologies $X$ and $Y$.
    \end{definition}

    \begin{example}
        Let us consider a function like those studied in analysis, a "real-valued function of a real variable,"
        \[ f: \mathbb{R} \to \mathbb{R} \]
        In analysis, one defines continuity of $f$ via the "$\epsilon-\delta$ definition". To prove that our definition implies
        the $\epsilon-\delta$ definition, for instance, we proceed as follows:\\
        Given $x_0\in\mathbb{R}$, and given $\epsilon>0$, the interval $V = (f(x_0)-\epsilon,f(x_0)+\epsilon)$ is an open set
        of the range space $\mathbb{R}$. Therefore, $f^{-1}(V)$ is an open set of the domain space $\mathbb{R}$ (we assumed our
        topological definition holds). Because $f^{-1}(V)$ contains the point $x_0$, it contains some basic element $(a,b)$ about
        $x_0$. We choose $\delta$ to be the smaller of the two numbers $x_0-a$ and $b-x_0$. Then if $|x-x_0|<\delta$, the point
        $x$ must be in $(a,b)$, so that $f(x)\in V$, and $|f(x) - f(x_0)| < \epsilon$, as desired.
    \end{example}

    \begin{example}
        Let
        \[ f: \mathbb{R} \to \mathbb{R}_l \]
        be the identity function; $f(x) = x$ for every real number $x$. Then $f$ is not a continuous function; the inverse image of
        the open set $[a,b)$ of $\mathbb{R}_l$ equals itself, which is not open in $\mathbb{R}$. On the other hand, the identity
        function
        \[ g: \mathbb{R}_l \to \mathbb{R} \]
        is continuous, because the inverse image of $(a,b)$ is itself, which is open in $\mathbb{R}_l$.
    \end{example}

    \begin{theorem}[Theorem 18.1]
        Let $X$ and $Y$ be topological spaces; let $f: X\to Y$. Then the following are equivalent:
        \begin{enumerate}
            \item $f$ is continuous.
            \item For every subset $A$ of $X$, one has $f(\overline{A})\subseteq \overline{f(A)}$.
            \item For every closed set $B$ of $Y$, the set $f^{-1}(B)$ is closed in $X$.
            \item For each $x\in X$ and each neighborhood $V$ of $f(x)$, there is a neighborhood $U$ of $x$ such that $f(U)\subseteq V$.
        \end{enumerate}
        If the condition in (4) holds for the point $x$ of $X$, we say that $f$ is \textbf{continuous at the point $x$}.
    \end{theorem}

    \begin{definition}
        Let $X$ and $Y$ be topological spaces; let $f: X\to Y$ be a bijection. If both the function $f$ and the inverse function
        \[ f^{-1}: Y \to X \]
        are continuous (and exist), then $f$ is called a \textbf{homeomorphism}.
    \end{definition}

    These functions have the property that any property that can be expressed solely via the open sets of a space are preserved
    when mapped onto a different space. Such a property (that can be expressed solely via open sets) is called a 
    \textbf{topological property}.

    \begin{example}
        The function $F: (-1,1) \to \mathbb{R}$ defined by
        \[ F(x) = \frac{x}{1-x^2} \]
        is a homeomorphism. A bijective continuous function does not necessairly have to be continuous, for example, the identity
        map $g: \mathbb{R}_l \to \mathbb{R}$ is bijective continuous but it's inverse is not continuous as noted earlier.
    \end{example}

    \begin{theorem}[Theorem 18.2]
        Let $X,Y$ and $Z$ be topological spaces.
        \begin{enumerate}
            \item (Constant function) If $f: X\to Y$ mapps all of $X$ into the single point $y_0$ of $Y$, then $f$ is continuous.
            \item (Inclusion) If $A$ is a subspace of $X$, the inclusion function $j: A\to X$ is continuous.
            \item (Composites) If $f: X\to Y$ and $g: Y\to Z$ are continuous, then the map $g\circ f: X\to Z$ is continuous.
            \item (Restricting the domain) If $f: X\to Y$ is continuous, and if $A$ is a subspace of $X$, then the restricted
                function $\restr{f}{A}: A\to Y$ is continuous.
            \item (Restricting or expanding the range) Let $f: X\to Y$ be continuous. If $Z$ is a subspace of $Y$ containing
                the image set $f(X)$, then the function $g: X\to Z$ obtained by restricting the range of $f$ is continuous. If
                $Z$ is a space having $Y$ as a subspace, then the function $h: X\to Z$ obtained by expanding the range of $f$
                is continuous.
            \item (Local formulation of continuity) The map $f:X\to Y$ is continuous if $X$ can be written as the union of open
                sets $U_{\alpha}$ such that $\restr{f}{U_{\alpha}}$ is continuous for each $\alpha$.
        \end{enumerate}
    \end{theorem}

    \begin{theorem}[Theorem 18.3]
        Let $X = A\cup B$, where $A$ and $B$ are closed in $X$. Let $f: A\to Y$ and $g: B\to Y$ be continuous. If $f(x) = g(x)$
        for every $x\in A\cap B$, then $f$ and $g$ combine to give a continuous function $h: X\to Y$, defined by setting
        $h(x) = f(x)$ if $x\in A$, and $h(x) = g(x)$ if $x\in B$.
    \end{theorem}

    \begin{theorem}[Theorem 18.4]
        Let $f: A\to X\times Y$ be given by the equation
        \[ f(a) = (f_1(a),f_2(a)) \]
        Then $f$ is continuous if and only if the functions
        \[ f_1:A\to X\qquad\text{and}\qquad f_2: A\to Y \]
        are continuous.
    \end{theorem}

    \newpage

    \textbf{Chapter 2: Section 19.}
    \begin{definition}
        Let $\{A_{\alpha}\}_{\alpha\in J}$ be an indexed family of sets; let $X = \bigcup_{\alpha\in J} A_{\alpha}$. The
        \textbf{cartesian product} of this indexed family, denoted by
        \[ \prod_{\alpha\in J} A_{\alpha} \]
        is defined to be the set of all $J$-tuples $(x_{\alpha})_{\alpha\in J}$ of elements of $X$ such that $x_{\alpha}\in A_{\alpha}$
        for each $\alpha\in J$. That is, it is the set of all functions
        \[ \mathbf{x}: J \to \bigcup_{\alpha\in J} A_{\alpha} \]
        such that $\mathbf{x}(\alpha)\in A_{\alpha}$ for each $\alpha\in J$. If all $A_{\alpha} = X$, then the cartesian product
        is denoted by $X^J$.
    \end{definition}

    \begin{definition}
        Let $\{X_{\alpha}\}_{\alpha\in J}$ be an indexed family of topological spaces. Let us take as a basis for a topology on the
        product space
        \[ \prod_{\alpha\in J} X_{\alpha} \]
        the collection of all sets of the form
        \[ \prod_{\alpha\in J} U_{\alpha} \]
        where $U_{\alpha}$ is open in $X_{\alpha}$, for each $\alpha\in J$. The topology generated by this basis is called the
        \textbf{the box topology}.
    \end{definition}

    Now we generalize the subbasis formulation of this definition. Let
    \[ \pi_{\beta}: \prod_{\alpha\in J} X_{\alpha} \to X_{\beta} \]
    be the function assigning to each element of the product space its $\beta$th coordinate,
    \[ \pi_{\beta}((x_{\alpha})_{\alpha\in J}) = x_{\beta}; \]
    it is called the \textbf{projection mapping} associated with the index $\beta$.

    \begin{definition}
        Let $\mathcal{S}_{\beta}$ denote the collection
        \[ \mathcal{S}_{\beta} = \{\pi_{\beta}^{-1}(U_{\beta}): U_{\beta}\:\text{open in}\: X_{\beta}\} \]
        and let $\mathcal{S}$ denote the union of these collections,
        \[ \mathcal{S} = \bigcup_{\beta\in J} \mathcal{S}_{\beta} \]
        The topology generated by the subbasis $\mathcal{S}$ is called the \textbf{product topology}. In this topology 
        $\prod_{\alpha\in J} X_{\alpha}$ is called a \textbf{product space}.
    \end{definition}

    The following two facts may be important:
    \begin{align*}
        (\prod_{\alpha\in J} U_{\alpha}) \cap (\prod_{\alpha\in J} V_{\alpha}) = \prod_{\alpha\in J} (U_{\alpha}\cap V_{\alpha}) \\
        \pi_{\beta}^{-1}(U_{\beta}) \cap \pi_{\beta}^{-1}(V_{\beta}) = \pi_{\beta}^{-1}(U_{\beta} \cap V_{\beta})
    \end{align*}

    \begin{theorem}[Theorem 19.1]
        The box topology on $\prod X_{\alpha}$ has as basis all sets of the form $\prod U_{\alpha}$, where $U_{\alpha}$
        is open in $X_{\alpha}$ for each $\alpha$. The product topology on $\prod X_{\alpha}$ has as basis all sets of the form
        $\prod U_{\alpha}$, where $U_{\alpha}$ is open in $X_{\alpha}$ for each $\alpha$ and $U_{\alpha}$ equals $X_{\alpha}$
        except for finitely many values of $\alpha$.
    \end{theorem}

    \begin{theorem}[Theorem 19.2]
        Suppose the topology on each space $X_{\alpha}$ is given by a basis $\mathcal{B}_{\alpha}$. The collection of all sets of
        the form
        \[ \prod_{\alpha\in J} B_{\alpha} \]
        where $B_{\alpha}\in\mathcal{B}_{\alpha}$ for each $\alpha$, will serve as a basis for the box topology on 
        $\prod_{\alpha\in J} X_{\alpha}$.\\
        The collection of all sets of the same form, where $B_{\alpha}\in\mathcal{B}_{\alpha}$ for finitely many
        indices $\alpha$ and $B_{\alpha} = X_{\alpha}$ for all the remaining indices, will serve as a basis for the product topology
        $\prod_{\alpha\in J} X_{\alpha}$.
    \end{theorem}

    \begin{theorem}[Theorem 19.3]
        Let $A_{\alpha}$ be a subspace of $X_{\alpha}$, for each $\alpha\in J$. Then $\prod A_{\alpha}$ is a subspace of
        $\prod X_{\alpha}$ if both products are given the box topology, or if both products are given the same product topology.
    \end{theorem}

    \begin{theorem}[Theorem 19.4]
        If each space $X_{\alpha}$ is a Hausdorff space, then $\prod X_{\alpha}$ is a Hausdorff space in both the box and product
        topologies.
    \end{theorem}

    \begin{theorem}[Theorem 19.5]
        Let $\{X_{\alpha}\}$ be an indexed family of spaces; let $A_{\alpha}\subseteq X_{\alpha}$ for each $\alpha$. If
        $\prod X_{\alpha}$ is given either the product or the box topology, then
        \[ \prod \overline{A_{\alpha}} = \overline{\prod A_{\alpha}} \]
    \end{theorem}

    \begin{theorem}[Theorem 19.6]
        Let $f: A \to \prod_{\alpha\in J} X_{\alpha}$ be given by the equation
        \[ f(a) = (f_{\alpha}(a))_{\alpha\in J} \]
        where $f_{\alpha}: A \to X_{\alpha}$ for each $\alpha$. Let $\prod X_{\alpha}$ have the product topology. Then the function
        $f$ is continuous if and only if each function $f_{\alpha}$ is continuous.
    \end{theorem}

    \newpage

    \textbf{Chapter 2: Section 20.}
    \begin{definition}
        A \textbf{metric} on a set $X$ is a function
        \[ d: X \times X \to \mathbb{R} \]
        having the following properties:
        \begin{enumerate}
            \item $d(x,y)\geq 0$ for all $x,y\in X$; equality holds if and only if $x=y$
            \item $d(x,y) = d(y,x)$ for all $x,y\in X$.
            \item (Triangle Inequality) $d(x,y) + d(y,z) \geq d(x,z)$ for all $x,y,z\in X$.
        \end{enumerate}
    \end{definition}

    Given a metric $d$ on $X$, the number $d(x,y)$ is often called the \textbf{distance} between $x$ and $y$ in the metric $d$. Given
    $\epsilon>0$, consider the set
    \[ B_d(x,\epsilon) = \{y: d(x,y) < \epsilon\} \]
    of all points $y$ whose distance from $x$ is less than $\epsilon$. It is called the \textbf{$\epsilon$-ball centered at $x$}.
    
    \begin{definition}
        If $d$ is a metric on the set $X$, then the collection of all $\epsilon$-balls $B_d(x,\epsilon)$, for $x\in X$ and $\epsilon>0$,
        is a basis for a topology on $X$, called the \textbf{metric topology} induced by $d$.
    \end{definition}

    \begin{definition}
        If $X$ is a topological space, $X$ is said to be \textbf{metrizable} if there exists a metric $d$ on the set $X$ that
        induces the topology of $X$. A \textbf{metric space} is a metrizable space $X$ together with a specific metric $d$ that gives
        the topology of $X$.
    \end{definition}

    \begin{definition}
        Let $X$ be a metric space with metric $d$. A subset $A$ of $X$ is said to be \textbf{bounded} if there is some number $M$
        such that
        \[ d(a_1,a_2) \leq M \]
        for every pair $a_1,a_2$ of points of $A$. If $A$ is bounded and non-empty, the \textbf{diameter} of $A$ is defined to be
        the number
        \[ \text{diam}\: A = \:\text{sup}\{d(a_1,a_2): a_1,a_2\in A\}. \]
    \end{definition}

    \begin{theorem}[Theorem 20.1]
        Let $X$ be a metric space with metric $d$. Define $\overline{d}: X\times X \to \mathbb{R}$ by the equation
        \[ \overline{d}(x,y) = \:\text{min}\{d(x,y),1\} \]
        Then $\overline{d}$ is a metric that induces the same topology as $d$. The metric $\overline{d}$ is called the
        \textbf{standard bounded metric} corresponding to $d$.
    \end{theorem}

    \begin{definition}
        Given $\mathbf{x} = (x_1,x_2,\hdots,x_n)$ in $\mathbb{R}^n$, we define the \textbf{norm} of $\mathbf{x}$ by the equation
        \[ ||x|| = (x_1^2 + \hdots + x_n^2)^{\frac{1}{2}} \]
        and we define the \textbf{euclidean metric} $d$ on $\mathbb{R}^n$ by the equation
        \[ d(\mathbf{x},\mathbf{y}) = ||\mathbf{x} - \mathbf{y}|| = [(x_1-y_1)^2 + \hdots + (x_n-y_n)^2]^{\frac{1}{2}}. \]
        We define the \textbf{square metric} $\rho$ by the equation
        \[ \rho(\mathbf{x},\mathbf{y}) =\: \text{max}\{|x_1-y_1|,\hdots,|x_n-y_n|\}. \]
    \end{definition}

    \begin{lemma}[Lemma 20.2]
        Let $d$ and $d'$ be two metrics on the set $X$; let $\tau$ and $\tau'$ be the topologies they induce, respectively. Then
        $\tau'$ is finer than $\tau$ if and only if for each $x$ in $X$ and each $\epsilon>0$, there exists $\delta>0$ such that
        \[ B_{d'}(x,\delta) \subseteq B_d(x,\epsilon). \]
    \end{lemma}

    \newpage

    \textbf{Chapter 2: Section 21}

    \begin{theorem}[Theorem 21.1]
        Let $f: X\to Y$; let $X$ and $Y$ be metrizable with metrics $d_X$ and $d_Y$, respectively. Then continuity of $f$ is equivalent
        to the requirement that given $x\in X$ and given $\epsilon>0$, there exists $\delta>0$ such that
        \[ d_X(x,y)<\delta \implies d_Y(f(x),f(y)) < \epsilon. \]
    \end{theorem}

    \begin{lemma}[Lemma 21.2]
        Let $X$ be a topological space; let $A\subseteq X$. If there is a sequence of points of $A$ converging to $x$, then
        $x\in\overline{A}$; the converse holds if $X$ is metrizable.
    \end{lemma}

    \begin{theorem}[Theorem 21.3]
        Let $f: X\to Y$. If the function $f$ is continuous, then for every convergent sequence $x_n\to x$ in $X$, the sequence
        $f(x_n)$ converges to $f(x)$. The converse holds if $X$ is metrizable.
    \end{theorem}

    \begin{lemma}[Lemma 21.4]
        The addition, subtraction, and multiplication operation are continous functions from $\mathbb{R}\times\mathbb{R}$ \textit{into}
        $\mathbb{R}$; and the quotient operation is continous function from $(\mathbb{R}\times(\mathbb{R}-\{0\})$ into $\mathbb{R}$.
    \end{lemma}

    \begin{theorem}[Theorem 21.5]
        If $X$ is a topological space, and if $f,g: X\to \mathbb{R}$ are continuous functions, then $f+g,f-g$ and $f\cdot g$ are
        continuous. If $g(x)\neq 0$ for all $x$, then $\nicefrac{f}{g}$ is continuous.
    \end{theorem}

    \newpage

    \textbf{Chapter 2: Section 22}
    \begin{definition}
        Let $X$ and $Y$ be topological spaces; let $p: X\to Y$ be a surjective map. The map $p$ is said to be a \textbf{quotient map}
        provided a subset $U$ of $Y$ is open in $Y$ if and only if $p^{-1}(U)$ is open in $X$.
    \end{definition}

    Recall that a map $f:X\to Y$ is said to be an \textbf{open map} if for each open set $U$ of $X$, the set $f(U)$ is open
    in $Y$. It is said to be a closed map if for each closed set $A$ of $X$, the set $f(A)$ is closed in $Y$. Continuous
    and surjective open and closed maps are one type of quotient maps but not all quotient maps are open or closed.

    \begin{definition}
        If $X$ is a space and $A$ is a set and if $p: X\to A$ is a surjective map, then there exists exactly one
        topology $\tau$ on $A$ relative to which $p$ is a quotient map, it is called the \textbf{quotient topology} induced by $p$.
    \end{definition}

    \begin{example}
        \[ D^n \setminus S^{n-1} \cong S^n \]
        The square in $\mathbb{R}^2$ quotients to the taurus.
    \end{example}

    An example of a quotient map which is neither open or closed.
    \begin{example}
        Let $\pi_1: \mathbb{R}\times\mathbb{R}\to\mathbb{R}$ be a projection on the first coordinate. Let $A$ be the subspace
        of $\mathbb{R}\times\mathbb{R}$ consisting of all points $x\times y$ for which either $x\geq 0$ or $y=0$ (or both); let
        $q: A\to\mathbb{R}$ be obtained by restricting $\pi_1$.
    \end{example}

    \newpage

    \textbf{Chapter 3: Section 23}
    \begin{definition}
        Let $X$ be a topological space. A \textbf{separation} of $X$ is a pair $U,V$ of disjoint nonempty open subsets of $X$ whose
        union is $X$. The space $X$ is said to be \textbf{connected} if there does not exist a separation of $X$.
    \end{definition}
    A space $X$ is connected if and only if the only subsets of $X$ that are clopen in $X$ are the empty set and $X$ itself.
    
    \begin{lemma}[Lemma 23.1]
        If $Y$ is a subspace of $X$, a separation of $Y$ is a pair of disjoint nonempty sets $A$ and $B$ whose union is $Y$, neither
        of which contains a limit point of the other (i.e. $\overline{A}\cap B = \emptyset$ and $A\cap\overline{B}=\emptyset$). 
        The space $Y$ is connected if there exists no separation of $Y$.
    \end{lemma}

    The following is an example of a subspace of $\mathbb{R}^2$ which is not connected.
    \begin{example}[Page 149]
        Consider the following subset of the plane $\mathbb{R}^2$:
        \[ X = \{ x\times y: y = 0\} \cup \{x\times y: x > 0\:\text{and}\:y=\nicefrac{1}{x}\} \]
        Then $X$ is not connected; indeed, the two indicated sets form a separation of $X$ because neither contains a limit
        point of the other. Refer to figure 23.1 on page 149.
    \end{example}

    \begin{lemma}[Lemma 23.2]
        If the sets $C$ and $D$ form a separation of $X$, and if $Y$ is a connected subspace of $X$, then $Y$ lies entirely
        within $C$ or $D$.
    \end{lemma}

    \begin{theorem}[Theorem 23.3]
        The union of a collection of connected subspaces of $X$ that a point in common is connected.
    \end{theorem}

    \begin{theorem}[Theorem 23.4]
        Let $A$ be a connected subspace of $X$. If $A\subseteq B\subseteq\overline{A}$, then $B$ is also connected.
    \end{theorem}
    \begin{proof}
        Let $A$ be connected and let $A\subseteq B\subseteq\overline{A}$. Suppose $B = C\cup D$ is a separation of $B$. By Lemma 23.2,
        the set $A$ must lie entirely within in $C$ or $D$. Suppose it lies within $C$, so $A\subseteq C$. Then,
        $\overline{A}\subseteq\overline{C}$. By Lemma 23.1, we know $\overline{C}\cap D=\emptyset$ and more specifically, since
        $B\subseteq\overline{A}$, we know that $B\cap D=\emptyset$ implying $D$ is empty which is a contradiction.
    \end{proof}

    \begin{theorem}[Theorem 23.5]
        The image of a connected space under a continuous map is connected.
    \end{theorem}

    \begin{theorem}
        A finite cartesian product of connected spaces is connected.
    \end{theorem}
    \begin{proof}
        We prove the theorem first for the product of two conncted spaces $X$ and $Y$. Choose a base point $a\times b$ in the product
        $X\times Y$. Note that the horizontal slice $X\times b$ is connected (homeomorphic to $X$) and similarly $x\times Y$ is
        connected. As a result, each T-shaped space
        \[ T_x = (X\times b) \cup (x\times Y) \]
        is connected since they have the point $x\times b$ in common. Now form union $\bigcup_{x\in X} T_x$ of all these
        T-shaped spaces. This union is connected because each T-shaped space contains $a\times b$ so that the union has
        $a\times b$ in common. Since this union equals $X\times Y$, the space $X\times Y$ is connected.
    \end{proof}

    \newpage

    \textbf{Chapter 3: Section 24}\\

    An arbitrary product of connected spaces is connected in the product topology but not in the box topology.

    \begin{definition}
        Given points $x$ and $y$ of the space $X$, a \textbf{path} in $X$ from $x$ to $y$ is a continuous map $f:[a,b]\to X$ of some
        closed intervals in the real line into $X$, such that $f(a) = x$ and $f(b) = y$. A space $X$ is \textbf{path-connected} if
        every pair of points of $X$ can be joined by a path in $X$.
    \end{definition}

    Path connectedness implies connectedness.

    Continuous image of a path-connected space is path connected.

    \begin{example}
        Let $S$ denote the following subset of the plane.
        \[ S = \{x\times \sin{(\nicefrac{1}{x})}: 0<x\leq 1\}. \]
        Because $S$ is the image of the connected set $(0,1]$ under a continuous map, $S$ is connected. Therefore, its closure
        $\overline{S}$ in $\mathbb{R}^2$ is also connected. The set $\overline{S}$ is a classical example in topology called
        \textbf{topologist's sine curve}. It equals the union of $S$ and the vertical interval $0\times [-1,1]$.
    \end{example}

    \newpage

    \textbf{Chapter 3: Section 26}

    \begin{definition}
        A collection $\mathcal{A}$ of subsets of a space $X$ is said to \textbf{cover} $X$, or to be a \textbf{covering} of $X$,
        if the union of the elements of $\mathcal{A}$ is equal to $X$. It is called an \textbf{open covering} of $X$ if its elements
        are open subsets of $X$.
    \end{definition}

    \begin{definition}
        A space $X$ is said to be \textbf{compact} if every open covering $\mathcal{A}$ of $X$ contains a finite subcollection that
        also covers $X$.
    \end{definition}

    If $Y$ is a subspace of $X$, a collection $\mathcal{A}$ of subsets of $X$ is said to \textbf{cover} $Y$ if the union of its
    elements \textit{contains} $Y$.

    \begin{lemma}[Lemma 26.1]
        Let $Y$ be a subspace of $X$. Then $Y$ is compact if and only if every covering of $Y$ by sets open in $X$ contains a finite
        subcollection covering $Y$.
    \end{lemma}

    \begin{theorem}[Theorem 26.2]
        Every closed subspace of a compact space is compact.
    \end{theorem}
    \begin{proof}
        Let $Y$ be a closed subspace of the compact space $X$. Given a covering $\mathcal{A}$ of $Y$ by sets open in $X$, let us
        form an open covering $\mathcal{B}$ of $X$ adjoining to $\mathcal{A}$ the single open set $X-Y$, that is,
        \[ \mathcal{B} = \mathcal{A} \cup \{X-Y\}. \]
        Some finite subcollection of $\mathcal{B}$ covers $X$. If this subcollection contains the set $X-Y$, discard $X-Y$; otherwise,
        leave the subcollection alone. The resulting collection is a finite subcollection of $\mathcal{A}$ that covers $Y$.
    \end{proof}

    \begin{theorem}[Theorem 26.3]
        Every compact subspace of Hausdorff space is closed.
    \end{theorem}

    \begin{lemma}[Lemma 26.4]
        If $Y$ is a compact subspace of the Hausdorff space $X$ and $x_0$ is not in $Y$, then there exist disjoint open sets
        $U$ and $V$ of $X$ containing $x_0$ and $Y$, respectively.
    \end{lemma}

    \begin{theorem}[Theorem 26.5]
        The image of a compact space under a continuous map is compact.
    \end{theorem}

    \begin{theorem}[Theorem 26.6]
        Let $f: X\to Y$ be a bijective continuous function. If $X$ is compact and $Y$ is Hausdorff, then $f$ is a homeomorphism.
    \end{theorem}

    \begin{theorem}[Theorem 26.7]
        The product of finitely many compact space is compact.
    \end{theorem}

    \begin{theorem}[Theorem 27.3]
        A subspace $A$ of $\mathbb{R}^n$ is compact if and only if it is closed and is bounded in the euclidean metric $d$ or the
        square metric $\rho$.
    \end{theorem}
    \begin{proof}
        It will suffice to consider only the metric $\rho$; the inequalities
        \[ \rho(x,y) \leq d(x,y) \leq \sqrt{n}\rho(x,y) \]
        imply that $A$ is bounded under $d$ if and only if it is bounded under $\rho$.\\
        Suppose that $A$ is compact. Then by Theorem 26.3, it is closed ($\mathbb{R}^n$ Hausdorff). Consider the collection of open
        sets
        \[ \{ B_{\rho}(\mathbf{0},m): m\in\mathbb{Z_+}\}, \]
        whose union covers all of $\mathbb{R}^n$. Some finite subcollection covers $A$ ($A$ compact). It follows that
        $A\subseteq B_{\rho}(\mathbf{0},M)$ for some $M$. Therefore, for any two points $x$ and $y$ of $A$, we have $\rho(x,y)\leq2M$.
        Thus $A$ is bounded under $\rho$ so that $A$ is cloesd and bounded.\\
        Conversely, suppose that $A$ is closed and bounded under $\rho$; suppose that $\rho(x,y)\leq N$ for every pair $x,y$ of points
        of $A$. choose a point $x_0$ of $A$, and let $\rho(x_0,\mathbf{0}) = b$. The triangle inequality implies that 
        $\rho(x,\mathbf{0}) \leq N + b$ for every $x\in A$. If $P = N + b$, then $A$ is a subset of the cube $[-P,P]^n$, which is
        compact. Being closed, $A$ is also compact.
    \end{proof}

    \begin{definition}
        Let $(X,d)$ be a metric space; let $A$ be a nonempty subset of $X$. For each $x\in X$, we define the \textbf{distance from $x_0$
            to $A$} by the equation
        \[ d(x,A) =\:\text{inf}\{d(x,a): a\in A\} \]
    \end{definition}

    \newpage

    \textbf{Chapter 3: Section 30}

    \begin{definition}
        A space $X$ is said to have a \textbf{countable basis at $x$} if there is a countable collection $\mathbf{B}$ of neighborhood
        of $x$ such that each neighborhood of $x$ contains at least one of the elements of $\mathcal{B}$ or equivalently, for every
        open neighborhood $U$ of $x$, there exists an element of the countable collection contained in $U$. A space that has a countable
        basis at each of its points is said to satisfy the \textbf{first countability axiom}, or to be \textbf{first-countable}.
    \end{definition}

    \begin{theorem}[Theorem 30.1]
        Let $X$ be a topological space.
        \begin{enumerate}
            \item Let $A$ be a subset of $X$. If there is a sequence of points of $A$, converging to $x$, then $x\in\overline{A}$;
                the converse holds if $X$ is first-countable.
            \item Let $f:X\to Y$. If $f$ is continuous, then for every convergent sequence $x_n\to x$ in $X$, the sequence $f(x_n)$
                converges to $f(x)$. The converse holds if $X$ is first-countable.
        \end{enumerate}
    \end{theorem}

    \begin{definition}
        IF a space $X$ has a countable basis for its topology, then $X$ is said to satisfy the \textbf{second countability axiom},
        or to be \textbf{second-countable}.
    \end{definition}
    Note not every metrizable space is second-countable. Every second countable space is Lindelöf. The product of two Lindelöf spaces
    need not be Lindelöf. The counter-example to this is the Sorgenfrey plane i.e. the topology generated by the basis which has
    sets of the form $[a,b)\times [c,d)$. The lower limit topology is first-countable but not second.

    \begin{theorem}[Theorem 30.2]
        A subspace of a first-countable space is first-countable, and a countable product of first-countable spaces is first-countable.
        A subspace of a second-countable space is second-countable, and a countable product of second-countable spaces is
        second-countable.
    \end{theorem}

    \newpage

    \textbf{Chapter 3: Section 31}

    \begin{definition}
        Suppose that one point sets are closed in $X$. Then $X$ is said to be \textbf{regular} if for each pair consisting of a point
        $x$ and a closed set $B$ disjoint from $x$, there exist disjoint open sets containing $x$ and $B$, respectively. The space $X$
        is said to be \textbf{normal} if for each pair $A,B$ of disjoint closed sets of $X$, there exist disjoint open sets
        containing $A$ and $B$, respectively.
    \end{definition}

    The space $\mathbb{R}_K$ is Hausdorff but not regular. The space $\mathbb{R}_l$ is normal. The Sorgenfrey plane is not normal but
    is regular (this space is the product of two regular spaces namely $\mathbb{R}_l\times \mathbb{R}_l$). Thus, the cartesian
    product of two normal spaces need not be normal.

    \begin{lemma}[Lemma 31.1]
        Let $X$ be a topological space. Let one points sets in $X$ be closed.
        \begin{enumerate}
            \item $X$ is regular if and only if given a point $x$ of $X$ and a neighborhood $U$ of $x$, there is a neighborhood
                $V$ of $x$ such that $\overline{V}\subseteq U$.
            \item $X$ is normal if and only if given a closed set $A$ and an open set $U$ containing $A$, there is an open set $V$
                containing $A$ such that $\overline{V}\subseteq U$.
        \end{enumerate}
    \end{lemma}

    \begin{theorem}[Theorem 31.2]
        \begin{enumerate}
            \item A subspace of a Hausdorff space is Hausdorff; a product of Hausdorff spaces is Hausdorff.
            \item A subspace of a regular space is regular; a product of regular spaces is regular.
        \end{enumerate}
    \end{theorem}

    \begin{definition}
        A \textbf{T}$_0$ or \textbf{Kolmogrov} space is defined to be a space where if $x,y\in X$, $x\neq y$, then there exists
        a neighborhood of $x$ such that $y\not\in U$ or vice versa.\\
        A \textbf{T}$_1$ or \textbf{Frechét} space is defined to be a space where if $x,y\in X$, then there exists $U,V$ open sets
        such that $x\in U$ and $y\in V$ but $x\not\in V$ and $y\not\in U$.
    \end{definition}

    \newpage

    \textbf{Chapter 3: Section 32}

    Arbitrary products of normal spaces is not normal.

    \begin{theorem}[Theorem 32.1]
        Every regular space with a countable basis is normal.
    \end{theorem}

    \begin{theorem}[Theorem 32.2]
        Every metrizable space is normal.
    \end{theorem}
    \begin{proof}
        Let $X$ be a metrizable space with metric $d$. Let $A$ and $B$ be disjoint closed subsets of $X$. For each $a\in A$, choose
        $\epsilon_a$ so that the ball $B(a,\epsilon_a)$ does not intersect $B$. Similarly, for each $b\in B$, choose $\epsilon_b$
        so that the ball $B(b,\epsilon_b)$ does not intersect $A$. Define
        \[ U = \bigcup_{a\in A} B(a,\nicefrac{\epsilon_a}{2})\qquad\text{and}\qquad V=\bigcup_{b\in B}B(b,\nicefrac{\epsilon_b}{2}) \]
        Then $U$ and $V$ are disjoint open sets containing $A$ and $B$, respectively; we assert they are disjoint. For if,
        $z\in U\cap V$, then
        \[ z\in B(a,\nicefrac{\epsilon_a}{2}) \cap B(b,\nicefrac{\epsilon_b}{2}) \]
        for some $a\in A$ and $b\in B$. The triangle inequality applies to show that $d(a,b) < \nicefrac{(\epsilon_a+\epsilon_b)}{2}$.
        If $\epsilon_a\leq \epsilon_b$, then $d(a,b)<\epsilon_b$ so that the ball $B(b,\epsilon_b)$ contains $a$ or vice versa, if
        $\epsilon_b\leq \epsilon_a$, then $d(a,b)<\epsilon_a$, so that the ball $B(a,\epsilon_a)$ contains the point $b$. Neither
        situation is possible.
    \end{proof}

    \begin{theorem}[Theorem 32.3]
        Every compact Hausdorff space is normal.
    \end{theorem}


\end{document}
