\documentclass{article}
\usepackage[utf8]{inputenc}
\usepackage[english]{babel}
\usepackage{amsfonts}
\usepackage{amsthm}
\usepackage{amsmath}
\usepackage{amssymb}

\newtheorem{theorem}{Theorem}
\newtheorem{es}{Examples}

\newcommand{\inter}[1]{int(#1)}

\title{The Product Metric}
\author{Anmol Bhullar}

\begin{document}
    \maketitle
    The \textit{natural} metric for a product of finite sequence of metric spaces is called the \textit{product metric}. We define what this
    is mention why it is the most natural of other choices.\\

    If $(X_1,d_1),(X_2,d_2),\hdots,(X_n,d_n)$ is a finite sequence of metric spaces and $N$ is the \textit{Euclidean norm}, then:
    \[ \big{(}X_1 \times X_2 \times \hdots \times X_n,N(d_1,\hdots,d_n)\big{)} \] 
    is a metric space called the \textbf{product metric}. This is defined by
    \[ N(d_1,\hdots,d_n)\big{(}(x_1,\hdots,x_n),(y_1,\hdots,y_n)\big{)} = N(d_1(x_1,y_2),\dots,d_n(x_n,y_n)) \]

    This actually induces the \textit{product topology} making it the most natural choice for a product metric.
\end{document}
