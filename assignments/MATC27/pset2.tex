\documentclass[12pt]{article}
\usepackage{amsmath} 
\usepackage{amsthm} % Theorem Formatting
\usepackage{amssymb}    % Math symbols such as \mathbb
\usepackage{mathrsfs}
\usepackage{graphicx} % Allows for eps images
\graphicspath{ {images/} }
\usepackage[dvips,letterpaper,margin=1in,bottom=0.7in]{geometry}
\usepackage{tensor}
 % Sets margins and page size
\usepackage{amsmath}

\renewcommand{\labelenumi}{(\alph{enumi})} % Use letters for enumerate
% \DeclareMathOperator{\Sample}{Sample}
\let\vaccent=\v % rename builtin command \v{} to \vaccent{}
\usepackage{enumerate}
\renewcommand{\v}[1]{\ensuremath{\mathbf{#1}}} % for vectors
\newcommand{\gv}[1]{\ensuremath{\mbox{\boldmath$ #1 $}}}
% for vectors of Greek letters
\newcommand{\uv}[1]{\ensuremath{\mathbf{\hat{#1}}}} % for unit vector
\newcommand{\abs}[1]{\left| #1 \right|} % for absolute value
\newcommand{\avg}[1]{\left< #1 \right>} % for average
\let\underdot=\d % rename builtin command \d{} to \underdot{}
\renewcommand{\d}[2]{\frac{d #1}{d #2}} % for derivatives
\newcommand{\dd}[2]{\frac{d^2 #1}{d #2^2}} % for double derivatives
\newcommand{\pd}[2]{\frac{\partial #1}{\partial #2}}
% for partial derivatives
\newcommand{\pdd}[2]{\frac{\partial^2 #1}{\partial #2^2}}
% for double partial derivatives
\newcommand{\pdc}[3]{\left( \frac{\partial #1}{\partial #2}
 \right)_{#3}} % for thermodynamic partial derivatives
\newcommand{\ket}[1]{\left| #1 \right>} % for Dirac bras
\newcommand{\bra}[1]{\left< #1 \right|} % for Dirac kets
\newcommand{\braket}[2]{\left< #1 \vphantom{#2} \right|
 \left. #2 \vphantom{#1} \right>} % for Dirac brackets
\newcommand{\matrixel}[3]{\left< #1 \vphantom{#2#3} \right|
 #2 \left| #3 \vphantom{#1#2} \right>} % for Dirac matrix elements
\newcommand{\grad}[1]{\gv{\nabla} #1} % for gradient
\let\divsymb=\div % rename builtin command \div to \divsymb
\renewcommand{\div}[1]{\gv{\nabla} \cdot \v{#1}} % for divergence
\newcommand{\curl}[1]{\gv{\nabla} \times \v{#1}} % for curl
\let\baraccent=\= % rename builtin command \= to \baraccent
\renewcommand{\=}[1]{\stackrel{#1}{=}} % for putting numbers above =
\providecommand{\wave}[1]{\v{\tilde{#1}}}
\providecommand{\fr}{\frac}
\providecommand{\RR}{\mathbb{R}}
\providecommand{\NN}{\mathbb{N}}
\providecommand{\QQ}{\mathbb{Q}}
\providecommand{\seq}{\subseteq}
\providecommand{\e}{\epsilon}
\providecommand{\T}{\mathcal{T}}

\newtheorem{prop}{Proposition}
\newtheorem{thm}{Theorem}[section]
\newtheorem{axiom}{Axiom}[section]
\newtheorem{p}{Problem}[section]
\usepackage{cancel}
\newtheorem*{lem}{Lemma}
\theoremstyle{definition}
\newtheorem*{dfn}{Definition}
 \newenvironment{s}{%\small%
        \begin{trivlist} \item \textbf{Solution}. }{%
            \hspace*{\fill} $\blacksquare$\end{trivlist}}%
% ***********************************************************
% ********************** END HEADER *************************
% ***********************************************************

\begin{document}

{\noindent\Huge\bf  \\[0.5\baselineskip] {\fontfamily{cmr}\selectfont  %
Problem Set 2}         }\\[2\baselineskip] % Title
{ {\bf \fontfamily{cmr}\selectfont MATC27: Introduction to Topology}\\ {\textit{\fontfamily{cmr}%
\selectfont September 29, 2017, 1002678140}}}
{\large \textsc{Anmol Bhullar}} % Author name
\\[1.4\baselineskip]

\begin{lem}
    Let $\mathcal{B}$ be a basis for the set $X$. Then the topology that $\mathcal{B}$ generates is unique i.e. if $\mathcal{B}$
    generates two topologies $\T$ and $\T^{'}$, then $\T = \T^{'}$.
\end{lem}
\begin{s}
    Since $\mathcal{B}$ generates $\T$, then
    \[ \T = \{\cup\: \mathcal{A}: \mathcal{A}\subseteq\mathcal{B}\} \]
    but this must also be true for $\T^{'}$ since $\mathcal{B}$ also generates $\T^{'}$. So,
    \[ \T^{'} = \{\cup\: \mathcal{A}: \mathcal{A}\subseteq\mathcal{B}\} = \T \implies \T = \T^{'} \]
\end{s}

\begin{p}
    Let $X\neq 0$. Suppose that $\mathcal{B}$ and $\mathcal{B}^{'}$ are bases on $X$. Prove that $\mathcal{B}$ and $\mathcal{B}^{'}$
    are equivalent if and only if
    \begin{enumerate}
        \item[(a)] $\forall\: B\in\mathcal{B}$ and $\forall\:x\in B$, $\exists\:B^{'}\in\mathcal{B}^{'}$ such that $x\in B^{'}\subseteq B$, and
        \item[(b)] $\forall\: B^{'}\in\mathcal{B}^{'}$ and $\forall\:x\in B^{'}$, $\exists\:B\in\mathcal{B}$ such that $x\in B\subseteq B^{'}$
    \end{enumerate}
\end{p}
\begin{s}
    First, we prove the $\implies$ direction.\\
    Assume $\mathcal{B}$ and $\mathcal{B}^{'}$ are equivalent. Then the topologies they generate are equal i.e. $\T_{\mathcal{B}} = 
    \T_{\mathcal{B}^{'}}$. This implies $\T_{\mathcal{B}} \subseteq \T_{\mathcal{B}^{'}}$ so that we fulfill condition (1) of
    Lemma 13.3 (Munkres, Topology pg. 81), so we get that condition (2) (of the same Lemma) holds true for the bases $\mathcal{B}^{'}$ and
    $\mathcal{B}$. Choose arbitrary $B\in\mathcal{B}$, and subsequently any $x\in B$, then by condition (2) of Lemma 13.3, we get that
    the existence of a set $B^{'}\in\mathcal{B}^{'}$ such that $x\in B^{'}\subseteq B$. This is equivalent to condition (a) of this question.
    Thus (a) holds. (b) is proved a similar way if we notice that $\T_{\mathcal{B}} = \T_{\mathcal{B}^{'}} \implies \T_{\mathcal{B}^{'}} \subseteq
    \T_{\mathcal{B}}$.\\
    Now, we prove the $\impliedby$ direction.\\
    If we assume (a) holds, then we fulfill condition (2) of Lemma 13.3 which tells us that $\T_{\mathcal{B}^{'}}$ is finer than $\T$, 
    which then implies that $\T_{\mathcal{B}} \subseteq \T_{\mathcal{B}^{'}}$. 
    Similarly, if we assume (b) holds, then Lemma 13.3 tells us that $\T_{\mathcal{B}}$ is finer than $\T_{\mathcal{B}}$ 
    i.e. $\T_{\mathcal{B}^{'}} \subseteq \T_{\mathcal{B}}$.
    Combining these two inequalities, we obtain that $\T_{\mathcal{B}} = \T_{\mathcal{B}^{'}}$. Since $\mathcal{B}$ generates $\T_{\mathcal{B}}$,
    this implies that $\mathcal{B}$ generates
    $\T_{\mathcal{B}^{'}}$ and since $\mathcal{B}^{'}$ also generates $\T_{\mathcal{B}^{'}}$, we have that the two bases generate the same topology
    and so by definition of the equivalency of bases, we get that $\mathcal{B}$ and $\mathcal{B}^{'}$ are equivalent.
\end{s}

\newpage

\begin{p}
    Let $\mathscr{S}$ be a subbasis for $(X,\T)$. Let $Y\subseteq X$ be non-empty. Prove that the set
    \[ \mathscr{S}_Y = \{Y\cap S: S\in\mathscr{S}\} \]
    is a subbasis for the subspace $(Y,\T_Y)$.
\end{p}
\begin{s}
    By definition of a subbasis, it is enough to show that $\cup\:(\mathscr{S}_Y) = Y$. Consider:
    \[ Y = Y \cap X = Y \cap (\cup\:\mathscr{S}) = \bigcup_{S\in\mathscr{S}} (Y \cap S) = \cup\:\mathscr{S}_Y\]
    as wanted.
\end{s}

\begin{p}
    Let $X = \{a,b,c,d,e,f\}$ be equipped with the topology
    \[ \T_X = \{X,\emptyset,\{a\},\{c,d\},\{a,c,d\},\{b,c,d,e,f\} \]
    Let $Y = \{b,c,e\} \subset X$. Find $\T_Y$. Justify.
\end{p}
\begin{s}
    By definition of the subspace topology, 
    $T_Y = \{Y\cap S: S\in \T_X\}$ so to find $\T_Y$, it is enough to find the intersection of $\{b,c,e\}$ with every element
    of $\T_X$. Note:
    \begin{align*}
        Y \cap X = Y \quad Y \cap \{a\} = \emptyset \quad Y \cap \{c,d\} = \{c\} \quad Y \cap \{a,c,d\} = \{c\} \\
        Y \cap \{b,c,d,e,f\} = \{b,c,e\} = Y
    \end{align*}
    So $\T_Y = \{\emptyset, Y, \{c\}\}$. Furthermore, since every subspace
    topology is a topology, we get for free that the $\T_Y$ we calculated is a topology.
\end{s}

\begin{p}
    Consider $\RR$ with the standard topology. Determine whether or not each of the following subsets of the subspace $I = [0,1]$ are open
    in $I$. Justify.
    \begin{enumerate}
        \item[(a)] $(0.5,1]$
        \item[(b)] $(\frac{1}{2}, \frac{2}{3})$
        \item[(c)] $(0,\frac{1}{2}]$
    \end{enumerate}
\end{p}
\begin{s}
    To show a set $V$ is open in $I$, we will try to find an open set $U$ in $\RR$ such that $U\cap I = V$. This is enough to show that $V$ is 
    open in $I$
    because by the definition of a subspace topology, all open sets of $I$ can be written as the intersection of $I$ and an open set of
    $\RR$. Likewise, to show a set $V$ is not open in $I$, we will show that there is no open set $U$ of $\RR$ such that $V = U \cap I$.
    \begin{enumerate}
        \item[(a)] Note $(0.5,1.5)$ is an open interval so it is open in $\RR$ and furthermore $(0.5,1.5)\cap I = (0.5,1]$ so $(0.5,1]$ is
            open in $I$.
        \item[(b)] Similarly to (a), $(\frac{1}{2},\frac{2}{3})$ is open in $\RR$ and note that $(\frac{1}{2},\frac{2}{3})\cap I = 
            (\frac{1}{2},\frac{2}{3})$ so $(\frac{1}{2},\frac{2}{3})$ is open in $I$.
        \item[(c)] We claim that this is not open in $I$. To show that it is not, assume the opposite. So, $(0,\frac{1}{2}]$ is open
            in $I$, and thus there exists an open set $U$ in $\RR$ such that $(0,\frac{1}{2}] = U \cap I$. Note $(0,\frac{1}{2})\subsetneq I$
            so $U$ must be a subset of $I$, thus $U \cap I = U$. Thus $(0,\frac{1}{2}] = U$ which implies $(0,\frac{1}{2}]$ is open in $\RR$.
            However, this is not an open subset of $\RR$ since $(0,\frac{1}{2}]$ is not an open interval implying that it cannot be an open
            set of $\RR$. Thus $U$ does not exist which implies $(0,\frac{1}{2}]$ is not open in $I$.
    \end{enumerate}
\end{s}

\begin{p}
    Consider $\RR$ with the standard topology. Prove the relative topology on $\mathbb{Z}$ is the discrete topology.
\end{p}
\begin{s}
    Choose $z\in\mathbb{Z}$. Note that $(z-1,z+1)$ is open in $\RR$ and that $(z-1,z+1)\:\cap\:\mathbb{Z} = \{z\}$ so that for any
    $z\in\mathbb{Z}$, we have that $\{z\}$ is open in the relative topology on $\mathbb{Z}$. We know that the relative topology is always a topology
    so the relative topology on $\mathbb{Z}$ (hereby denoted via $\T$) must follow the three axioms for a topology. Furthermore, note that if
    $X$ is an arbitrary subset of $\mathbb{Z}$, then we can form $X$ by unioning all of the elements of $X$ i.e. for all $x\in X$,
    $\{x\}\in\T$ since $x$ is just an integer, thus $\cup_{x\in X} x = X$ which implies $X\in\T$ since it can be written as the union of elements
    of $\T$ (axiom 2 of a topology). Since an arbitrary subset is an element of the set $P(\mathbb{Z})$ (power set of $\mathbb{Z}$), we have that
    $P(\mathbb{Z}) \subseteq \T$. However, since $\T$ is a topology on $\mathbb{Z}$, $\T$ is defined to be a set which contains only the subsets
    of $\mathbb{Z}$ so we get for free that $\T \subseteq P(\mathbb{Z})$ which implies $P(\mathbb{Z}) = \T$ which implies that $\T$ is the discrete
    topology (by the definition of a discrete topology).
\end{s}

\begin{p}
    Let $X_1,\hdots,X_n$ be a finite collection of topological spaces. Suppose that $A_i$ is a closed subset of $X_i$, for $i=1,\hdots,n$. Prove
    that $A_1\times\hdots\times A_n$ is a closed subset of $X_1\times\hdots\times X_n$.
\end{p}
\begin{s}
    If we want to show that $\prod_{i=1}^n A_i$ is a closed subset of $\prod_{i=1}^n X_i$ then by definition of a closed subset, we have to show
    that $(\prod_{i=1}^n X_i) - (\prod_{i=1}^n A_i)$ is open in $\prod_{i=1}^n X_i$. To do this, first recall the set difference of cartesian
    product formula:
    \begin{align*}
        \prod_{i=1}^n X_i - \prod_{i=1}^n A_i &= (X_1-A_1,X_2,\hdots,X_n) \cup (X_1, X_2-A_2, X_3, \hdots, X_n)\: \cup \\
            &(X_1, X_2, X_3-A_3, X_4, \hdots, X_n) \cup \hdots \cup (X_1,\hdots,X_{n-1},X_n-A_n)
    \end{align*}
    Since $A_i$ for $i\in\{1,\hdots,n\}$ is a closed subset of $X_i$, then $X_i - A_i$ is open. Furthermore, since $X_i$ is a topological space,
    we have that $X_i$ is also open. Thus $\prod_{i=1}^n X_i - \prod_{i=1}^n A_i$ is equal to the union of open sets, and this is equal to an open
    set since $\prod_{i=1}^n X_i$ is a topological space, so the union of open sets of $\prod_{i=1}^n X_i$ must be open. Therefore, we have that
    $\prod_{i=1}^n X_i - \prod_{i=1}^n A_i$ is open so by the definition of a closed set, $\prod_{i=1}^n A_i$ must then be a closed set of
    $\prod_{i=1}^n X_i$ as we wanted.

\end{s}

\newpage

\begin{p}
    Let $\{X_{\alpha}\}_{\alpha\in J}$ be an indexed family of topological spaces. Suppose that each $X_{\alpha}$ is equipped with the trivial
    topology. Prove that the product space $\prod_{\alpha\in J}X_{\alpha}$ must have the trivial topology.
\end{p}
\begin{s}
    We are given that $\prod_{\alpha\in J} X_{\alpha}$ is a product space, thus by definition of a product topology, we have that the subbasis for
    the product space $\prod_{\alpha\in J} X_{\alpha}$ is equal to $\mathcal{S} = \cup_{\beta\in J} \mathcal{S}_{\beta}$ where each $\mathcal{S}_{\beta}$
    is defined to be the set $\{\pi_{\beta}^{-1}(U): U_{\beta}$ is open in $X_{\beta}\}$. To show that $\prod_{\alpha\in J} X_i$ is the trivial topology,
    we will show each $\mathcal{S}_{\beta} = \{\emptyset, \prod_{\alpha\in J} X_{\alpha}\}$, so $\cup_{\mathcal{\beta\in J}} \mathcal{S}_{\beta} = 
    \{\emptyset, \prod_{\alpha\in J} X_{\alpha}\}$ and show this subbasis generates the trivial topology.\\
    First, show $\mathcal{S}_{\beta} = \{\emptyset, \prod_{\alpha\in J} X_{\alpha}\}$. To do this, note that the only sets which are open in $X_{\beta}$
    is the empty set and $X_{\beta}$ itself (follows since $X_{\beta}$ is given as the trivial topology) so $\mathcal{S}_{\beta} = \{\pi_{\beta}^{-1}(\emptyset),
    \pi_{\beta}^{-1}(X_{\beta})\}$. Note that $\pi_{\beta}^{-1}(\emptyset) = \emptyset$ follows vacuously since there are no sets which map to the emptyset, and
    furthermore, $\pi_{\beta}^{-1}(X_{\beta}) = \prod_{\alpha\in J}X_{\alpha}$ since the only non empty set in each coordinate $\alpha\in J$ is $X_{\alpha}$ (since
    each $X_{\alpha}$ is simply the trivial topological space). So, we have that $\mathcal{S}_{\beta} = \{\emptyset, \prod_{\alpha\in J} X_{\alpha}\}$.\\
    Since this holds for all $\beta\in J$ (follows since our choice of $\beta$ was arbitrary), we have that our subbasis is 
    $\cup_{\beta\in J} \mathcal{S}_{\beta} = \mathcal{S} = \{\emptyset,\prod_{\alpha\in J} X_{\alpha}\}$. By our choice of $\mathcal{S}$, we know that
    $\mathcal{S}$ generates the product topology on $\prod_{\alpha\in J} X_{\alpha}$. Since we have already calculated our subbasis, we can calculate this
    product topology explicitly via finding the collection of all unions of finite intersections of elements of $\mathcal{S}$. Since the elements of
    $\mathcal{S}$ are simply $\emptyset$ and $\prod_{\alpha\in J} X_{\alpha}$, we have that this collection is equal to $\{\emptyset, \prod_{\alpha\in J} X_{\alpha}\}$.
    Thus the topology generated by $\mathcal{S}$ is equal to the trivial topology. So, we have that the product topology on $\prod_{\alpha\in J} X_{\alpha}$ is equal
    to the trivial topology as we wanted.
\end{s}

\begin{p}
    Let $A_{\alpha}$ be a subspace of $X_{\alpha}$, for each $\alpha\in J$. Then $\prod\:A_{\alpha}$ is a subspace of $\prod\:X_{\alpha}$ if both
    products are given the box topology.
\end{p}
\begin{s}
    We want to show that $\prod A_{\alpha} = \{W \cap \prod A_{\alpha}: W$ open in $\prod X_{\alpha}\}$. This is equivalent to showing that any element
    $V$ of $\prod A_{\alpha}$ is of the form $W \cap \prod A_{\alpha}$ (where $W$ is open in $\prod X_{\alpha}$).\\
    Therefore, choose an arbitrary $V \in \prod A_{\alpha}$. We know that $\prod A_{\alpha}$ has the box topology on it. Thus, there exists a basis
    $U = \{U_{\beta\in J}\}$ of $\prod A_{\alpha}$ where each $U_{\beta}$ is the cross product of open sets of each $A_{\alpha}$. Then $V$ can be written
    as the union of some basis elements of $U$ i.e. $V = \cup_{\gamma\in J^{'}} U_{\gamma}$ where $J^{'}\subseteq J$ and additionally, 
    since $\pi_{\alpha}(\cup_{\gamma\in J^{'}} U_{\gamma})$ is open, we obtain that $\pi_{\alpha}(V) = V_{\alpha}$ is an open set of $A_{\alpha}$. Since, we are
    given that $A_{\alpha}$ is a subspace of $X_{\alpha}$, we obtain that $V_{\alpha} = W_{\alpha} \cap A_{\alpha}$ where $W_{\alpha}$ is open in $X_{\alpha}$.
    Thus, we can write $V$ as the product $\prod\: (W_{\alpha} \cap A_{\alpha})$. Recalling the formula for a cartesian product of intersections, we obtain that
    $V = \prod W_{\alpha} \cap \prod A_{\alpha}$. Since $W_{\alpha}$ was open in $X_{\alpha}$ and we are given the box topology on $X_{\alpha}$, it follows that
    $\prod W_{\alpha}$ is open in $\prod X_{\alpha}$. Thus, we can write any open set $V$ of $\prod A_{\alpha}$ as the intersection $\prod W_{\alpha} \cap 
    \prod A_{\alpha}$, thus obtaining that $\prod A_{\alpha}$ is a subspace of $\prod X_{\alpha}$ as we wanted.
\end{s}

\end{document}
