\documentclass{article}
\usepackage[utf8]{inputenc}
\usepackage[english]{babel}
\usepackage{amsfonts}
\usepackage{amsthm}
\usepackage{amsmath}
\usepackage{amssymb}

\newtheorem{theorem}{Theorem}
\newtheorem{es}{Examples}

\newcommand{\inter}[1]{int(#1)}

\title{MATC27 Assignment 4}
\author{Anmol Bhullar - 1002678140}

\begin{document}
    \textbf{Question 7}.\\
    Assume $A\subseteq X$ is closed in $X$. We show that if $(x_n)$ is a convergent sequence (converging to $x\in X$) in $A$ (i.e. $(x_n)\subseteq A$), 
    then $x\in X$. We know that for any set $A\subseteq X$, $\overline{A} = A \cup \partial{A}$ (where $\partial{A}$ is the set of accumulation points of $A$).
    However, since $A$ is closed, we know that $\overline{A} = A$. Thus, $A = A \cup \partial{A}$ which is enough to imply that $\partial{A} \subseteq A$.
    It is left to show that $x$ is an accumulation point of $A$. By definition of an accumulation point, $x$ is an accumulation point of $A$ if and only if
    every open neighbourhood of $x$ contains at least one point of $A$ different from $x$. Since $(x_n)\to x$, we know that for all $\epsilon>0$, there exists a
    natural number $N > 0$ such that for all natural numbers $n > N$, $|x_n - x| < \epsilon$ ($X$ is metrizable, so we can use this definition). Additionally,
    since $X$ is metrizable, every neighbourhood is given by $B_d(x,\epsilon_1)$ where $d$ is the metric that induces the topology on $X$ and $\epsilon_1>0$ is arbitrary.
    Thus, it is left to show $B_d(x,\epsilon_1) - \{x\} \cap A \neq \emptyset$. Fix $\epsilon_1 > 0$. Then $(x_n)\to x$ tells us that there exists $n\in\mathbb{N}$
    such that $|x_n - x| < \epsilon$. Thus, $x_n \in B_d(x,\epsilon_1)$. This is enough to imply $B_d(x,\epsilon_1) - \{x\} \cap A \neq \emptyset$.\\
    
    Now, assume that every convergent sequence $(x_n)$ in $A$ converges to a point in $A$. We show that $A$ is closed. Using similar reasoning from the above paragraph,
    we know that if $(x_n)\to x$, then $x\in\partial{A}$. Since, we are given $x\in A$, this is enough to imply that $\partial{A}\subseteq A$. From this, we can
    obtain the result $A = A \cup \partial{A}$ and since we know $\overline{A} = A \cup \partial{A}$, then $A = \overline{A}$ which is enough to imply that $A$ is
    closed in $X$ as wanted.\hfill$\blacksquare$
\end{document}
