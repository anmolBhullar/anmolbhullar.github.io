\documentclass[12pt]{article}
\usepackage{amsmath} 
\usepackage{amsthm} % Theorem Formatting
\usepackage{amssymb}    % Math symbols such as \mathbb
\usepackage{graphicx} % Allows for eps images
\usepackage[dvips,letterpaper,margin=1in,bottom=0.7in]{geometry}
\usepackage{tensor}
 % Sets margins and page size
\usepackage{amsmath}

\renewcommand{\labelenumi}{(\alph{enumi})} % Use letters for enumerate
% \DeclareMathOperator{\Sample}{Sample}
\let\vaccent=\v % rename builtin command \v{} to \vaccent{}
\usepackage{enumerate}
\renewcommand{\v}[1]{\ensuremath{\mathbf{#1}}} % for vectors
\newcommand{\gv}[1]{\ensuremath{\mbox{\boldmath$ #1 $}}}
% for vectors of Greek letters
\newcommand{\uv}[1]{\ensuremath{\mathbf{\hat{#1}}}} % for unit vector
\newcommand{\abs}[1]{\left| #1 \right|} % for absolute value
\newcommand{\avg}[1]{\left< #1 \right>} % for average
\let\underdot=\d % rename builtin command \d{} to \underdot{}
\renewcommand{\d}[2]{\frac{d #1}{d #2}} % for derivatives
\newcommand{\dd}[2]{\frac{d^2 #1}{d #2^2}} % for double derivatives
\newcommand{\pd}[2]{\frac{\partial #1}{\partial #2}}
% for partial derivatives
\newcommand{\pdd}[2]{\frac{\partial^2 #1}{\partial #2^2}}
% for double partial derivatives
\newcommand{\pdc}[3]{\left( \frac{\partial #1}{\partial #2}
 \right)_{#3}} % for thermodynamic partial derivatives
\newcommand{\ket}[1]{\left| #1 \right>} % for Dirac bras
\newcommand{\bra}[1]{\left< #1 \right|} % for Dirac kets
\newcommand{\braket}[2]{\left< #1 \vphantom{#2} \right|
 \left. #2 \vphantom{#1} \right>} % for Dirac brackets
\newcommand{\matrixel}[3]{\left< #1 \vphantom{#2#3} \right|
 #2 \left| #3 \vphantom{#1#2} \right>} % for Dirac matrix elements
\newcommand{\grad}[1]{\gv{\nabla} #1} % for gradient
\let\divsymb=\div % rename builtin command \div to \divsymb
\renewcommand{\div}[1]{\gv{\nabla} \cdot \v{#1}} % for divergence
\newcommand{\curl}[1]{\gv{\nabla} \times \v{#1}} % for curl
\let\baraccent=\= % rename builtin command \= to \baraccent
\renewcommand{\=}[1]{\stackrel{#1}{=}} % for putting numbers above =
\providecommand{\wave}[1]{\v{\tilde{#1}}}
\providecommand{\fr}{\frac}
\providecommand{\RR}{\mathbb{R}}
\providecommand{\NN}{\mathbb{N}}
\providecommand{\seq}{\subseteq}
\providecommand{\e}{\epsilon}

\newtheorem{prop}{Proposition}
\newtheorem{thm}{Theorem}[section]
\newtheorem{axiom}{Axiom}[section]
\newtheorem{p}{Problem}[section]
\usepackage{cancel}
\newtheorem*{lem}{Lemma}
\theoremstyle{definition}
\newtheorem*{dfn}{Definition}
 \newenvironment{s}{%\small%
        \begin{trivlist} \item \textbf{Solution}. }{%
            \hspace*{\fill} $\blacksquare$\end{trivlist}}%
% ***********************************************************
% ********************** END HEADER *************************
% ***********************************************************

\begin{document}

{\noindent\Huge\bf  \\[0.5\baselineskip] {\fontfamily{cmr}\selectfont  %
Assignment 1}         }\\[2\baselineskip] % Title
{ {\bf \fontfamily{cmr}\selectfont MATC37: Introduction to Real Analysis}\\ {\textit{\fontfamily{cmr}%
\selectfont January 19, 2017}}}
{\large \textsc{Anmol Bhullar}} % Author name
\\[1.4\baselineskip]

\begin{p}
    Explain why the empty set $\emptyset$ is both open and closed.
\end{p}
\begin{s}
    By definition of $\emptyset$, we know that there does \textit{not} exist any $x\in\mathbb{R}^d$ such that $x\in\emptyset$. 
    Therefore, every $x\in\emptyset$ has the property that for some $r>0$, $B_r(x)\subseteq\emptyset$. Thus, $\emptyset$ is open.
    Furthermore, choose any $x\in\mathbb{R}^d$ and $r>0$. It is clear that $B_r(x)\subseteq\mathbb{R}^d$. 
    Therefore, $\mathbb{R}^d$ is open. Since $(\mathbb{R}^d)^c = \emptyset$, then the empty set is closed by definition.\\
    Therefore, the empty set is both open and closed.
\end{s}

\begin{p}
    Prove that the unit disk $D := \{x\in\mathbb{R}^d: |x|\leq 1\}$ is closed.
\end{p}
\begin{s}
    To prove $D$ is closed, we show that $D^c$ is open. Thus, it is left to show that for any $x\in D^c$, there exists $r>0$ such that
    $B_r(x) \subseteq D^c$.\\
    Therefore, choose any $x\in D^c$ and choose $r = |x| - 1$. To show, $B_r(x)\subseteq D^c$, we choose $y\in B_r(x)$ and show
    $y\in D^c$ or equivalently, showing that $|y|>1$.\\
    We know,
    \begin{align*}
        |x-y| &< r \qquad\qquad\;\text{definition of}\;y\in B_r(x)\\
        \iff |x-y| &< |x| - 1 \qquad\text{definition of }\; r\\
        \iff 1 &< |x| - |x-y|
    \end{align*}
    Note, $|x| > |x-y|$ since $|x|>|x|-1 = r > |x-y|$ so that $|x|-|x-y|>0$. Therefore, by the reverse triangle inequality,
    we have that,
    \[ |x| - |x-y| = | |x| - |x-y| | \leq |x - y - x| = |y| \]
    Therefore, $1 < |y|$ as wanted.
\end{s}

\begin{p}
    Prove that if $E_1,\hdots,E_k\subseteq\mathbb{R}^d$ are closed sets, then their union $E_1\cup\hdots\cup E_k$ is also closed.
\end{p}
\begin{s}
    To show
    \[ E := \bigcup_{i=1}^k E_i \]
    is closed, we show its complement is open. By the DeMorgan laws,
    \[ E^c = (\bigcup_{i=1}^k E_i)^c = \bigcap_{i=1}^k (E_i)^c \]
    Since $E_i$ is closed, $(E_i)^c$ must be open. We know finite intersections of open sets are open. Thus, $\cap_{i=1}^k (E_i)^c$
    is open. Therefore, $(\cup_{i=1}^k E_i)^c$ is open. Therefore, $\cup_{i=1}^k E_i$ is closed as wanted.
\end{s}

\begin{p}
    Given an example of a countable collection of closed $E_1,E_2,\hdots\subseteq\mathbb{R}$ such that their union is not closed.
\end{p}
\begin{s}
    The following sequence of real numbers converges to $\pi$:
    \[ \{a_n\}_{n=1}^{\infty} := \{3,3.1,3.14,3.141,3.1415,\hdots\} \]
    Note each interval $[a_n,a_{n+1}]$ is closed, however,
    \[ \bigcup_{n=1}^{\infty} [a_n,a_{n+1}] = [3,\pi) \]
    which is not closed. To see why, suppose $\cup_{n=1}^{\infty} [a_n,a_{n+1}] = [3,\pi]$. Then, by definition of union,
    we know that $\pi\in[a_k,a_{k+1}]$ for some $k\in\mathbb{N}$, but this is not possible since the sequence $\{a_n\}$ is bounded
    above by $\pi$ so there does not exist any interval $[a_k,a_{k+1}]$ of which $\pi$ is an element of. Therefore, $\{[a_n,a_{n+1}]\}$
    is an example of a countable collection of closed sets such that their union is not closed.
\end{s}

\begin{p}
    Consider the annular region $A:= \{x\in\mathbb{R}^2: 1<|x|\leq 2\}\subseteq\mathbb{R}^2$. Find $\bar{A},A^{\circ},\partial{A}$.
\end{p}
\begin{s}
    Let $E = \{x\in\mathbb{R}^d: |x| = 1\}$ and $F = \{x\in\mathbb{R}^d: |x| = 2\}$. We claim that $E \cup F = \partial{A}$.
    First, choose any $x\in E$ and any $r>0$. Then, we want to show $B_r(x)\cap A\neq \emptyset$. We know there exists $y\in B_r(x)$
    such that $|y| > |x|$ since $r>0$. Since $|y| = |x| > 1$, then by definition $y\in A$. Therefore, $x$ is a boundary point
    of $A$. In the event, $|y| > 2$, then $B_{2}(x) \subseteq B_r(x)$, so instead choose some $y\in B_2(x)$ since, it will imply
    $y\in B_r(x)$. We can repeat this same process for $F$ but instead choosing some $|y| < |x|$. Therefore, $E \cup F$ contain
    boundary points of $A$ and we know $A$ does not have any other boundary points since 
    \[ E\cup F \cup A = \{x\in\mathbb{R}^d: 1\leq |x|\leq 2\} \]
    is clearly a closed set so that it contains \textit{all} of its boundary points. Therefore, $\partial{A} = E \cup F$.\\

    Claim: $\bar{A} = \{x\in\mathbb{R}^2: 1\leq|x|\leq 2\}$. It is clear that $A$ does not contain all of its boundary points. 
    Namely, it does not contain the boundary points, $E = \{x\in\mathbb{R}^d: |x| = 1\}$. 
    Since $\bar{A} = E \cup A$, we have that $\bar{A}$ is the closure of $A$. \\

    Claim: $A^{\circ} = \{x\in\mathbb{R}^2: 1<|x|<2\}$. This is most clearly seen through the fact that,
    \[ \bar{A} - \partial{A} = \{x\in\mathbb{R}^2: 1<|x|<2\} \]
    Equivalently, we can note that if there were some open set larger than $A^{\circ}$ contained in $A$, then it \textit{must} contain
    a boundary point. But, then this set cannot be open so there is a contradiction. Therefore, $A^{\circ}$ is the largest
    open set contained in $A$.
\end{s}

\newpage

\begin{p}
    Consider the set of rational numbers $\mathbb{Q}\subseteq\mathbb{R}$. Find $\bar{\mathbb{Q}},\mathbb{Q}^{\circ}$, and
    $\partial{\mathbb{Q}}$.
\end{p}

\begin{s}
    Claim: $\mathbb{Q}^{\circ} = \emptyset$. To see why, we claim that there exists no open ball $B_r(x)$ which contains
    \textit{only} rational numbers. Choose some $y\in B_r(x)$ such that $y\neq x$. Then either $y<x$ or $y>x$. Without loss
    of generlization, choose the first so that $y<x$. Then by densite of the irrationals, we know there exists some irrational
    number $z$ such that $y < z < x$. Since $d(x,y) < r$, then it must also be that $d(x,z) < r$. Thus, $z\in B_r(x)$ which is
    a contradiction. Thus, there exist no open balls in the set $\mathbb{Q}$ which implies $\mathbb{Q}^{\circ} = \emptyset$.\\

    Claim: $\bar{\mathbb{Q}} = \mathbb{R}$. Recall that every irrational number has an infinite decimal expansion. Specifically,
    this means, for every irrational number, we can construct a sequence of rational numbers converging to that irrational number.
    For an example of this sequence refer to problem 4. Thus, we see that every irrational number is a limit point of $\mathbb{Q}$.
    Since $\mathbb{Q}\cup\mathbb{R}-\mathbb{Q} = \mathbb{R}$, we see that there are no other limit points, thus $\mathbb{R}$ is the
    closure of $\mathbb{Q}$.\\

    It is clear that $\partial{\mathbb{Q}} = \mathbb{R}$ from the definition of the boundary of a set ($\mathbb{R}-\emptyset =
    \mathbb{R}$)
\end{s}

\begin{p}
    Let $\mathcal{C} = \{f: [0,1]\to\mathbb{R}: f\;\text{continuous}\}$ be the space of continuous functions on the unit interval
    and let
    \[ d(f,g) := \int_0^1 |f(x) - g(x)|dx\qquad(f,g\in\mathcal{C}) \]
    Prove that $d$ is a metric, i.e. prove that it is positive definite, symmetric, and satisfies the triangle inequality.
\end{p}

\begin{s}
    First we prove that $d$ is positive definite. It is clear that $|f(x)-g(x)|\geq 0$. Define $h(x) := |f(x) - g(x)|$. Then,
    it also clear that $\int_0^1 h(x)dx\geq 0$ since $h(x)\geq 0$ for all $x\in[0,1]$.\\

    Next, we prove that $d$ is symmetric. Note:
    \begin{align*}
        d(f,g) = \int_0^1 |f(x)-g(x)|dx = \int_0^1 |g(x) - f(x)|dx = d(g,f)
    \end{align*}
    as wanted.\\

    Finally, we prove the triangle inequality. Note for any $h\in\mathcal{C}$:
    \begin{align*}
        d(f,g) = \int_0^1 |f(x)-g(x)|dx = \int_0^1 |f(x)-h(x)+h(x)-g(x)|dx
    \end{align*}
    Note by the Euclidean metric (of real numbers), we have that 
    \[ |f(x)-h(x)+h(x)-g(x)|\leq |f(x)-h(x)| + |h(x)-g(x)| \]
    Thus:
    \begin{align*}
        d(f,g) &\leq \int_0^1 [|f(x)-h(x)| + |h(x)-g(x)|]dx \\
        &= \int_0^1 |f(x)-h(x)|dx + \int_0^1 |h(x)-g(x)|dx  \\
        &= d(f,h) + d(h,g)
    \end{align*}
    as wanted.
\end{s}

\begin{p}
    Let $E$ and $F$ be nonempty subsets of $\mathbb{R}^d$. Prove that if $E$ and $F$ are compact, then there exists 
    $\hat{x}\in E$ and $\hat{y}\in F$ such that
    \[ d(E,F) = |\hat{x}-\hat{y}|. \]
\end{p}

\begin{s}
    Define $f: E\times F \to \mathbb{R}$ by $x \times y \mapsto |x-y|$. Since $E$ and $F$ are compact, so is $E\times F$.
    Furthermore, since $f$ is clearly the Euclidean metric (if we consider $E$ and $F$ to be (topological) subspaces of $\mathbb{R}^d$),
    then $f[E\times F]$ is compact as well. Since $f[E\times F]$ is nonempty ($E$ and $F$ are non-empty) and bounded below (compact),
    there exists some $m\in\mathbb{R}$ such that $m =$ inf$\{x\in f[E\times F]\}$. We show that $m\in f[E\times F]$.\\

    By definition of infimum, we know for all $\epsilon>0$, there exists some $x_{\epsilon}\in f[E\times F]$ such that
    $x_{\epsilon} < m + \epsilon \iff x_{\epsilon} - m < \epsilon$. Since, $m$ is the infimum of $f[E\times F]$, then all
    $x_{\epsilon}\geq m$, therefore $|x_{\epsilon} - m| = x_{\epsilon} - m < \epsilon$. Define the sequence,
    \[ a_n := x_n\in f[E\times F] \qquad\text{such that}\qquad |x_n - m| < \frac{\epsilon}{n} \]
    so we have a sequence $\{a_n\}$ that clearly converges to $m$. Since $f[E\times F]$ is closed (compact), we have that
    this $m$ exists in $f[E\times F]$ i.e. $m\in f[E\times F]$.\\

    Since $m\in f[E\times F]$, then $f^{-1}(m)$ is non-empty, and a tuple $(\hat{x},\hat{y})$ in $f^{-1}(m)$ has the property that
    $|\hat{x}-\hat{y}| =$ inf$\{|x-y|: (x,y)\in E\times F\} = d(E,F)$. 
    Thus, we have the existence of some $\hat{x}\in E$ and $\hat{y}\in F$ such that $d(E,F) = |\hat{x}-\hat{y}|$.
\end{s}

\begin{p}
    Let $f:\mathbb{R}^d\to\mathbb{R}$ be a continuous function, and let $C\subset\mathbb{R}^d$ be compact. Prove that the image
    is also compact.
\end{p}

\begin{s}
    Choose some open cover $\{\theta_{\alpha}\}$ of $f[C]$. Since $f$ is continuous (pre-images of open sets are open) 
    and $f[C]$ is the \textit{image} of $C$ in $f$, then $f^{-1}(\{\theta_{\alpha}\}) = \{f^{-1}(\theta_{\alpha})\}$ is an 
    open cover of $C$ (open because each $\theta_{\alpha}$ is open in $f[C]$ so, $f^{-1}(\theta_{\alpha})$ is open in $C$).
    Since $C$ is compact, then there exists some finite subcover of $\{f^{-1}(\theta_{\alpha})\}$, call it
    \[ \{f^{-1}(\theta_{\alpha_i})\}_{i=1}^n \qquad\text{for some}\;n \]
    Since $f$ is surjective onto its image, then $f(\{f^{-1}(\theta_{\alpha_i})\}_{i=1}^n) = \{\theta_{\alpha_i}\}_{i=1}^n$
    is a finite subcover of $\{\theta_{\alpha}\}$ so that $f[C]$ is compact as wanted.
\end{s}

\begin{p}
    Let $C\subset \mathbb{R}$ be the Cantor set where $C = \cap_{k=0}^{\infty} C_k$. Prove that $C$ is compact and uncountable.
\end{p}

\begin{s}
    Note that each $C_k$ is the \textit{finite} union of closed intervals. Thus, $C_k$ is closed. Since $C$ is the intersection
    of closed sets, it is also closed. Furthermore, $C$ is clearly bounded by $0$ and $1$. Therefore, by Heine-Borel, we have that
    $C$ is compact. It is left to prove uncountability.\\

    First, we give an injection between binary sequences and points of $C$, then we show that this set of binary sequences
    is uncountable. This will imply imply that $C$ is either uncountable or larger than uncountable but in both cases, we see
    the conclusion that $C$ is uncountable. Start with an arbitrary
    binary sequence. Note in $C_1$, the previous interval (in $C_0$) splits in two intervals. Label these intervals 0 and 1. If
    the first term in the sequence is 0, then only consider interval 0, and consider interval 1 if the first term is 1. Note,
    in $C_2$, the interval you considered is again split into two. Repeat the same process but choosing the new interval depending
    on if the \textit{second} term of the sequence is 0 or 1. In this way, we can keep on repeating process to obtain a point
    in $C$ and this point is unique with respect to the binary sequence we chose because it depended on its terms i.e. if we changed
    one of the terms in the binary sequence from a 0 to a 1, then the point we obtain would be different, because at some iteration
    $C_k$, we would be considering a different interval. Thus, this mapping is injective i.e. each binary sequence corresponds
    to exactly one point of $C$. Also note that, in this process, each iteration $C_k$ can be associated with the $k$th term of
    a binary sequence. In this way, we can be sure that by following this process, we can actually get to a point of $C$.\\

    It is left to show that the set of binary sequences are uncountable. Let $A$ be a countable subset of this set of binary sequence.
    Since $A$ is countable, we can index this set by $\mathbb{N}$. Then, we construct a new binary sequence as follows:
    Let $a_1$ be the first term of the first sequence of $A$. Let $a_2$ be the second term of the second sequence of $A$ and so on.
    Now invert every term in the sequence $\{a_n\}$ i.e. if $\{a_n\} = \{0,1,0,1,1,\hdots\}$, then after we invert, it looks like:
    $\{a_n\} = \{1,0,1,0,0,\hdots\}$ so that we are changing every 0 to a 1 and every 1 to 0. This is clearly not in $A$. Therefore,
    we have that $A$ is a proper countable subset of the set of binary sequences implying the set of binary sequences is uncountable.\\

    Therefore, we have that $C$ is countable.
\end{s}

\end{document}
