\documentclass{article}
\usepackage[utf8]{inputenc}
\usepackage[english]{babel}
\usepackage{amsfonts}
\usepackage{amsthm}
\usepackage{amsmath}
\usepackage{amssymb}

\newtheorem{theorem}{Theorem}
\newtheorem{es}{Examples}

\title{MATC34 Problem Set 5}
\date{November 20, 2017}
\author{Anmol Bhullar | 1002678140}

\begin{document}
    \maketitle 

    \textbf{Q1(a)}.\\
    Define $f(z) := z^3$ and $g(z) := z + 2$. Since $f$ and $g$ are both polynomials, we know that $f$ and $g$ are holomorphic on all
    of $\mathbb{C}$. More specifically, if we let $C = \{|z| = 2\}$, then $f$ and $g$ are holomorphic on $C$ and on the interior of
    $C$. Additionally, note that for $z\in\mathbb{C}$ such that $|z|=2$, we have that $|f(z)| = 2^3 = 8 > 4 = 2 + 2 = |z| + 2 \geq 
    |z+2| = |g(z)|$
    which implies that $|f(z)| > |g(z)|$ for all $z\in C$. Thus, by Rouché's theorem, we obtain the result that $f$ and $f+g$ have
    the same number of zeroes. It is trivially seen that $f(z)=z^3$ has 3 zeroes in the interior of $C$ (they all occur at $z=0$),
    therefore, $f+g = z^3+z+2 = p(z)$ has 3 zeroes in the disk $\{|z|<2\}$.\hfill$\blacksquare$\\

    \textbf{Q1(b)}.\\
    Define $f(z) := 2$ and $g(z) := z^3 + z$. Similarly, to before, these functions are holomorphic on all of $\mathbb{C}$ and
    specifically, on $C$ which we will define to be the circle $\{|z| = \frac{1}{2}\}$ and the interior of $C$. Note that
    for all $z\in\mathbb{C}$ such that $|z|=\frac{1}{2}$, we have that $|z^3|=\frac{1}{8}$ and $|z|=\frac{1}{2}$ so that 
    $|f(z)| = 2 > \frac{1}{2} + \frac{1}{2} > \frac{1}{8} + \frac{1}{2} = |z^3| + |z| \geq |z^3+z| = g(z)$. So by Rouché's
    theorem, we obtain that $f$ and $f+g$ have the same number of zeroes in the interior of $C$. The polynomial $f$ clearly has
    no roots, thus $f+g = p(z)$ must not have any roots inside the interior of $C$ also.\hfill$\blacksquare$\\

    \textbf{Q2(a)}.\\
    We know the taylor series of $e^z = 1 + z + \frac{z^2}{2} + \frac{z^3}{6} + O(z^4)$. 
    Subsitute in $\frac{1}{z}$ for $z$ to get the taylor series of  $e^{\frac{1}{z}} = 1 + \frac{1}{z} + \frac{1}{2z^2} 
    + \frac{1}{6z^3} + O(z^{-4})$. Then $e^{\frac{1}{z}}-1 = \frac{1}{z} + \frac{1}{2z^2} + \frac{1}{6z^3} + O(z^{-4})$ and
    finally, we obtain that:
    \[ \frac{e^{\frac{1}{z}}-1}{z} = \frac{1}{z^2} + \frac{1}{2z^3} + \frac{1}{6z^4} + O(z^{-5}) \]
    Since this has infinitely many non zero negative power terms, from the theorem singularity classification theorem (covered in
    lecture), we obtain that this is an essential singularity at $z=0$.\hfill$\blacksquare$.\\

    \textbf{Q2(b)}.\\
    We know the taylor series of $\sin{z} = z - \frac{z^3}{6} + \frac{z^5}{120} + O(z^7)$. Like in the previous solution,
    subsitute in $\frac{1}{z}$ for $z$ to obtain: $\sin{(\frac{1}{z})} = \frac{1}{z} - \frac{1}{6z^3} + \frac{1}{120z^5} + O(z^{-7})$.
    Thus:
    \[ z\sin{(\frac{1}{z})} = 1 - \frac{1}{6z^2} + \frac{1}{120z^4} + O(z^{-6}) \]
    which still clearly has infinitely many non zero negative power terms. From the singularity classification theorem, we obtain
    that this function has an essential singularity at $z=0$.\hfill$\blacksquare$\\

    \textbf{Q2(c)}.\\
    Note 
    \[ f(z) = \frac{\sin{(\frac{1}{z})}}{\cos{(\frac{1}{z})}} = \tan{(\frac{1}{z})} \]
    Since $\tan{z} = z + \frac{z^3}{3} + \frac{2z^5}{15} + O(z^7)$, we obtain that:
    \[ \tan{(\frac{1}{z})} = \frac{1}{z} + \frac{1}{3z^3} + \frac{2}{15z^5} + O(z^{-7}) \]
    which also has an essential singularity for similar reasons as a) and b).\hfill$\blacksquare$\\

    \textbf{Q2(d)}.\\
    Note $\cos{z} = 1 - \frac{z^2}{2} + \frac{z^4}{24} - \frac{z^6}{720} + O(z^7)$, then $1-\cos{z} = \frac{z^2}{2} - \frac{z^4}{24} +
    \frac{z^6}{720} + O(z^8)$. Now, using the power series of $e^z$, subsitute in $z^2$ for $z$ to obtain:
    $e^{z^2} = 1 + z^2 + \frac{z^4}{2} + \frac{z^6}{6} + O(z^8)$ and $e^{z^2} - 1 = z^2 + \frac{z^4}{2} + \frac{z^6}{6} + O(z^8)$.
    Then:
    \begin{align*}
        \frac{1-\cos{z}}{e^{z^2}-1} &= \frac{\frac{z^2}{2}-\frac{z^4}{24}+\frac{z^6}{720}+O(z^7)}{z^2+\frac{z^4}{2}+\frac{z^6}{6} 
            + O(z^8)}\\
        &= \frac{z^2}{z^2}\cdot\frac{\frac{1}{2} + \frac{-z^2}{24}+\frac{z^4}{720}+O(z^5)}{1+\frac{z^2}{2}+\frac{z^4}{6}+O(z^6)}
    \end{align*}
    Now this is clearly holomorphic at $z=0$, since the terms in the numerator and the denominator are of the same order, we obtain
    that $z=0$ is a removable singularity.\hfill$\blacksquare$\\

    \textbf{Q3}.\\
    Let $\gamma_R$ be a toy contour consisting of the upper half of the semi-circle of diameter $R$ centered at $z=0$ 
    (along with the base of the 
    semi-circle) so that $\gamma_R$ is a closed curve. Call the real part of $\gamma_R$ by $H_R$ and call the semi-circle (without the
    base) by $II$. Note $II$ is oriented in a counter-clockwise fashion so that $H_R$ is given the orientation going from left to
    right. The residue theorem tells us that if we have isolated singularties in the interior $\gamma_R$, then:
    \[ \int_{\gamma_R} f(z)dz = 2\pi i\sum_{z_0} Res_{z_0}(f) \qquad\text{where $z_0$ is a pole of $\gamma_R$}\]
    Note, $\int_{\gamma_R} f(z)dz = \int_{H_R}f(z)dz + \int_{II} f(z)dz$. Our goal is to show that as $R\to\infty$, 
    $\int_{II}f(z)dz\to0$, and then compute $\int_{H_R}f(z)dz$ using the residue theorem as stated earlier.\\
    First, we find the poles of $f$. Write $z^4 + 1 = 0$ so that $z^4 = -1$. By using DeMoivre for $n=4,r=1$, to obtain that
    the four angles $\theta = \frac{\pi}{4}, \frac{3\pi}{4}, \frac{5\pi}{4}, \frac{7\pi}{4}$. Index these for $\theta_1,\theta_2,
    \theta_3$ and $\theta_4$. Combining these results, we obtain that the four roots of our equation is: $\omega=e^{\frac{\pi}{4}},
    \omega^3=e^{\frac{3\pi}{4}},\omega^5=e^{\frac{5\pi}{4}},\omega^7=e^{\frac{7\pi}{4}}$. Therefore, we can write
    $z^4 + 1 = (z-\omega)(z-\omega^3)(z-\omega^5)(z-\omega^7)$ to obtain the poles of our function $f(z)$. Note, these are 
    all simple poles. Therefore, we can use the formula on page 75 to compute the residues at these poles. Specifically, for each
    pole $z_0$, we have that res$_{z_0}(f) = \lim_{z\to z_0}(z-z_0)f(z)$. Let $z_0=\omega$. Then:
    \begin{align*}
        \text{res}_{\omega}f &= \lim_{z\to\omega}\frac{(z-\omega)}{z^4+1} \\
        &= \lim_{z\to\omega} \frac{1}{4z^3} \qquad \text{L'hopital's rule} \\
        &= \frac{1}{4\omega^3}
    \end{align*}
    Using a similar process, we also obtain that res$_{\omega^3}f = \frac{1}{4\omega^9}$. Recalling that $\omega,\omega^3,\omega^5,
    \omega^7$ are distinct roots of unity, we know they form a square centered at 0. This implies $\omega^5$ and $\omega^7$ are poles
    but they do not occur in the upper half plane, but at the lower half plane. This implies $\omega^5$ and $\omega^7$ will not be
    poles of $\gamma_R$ for all $R\in\mathbb{R}$ so we can just ignore them. Using these residues, we know that
    \[ 2\pi \sum_{z_0}\text{res}_{z_0} f = 2\pi[\frac{1}{4}e^{\frac{5\pi}{4}} + \frac{1}{4}e^{\frac{7\pi}{4}}] \]
    To simplify this, $e^{\frac{7\pi}{4}} = e^{\frac{-i\pi}{4}} = \frac{-\sqrt{2}}{2} - \frac{-i\sqrt{2}}{\sqrt{2}}$ and
    $e^{\frac{5\pi}{4}} = \frac{\sqrt{2}}{2} - \frac{\sqrt{2}i}{2}$ so that:
    \[ 2\pi \sum_{z_0}\text{res}_{z_0} f = 2\pi[\frac{1}{4}(-\frac{\sqrt{2}}{2} - \frac{\sqrt{2}i}{2})+\frac{1}{4}(
        \frac{\sqrt{2}}{2}-\frac{\sqrt{2}i}{2})] \]
    which is simply just equal to $\frac{\pi\sqrt{2}}{2}$. Therefore $\int_{H_R}f(z)dz = \frac{\pi\sqrt{2}}{2}$.\\
    Parameterize $II$ so that $z(t) = Re^{it}$ for $0\leq t\leq \pi$. Then:
    \[ \int_{II} f(z)dz = \int_0^{\pi} \frac{1}{1+R^4e^{4it}}dt \]
    Since $|R^4e^{4it}| > |R^4|$, it follows:
    \[ |\int_{II}f(z)dz| \leq (\frac{1}{R^4})(\pi R) = \frac{\pi}{R^3}  
        \qquad \text{(max of $f$ on $II$ mulitiplied by length of the curve)} \]
    so that $\int_{II}f(z)dz \to 0$ as $R\to\infty$ as wanted.\\
    Since $H_R$ as $R\to\infty$ is the real line, we obtain that: $\int_{H_R} f(z)dz \to \int_{-\infty}^{\infty}f(x)dx$ as $R\to\infty$.
    Therefore, as $R\to\infty$, $\int_{\gamma_R} f(z)dz = \frac{\pi\sqrt{2}}{2} = \int_{-\infty}^{\infty} \frac{1}{1+x^4}dx$ as 
    wanted.\hfill$\blacksquare$

    \textbf{Q5}.\\
    Suppose first that $f:\mathbb{C}\to\mathbb{C}$ is entire and injective. Then suppose that $f$ is a polynomial. If it is of degree
    at most 1, we are done (degree 0 polynomials are constant functions which are not injective but we supposed $f$ is injective which 
    is a contradiction). Therefore, assume the degree of $f$ is \textit{at least two}, more specifically, it is $n\geq 2$ for some
    $n\in\mathbb{N}$. By the Fundamental Theorem of Algebra, we can say that $f$ has $n$ roots (counted with multiplicities). Suppose
    there exist two distinct roots of $f$, say at $z=z_0$ and $z=z_1$. Then, clearly for $z_0\neq z_1$ we have that $f(z_0)=f(z_1)$
    which shows $f$ is not injective. This is a contradiction to the injective assumption of $f$. Thus, there cannot exist any two
    distinct roots if $f$ is of degree $n\geq 2$. This, in turn implies that there exists one root of multiplicity $n$, say it occurs
    at $z=z_0$. Then $f(z) = a(z-z_0)^n$ for some $a\in\mathbb{C}-\{0\}$ but this maps any distinct $n$th root of unity to the same
    value i.e. $f(z_1) = f(z_2)$ for any $z_1,z_2$ that is a distinct root of unity of $a(z-z_0)^n$. Thus $f$ is not injective which
    is again a contradiction. This implies that $f$ cannot have degree $n\geq 2$ and $n = 0$ so that if $f$ is to be a polynomial
    and be injective, it must have a degree of 1 i.e. if $f$ is a polynomial (then it is entire) and injective, then $f(z) = az+b$.\\
    Now suppose $f$ is not a polynomial. Shift $z$ so that $f(0) = 0$. Since $f$ is entire, then it can be written as 
    $f(z) = \sum_{n\in\mathbb{N}} a_nz^n$ (we know only finitely many $a_n$'s can be non-zero since by assumption $f$ is not 
    a polynomial).  Subsitute $\frac{1}{z}$ for $z$ to get:
    \[ f(\frac{1}{z}) = \sum_{n\in\mathbb{Z}} a_{n}z^{n} \]
    so that there is now an essential singularity. By the open mapping theorem, if $U$ is an open set containing 0, then $V := f(U)$
    is an open set also containing zero (injective). However, by the Casorati-Weierstrass theorem, we have that the image
    $f(U-\{0\})$ is dense in $\mathbb{C}$.

\end{document}
