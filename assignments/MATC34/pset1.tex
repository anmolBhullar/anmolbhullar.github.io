\documentclass[12pt]{article}
\usepackage{amsmath} 
\usepackage{amsthm} % Theorem Formatting
\usepackage{amssymb}    % Math symbols such as \mathbb
\usepackage{graphicx} % Allows for eps images
\usepackage[dvips,letterpaper,margin=1in,bottom=0.7in]{geometry}
\usepackage{tensor}
 % Sets margins and page size
\usepackage{amsmath}

\renewcommand{\labelenumi}{(\alph{enumi})} % Use letters for enumerate
% \DeclareMathOperator{\Sample}{Sample}
\let\vaccent=\v % rename builtin command \v{} to \vaccent{}
\usepackage{enumerate}
\renewcommand{\v}[1]{\ensuremath{\mathbf{#1}}} % for vectors
\newcommand{\gv}[1]{\ensuremath{\mbox{\boldmath$ #1 $}}}
% for vectors of Greek letters
\newcommand{\uv}[1]{\ensuremath{\mathbf{\hat{#1}}}} % for unit vector
\newcommand{\abs}[1]{\left| #1 \right|} % for absolute value
\newcommand{\avg}[1]{\left< #1 \right>} % for average
\let\underdot=\d % rename builtin command \d{} to \underdot{}
\renewcommand{\d}[2]{\frac{d #1}{d #2}} % for derivatives
\newcommand{\dd}[2]{\frac{d^2 #1}{d #2^2}} % for double derivatives
\newcommand{\pd}[2]{\frac{\partial #1}{\partial #2}}
% for partial derivatives
\newcommand{\pdd}[2]{\frac{\partial^2 #1}{\partial #2^2}}
% for double partial derivatives
\newcommand{\pdc}[3]{\left( \frac{\partial #1}{\partial #2}
 \right)_{#3}} % for thermodynamic partial derivatives
\newcommand{\ket}[1]{\left| #1 \right>} % for Dirac bras
\newcommand{\bra}[1]{\left< #1 \right|} % for Dirac kets
\newcommand{\braket}[2]{\left< #1 \vphantom{#2} \right|
 \left. #2 \vphantom{#1} \right>} % for Dirac brackets
\newcommand{\matrixel}[3]{\left< #1 \vphantom{#2#3} \right|
 #2 \left| #3 \vphantom{#1#2} \right>} % for Dirac matrix elements
\newcommand{\grad}[1]{\gv{\nabla} #1} % for gradient
\let\divsymb=\div % rename builtin command \div to \divsymb
\renewcommand{\div}[1]{\gv{\nabla} \cdot \v{#1}} % for divergence
\newcommand{\curl}[1]{\gv{\nabla} \times \v{#1}} % for curl
\let\baraccent=\= % rename builtin command \= to \baraccent
\renewcommand{\=}[1]{\stackrel{#1}{=}} % for putting numbers above =
\providecommand{\wave}[1]{\v{\tilde{#1}}}
\providecommand{\fr}{\frac}
\providecommand{\RR}{\mathbb{R}}
\providecommand{\NN}{\mathbb{N}}
\providecommand{\seq}{\subseteq}
\providecommand{\e}{\epsilon}

\newtheorem{prop}{Proposition}
\newtheorem{thm}{Theorem}[section]
\newtheorem{axiom}{Axiom}[section]
\newtheorem{p}{Problem}[section]
\usepackage{cancel}
\newtheorem*{lem}{Lemma}
\theoremstyle{definition}
\newtheorem*{dfn}{Definition}
 \newenvironment{s}{%\small%
        \begin{trivlist} \item \textbf{Solution}. }{%
            \hspace*{\fill} $\blacksquare$\end{trivlist}}%
% ***********************************************************
% ********************** END HEADER *************************
% ***********************************************************

\begin{document}

{\noindent\Huge\bf  \\[0.5\baselineskip] {\fontfamily{cmr}\selectfont  %
Problem Set 1}         }\\[2\baselineskip] % Title
{ {\bf \fontfamily{cmr}\selectfont MATC34: Complex Variables}\\ {\textit{\fontfamily{cmr}%
\selectfont September 14, 2017}}}
{\large \textsc{Anmol Bhullar}} % Author name
\\[1.4\baselineskip]

\begin{p}
    For each of the following equations, describe geometrically the set of points $z$ in the
    complex plane defined by it
    \begin{enumerate}
        \item $\abs{z-2} = \abs{z-3}$
        \item $\frac{1}{z} = \overline{z}$
        \item Re$(z) = 3$
        \item $\abs{z} = $Re$(z) +1$
    \end{enumerate}
\end{p}
\begin{s}
    \begin{enumerate}
        \item These are the set of points which are equidistant from both 2 and 3. This forms a line
            equivalent to the perpendicular bisector of the straight line between $x=2$ and $x=3$.
        \item Let $z = x+iy$ for $x,y\in\RR$. Then 
            \[\frac{1}{z} = \overline{z} \to \frac{1}{x+iy} = x-iy \to 1 = x^2-y^2\]
            which is just the unit circle around the origin.
        \item This is the vertical line at $x=3$
        \item This is equivalent to the horizontal line at Im$(z) = 1$
    \end{enumerate}
\end{s}

\begin{p}
    \begin{enumerate}
        \item Find how many roots to the equation \[ z^5 = 2 \] there are in the complex
            plane, and describe their location in the plane.
        \item Given any $\alpha\in\mathbb{C}$ with $\alpha\neq0$ and $n\geq1$ an integer,
            describe the set of solutions to the equation \[ z^n = \alpha \]
    \end{enumerate}
\end{p}
\begin{s}
    \begin{enumerate}
        \item Note that there is a real root at $z = \sqrt[5]{2}$. Converting to polar
            coordinates, we see that this root occurs at $\theta_0 = 0$. Thus, the other
            four roots occur at: $\theta = \frac{2\pi k}{5}$ for $k = 1,\hdots,4$. So,
            we obtain, that the roots are: $z = \sqrt[5]{2}e^{\frac{i2\pi k}{5}}$.
        \item Repeating a similar process, we obtain that the real root is at 
            $z = \sqrt[n]{\alpha}$. Thus, the other roots are at: 
            $z = \sqrt[n]{\alpha}e^{\frac{i2\pi k}{n}}$ for $k = 0,\hdots,4$.
    \end{enumerate}
\end{s}

\begin{p}
    Let $z,w$ be two complex numbers such that $\overline{z}w\neq 1$. Prove that
    \[ \abs{\frac{w-z}{1-\overline{w}{z}}} = 1\]
\end{p}
\begin{s}
    Suppose $\abs{w} = 1$. Then:
    \[ \abs{\frac{w-z}{1-\overline{w}{z}}} = \abs{\frac{1}{w}\cdot\frac{w-z}{1-\overline{w}{z}}}\]
    since if $\abs{w} = 1$, then $\frac{1}{w}$ also has norm of 1. Simplifying, we obtain in the denominator:
    \[\abs{w - \overline{w}zw} = \abs{w - z} \]
    so the whole fraction reduces to 1. Now, suppose $\abs{z} = 1$, then $\abs{\overline{z}} = 1$, so that
    $\frac{1}{\overline{z}}$ also has a norm of 1. Repeating a similar process as above, we obtain that:
    \[ \frac{w-z}{1-\overline{w}z} = \abs{\frac{w-z}{\overline{z} - \overline{w}z\overline{z}}} = \abs{\frac{w-z}{\overline{z-w}}} = \abs{\frac{w-z}{w-z}} (\text{conjugate symmetry})\]
    so the fraction, again reduces down to 1 as wanted.
\end{s}

\begin{p}
    Using only the axioms which define a field, prove that if $F$ is a field, then
    $x\cdot 0 = 0$ for all $x\in F$.
\end{p}
\begin{s}
    Note that, $0$ is the additive identity, so that $0 + 0 = 0$. Then:
    \[ x \cdot 0 = x \cdot (0 + 0) = x\cdot 0 + x\cdot 0 \]
    Also, $(x\cdot 0) + 0 = x\cdot 0$ so by the line above, we have that:
    \[x\cdot 0 + 0 = x\cdot 0 + x\cdot 0\]
    Subtracting $x\cdot 0$ from both sides and switching orders (using commutivity),
    we obtain that:
    \[x\cdot 0 - x\cdot 0 + 0 = x\cdot 0 - x\cdot 0 + x\cdot 0\]
    Simplifying, we obtain that:
    \[0 = x\cdot 0\]
\end{s}

\begin{p}
    Show that there is no total ordering on $\mathbb{C}$ by:
    \begin{enumerate}
        \item By using $(-1)\cdot(-1)=1$, prove that it is impossible that $-1 > 0$.
            Conclude that $1 > 0$.
        \item Is it possible that $i > 0$? How about $i < 0$?
        \item Conclude that there is no total ordering on $\mathbb{C}$.
    \end{enumerate}
\end{p}
\begin{s}
    Suppose $-1 > 0$. Then note that, by using property (iii) (of total order) and $-1 > 0$, if we multiply
    both sides by $-1$, then the same inequality holds so:
    \[-1 > 0 \implies (-1) \cdot (-1) > 0 \implies 1 > 0\]
    By property (ii) of total order and that $1 > 0$, we obtain that:
    \[ -1 > 0 \implies -1 + 1 > 0 + 1 \implies 0 > 1\] which is a contradiction so $-1 > 0$ cannot hold. By
    property (i) of total order, then $-1 < 0$. Now suppose that $i > 0$, then via property (iii) of total
    order, we obtain the inequality:
    \[i > 0 \implies i \cdot i > 0 \cdot i \implies -1 > 0\]
    which we have already shown is not true so $i > 0$ cannot be true. Thus, suppose $i < 0$. Then, $-i > 0$.
    So, by using property (iii) of total order, we obtain that:
    \[ 0 \cdot (-i) < (-i) \cdot (-i) = -1 \implies 0 < -1 \]
    which is a contradiction. Thus $i < 0$ does not hold which implies $i = 0$ (by property (i)) but this is
    obviously not true (e.g. $3 \cdot i = 3i \neq 0$). Since property (i) of total order is broken, then
    $\mathbb{C}$ does not have a total ordering on it.
\end{s}
\end{document}
