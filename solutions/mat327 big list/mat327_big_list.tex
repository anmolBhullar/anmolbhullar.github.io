%-----------------------------------------------------------------------
% Beginning of amstext-l-template.tex
%-----------------------------------------------------------------------
%
%    This is a template file for the AMS book series Pure and Applied
%    Undergraduate Texts, for use with AMS-LaTeX 2.0.
%    Separate chapters should be included at the appropriate position.
%
%    Templates for various common text, math and figure elements are
%    given following the \end{document} line.
%
%%%%%%%%%%%%%%%%%%%%%%%%%%%%%%%%%%%%%%%%%%%%%%%%%%%%%%%%%%%%%%%%%%%%%%%%

%    Use this form if exercise blocks are only at the end of chapters.
\documentclass{amstext-l}
%    If you intersperse blocks of exercises throughout chapters, use
%    the documentclass with the [multixcb] option.
%\documentclass[multixcb]<amstext-l}

%    For use when working on individual chapters
%\includeonly{}

%    If you need symbols beyond the basic set, uncomment this command.
\usepackage{amssymb}

%    If your book includes graphics, uncomment this command.
%\usepackage{graphicx}

%    If you are using the author-year citation style:
%\usepackage{natbib}

%    Include other referenced packages here.
\usepackage{exercise}
\usepackage{amsmath}
\usepackage{csquotes}

\newtheorem{theorem}{Theorem}[chapter]
\newtheorem{lemma}[theorem]{Lemma}

\theoremstyle{definition}
\newtheorem{definition}[theorem]{Definition}
\newtheorem{example}[theorem]{Example}
\newtheorem{xca}[theorem]{Exercise}

\theoremstyle{remark}
\newtheorem{remark}[theorem]{Remark}

\numberwithin{section}{chapter}
\numberwithin{equation}{chapter}

%   Commands
\newcommand{\RR}{\mathbb{R}}
\newcommand{\B}{\mathcal{B}}
\newcommand{\e}{\epsilon}
\newcommand{\T}{\mathcal{T}}

%    For a single index; for multiple indexes, see the manual
%    "Instructions for preparation of papers and monographs:
%    AMS-LaTeX" (instr-l.pdf in the author package).
\makeindex

\begin{document}

\frontmatter

\title{MAT327: Introduction to Topology. Solutions to the Big List Problems}

%    Remove any unused author tags.

%    author one information
\author{Anmol Bhullar}
\address{ Undergraduate Student University of Toronto, Scarborough}
\email{anmol.bhullar@mail.utoronto.ca}
\thanks{Thank you to my instructor Ivan Khatchatourian for providing these wonderful problems}

% \subjclass[2000]{Primary }
%    The 2010 edition of the Mathematics Subject Classification is
%    now available.  If you are citing a classification from the
%    new scheme, use the following input coding instead.
%\subjclass[2010]{Primary }

\keywords{}

\maketitle

%    Dedication.  If the dedication is longer than a line or two,
%    remove the centering instructions and the line break.
%\cleardoublepage
%\thispagestyle{empty}
%    If this book uses the documentclass stml-l or mmono-s, change
%    13.5pc to 10.5pc.
%\vspace*{13.5pc}
%\begin{center}
%  Dedication text (use \\[2pt] for line break if necessary)
%\end{center}
%\cleardoublepage

%    Change page number to 6 if a dedication is present.
\setcounter{page}{5}

\tableofcontents

%    Include unnumbered chapters (preface, acknowledgments, etc.) here.
\chapter{Preface}

I attempt to answer and \LaTeX all of the solutions to the big list of problems posted by my
instructor Ivan Khatchatourian for MAT327: Introduction to Topology. The problems are 
separated into difficulties which are labelled via asterisks. One asterisk being the lowest
difficulty and 3 being the highest. Especially hard problems are marked via a cross. This
is the format my instructor uses and I'm merely copying it for consistency's sake.

\mainmatter
%    Include main chapters here.
\chapter{Topologies}

\begin{ExerciseList}
	\Exercise[difficulty=1] Fix $a<b\in\RR$. Show explicitly that the %
    open interval $(a,b)$ is open in $\RR_{\text{usual}}$. Show %
    explicitly that the interval $[a,b)$ is not open in %
    $\RR_{\text{usual}}$

	\Answer First, we show that $(a, b)$ is open in %
    $\mathbb{R}_{\text{usual}}$. To do this: we have to show that %
    for any point $x\in (a,b)$, there exists a neigbourhood $N(x,\e) %
    \subseteq (a,b)$.\\
    Thus, choose a point $x\in(a,b)$. By density of $\RR$, there %
    exists $x_1\in(a,x)$ and $x_2\in(x,b)$. Let $\e =$ min$(x_1,x_2).%
    $ Then $N(x,\e) \subseteq (a,b)$.\\
    Next, we show that $[a,b)$ is not open. To do this, we show that %
    $\exists$ no $\e>0$ such that $N(a,\e) \subseteq [a,b)$. Note %
    that any $x\in (a-\e,a]$ would not be in $(a,b)$ for any $\e>0$, %
    so $N(a,\e)\not\subseteq [a,b)$.

    \Exercise[difficulty=1] Let $X$ be a set and $\mathcal{B} = %
    \{ \{x\}: x\in X\}$. Show that the only topology on $X$ that %
    contains $\B$ as a subset is the discrete topology.

    \Answer To show that only $X_{\text{discrete}}$ has the property %
    $\B\subseteq X_{\text{discrete}}$, consider the set:
    \[\T := \{ \{x\}: x\in X\}\]
    For $\T$ to be a topology, it must be closed under finite %
    intersections so we must put $\emptyset$ in $\T$. Furthermore, %
    $\T$ must be closed under the union of an arbitrary collection %
    elements of $X$. If this is to be true, then $\T$ must contain %
    every subset of $X$ since any arbritrary subset can be written %
    as the union of all of its elements. Thus an arbitrary subset %
    $A\subseteq X$ must be in $\T$. This implies %
    $\T = X_{\text{discrete}}$.

    \stepcounter{Exercise}

    \Exercise[difficulty=1] Let $(X, \T_{\text{co-countable}})$ be %
    an infinite set with the co-countable topology. Show that %
    $\T_{\text{co-countable}}$ is closed under countable %
    intersections but not necessairly arbitrary ones.

    \Answer Given a countable indexing set $I$, we have to show %
    \[\bigcap_{\alpha\in I} U_{\alpha} \in \T_{\text{co-countable}}\]
    Equivalently, we can show $X \setminus (\cap U_{\alpha})$ %
    is countable. By DeMorgan's Law: $X\setminus(\cap_{\alpha})=
    \cup (X\setminus U_{\alpha})$. Since the right hand side %
    is the countable union of countable sets, it is countable.
    Thus, $\cap U_{\alpha} \in \T_{\text{co-countable}}$. To show %
    that an arbitrary intersection of elements is not open, it %
    suffices to state that the intersection of arbitrary many %
    elements does not necessairly form a countable set.

    \Exercise[difficulty=1] Let $(X, \T)$ be a topological space, %
    and let $A\subseteq X$ be a set with the property that for all %
    $x\in A,\;\exists$ an open set $U_x\in\T$ such that $x\in U_x %
    \subseteq A$. Show that $A$ is open.

    \Answer Let $I$ be a set which indexes the elements of $A$. Then:
    \[\bigcup_{\alpha\in I} \{x_{\alpha}\} = A\]
    Similarly, since $x_{\alpha} \in U_{x_{\alpha}}$ and %
    $U_{x_{\alpha}} \subseteq A$, then:
    \[\bigcup_{\alpha\in I} U_{x_{\alpha}} = A\]
    Since this is the union of arbitrary elements of $\T$, then %
    $A\in\T$.

\end{ExerciseList}


\appendix
%    Include appendix "chapters" here.
\include{}

\backmatter
%    Bibliographies can be prepared with BibTeX using amsplain,
%    amsalpha, or (for author-year style) natbib.
\bibliographystyle{amsalpha}
\bibliography{}

%    See note above about multiple indexes.
\printindex

\end{document}

%%%%%%%%%%%%%%%%%%%%%%%%%%%%%%%%%%%%%%%%%%%%%%%%%%%%%%%%%%%%%%%%%%%%%%%%

%    Templates for common elements in a book; for information specific
%    to this series, see the sample amstext-doc.pdf and .tex files.
%    For additional general information, see the AMS-LaTeX instructions
%    manual, instr-l.pdf, included in the author package, and the amsthm
%    user's guide, linked from http://www.ams.org/tex/amslatex.html .

%    Section headings
\section{}
\subsection{}

%    Exercises grouped in a section
\begin{xcb}
\begin{enumerate}
\item ...
\item ...
\end{enumerate}
\end{xcb}

%    Exercise standing alone in text
\begin{exa}[Optional exercise heading]
% text of exercise
\end{exa}

%    Framed environment for highlighting important (brief) information
\begin{framedthm}{<type of theorem-class element>}
% text
\end{framedthm}

%    Environment for inclusions to be skipped on first reading; note
%    that, even if there is no heading text, the second pair of braces
%    must be present.
\begin{inclusion}{<optional heading text>}
% text, will be set in smaller type
\end{inclusion}

%    Figure insertion; default placement is top; if the figure occupies
%    more than 75% of a page, the [p] option should be specified.
\begin{figure}
\includegraphics{filename}
\caption{text of caption}
\label{}
\end{figure}

%    Ordinary theorem and proof
\begin{theorem}[Optional addition to theorem head]
% text of theorem
\end{theorem}

\begin{proof}[Optional replacement proof heading]
% text of proof
\end{proof}

%    Marker for the end of some important element.  \lozenge is used
%    here only as an illustration; other symbols may also be used.
% text
\xqed{\lozenge}

%    Mathematical displays; for additional information, see the amsmath
%    user's guide, linked from http://www.ams.org/tex/amslatex.html .

%    Numbered equation
\begin{equation}
\end{equation}

%    Unnumbered equation
\begin{equation*}
\end{equation*}

%    Aligned equations
\begin{align}
  &  \\
  &
\end{align}

%    In-chapter bibliography
\renewcommand{\bibname}{References for this chapter}
\begin{inchapterbibliography}{AAA}
\bibitem[]{} text
\end{inchapterbibliography}

%-----------------------------------------------------------------------
% End of amstext-l-template.tex
%-----------------------------------------------------------------------
