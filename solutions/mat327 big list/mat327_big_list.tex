%-----------------------------------------------------------------------
% Beginning of amstext-l-template.tex
%-----------------------------------------------------------------------
%
%    This is a template file for the AMS book series Pure and Applied
%    Undergraduate Texts, for use with AMS-LaTeX 2.0.
%    Separate chapters should be included at the appropriate position.
%
%    Templates for various common text, math and figure elements are
%    given following the \end{document} line.
%
%%%%%%%%%%%%%%%%%%%%%%%%%%%%%%%%%%%%%%%%%%%%%%%%%%%%%%%%%%%%%%%%%%%%%%%%

%    Use this form if exercise blocks are only at the end of chapters.
\documentclass{amstext-l}
%    If you intersperse blocks of exercises throughout chapters, use
%    the documentclass with the [multixcb] option.
%\documentclass[multixcb]<amstext-l}

%    For use when working on individual chapters
%\includeonly{}

%    If you need symbols beyond the basic set, uncomment this command.
%\usepackage{amssymb}

%    If your book includes graphics, uncomment this command.
%\usepackage{graphicx}

%    If you are using the author-year citation style:
%\usepackage{natbib}

%    Include other referenced packages here.
\usepackage{}

\newtheorem{theorem}{Theorem}[chapter]
\newtheorem{lemma}[theorem]{Lemma}

\theoremstyle{definition}
\newtheorem{definition}[theorem]{Definition}
\newtheorem{example}[theorem]{Example}
\newtheorem{xca}[theorem]{Exercise}

\theoremstyle{remark}
\newtheorem{remark}[theorem]{Remark}

\numberwithin{section}{chapter}
\numberwithin{equation}{chapter}

%    For a single index; for multiple indexes, see the manual
%    "Instructions for preparation of papers and monographs:
%    AMS-LaTeX" (instr-l.pdf in the author package).
\makeindex

\begin{document}

\frontmatter

\title{}

%    Remove any unused author tags.

%    author one information
\author{}
\address{}
\curraddr{}
\email{}
\thanks{}

%    author two information
\author{}
\address{}
\curraddr{}
\email{}
\thanks{}

\subjclass[2000]{Primary }
%    The 2010 edition of the Mathematics Subject Classification is
%    now available.  If you are citing a classification from the
%    new scheme, use the following input coding instead.
%\subjclass[2010]{Primary }

\keywords{}

\maketitle

%    Dedication.  If the dedication is longer than a line or two,
%    remove the centering instructions and the line break.
%\cleardoublepage
%\thispagestyle{empty}
%    If this book uses the documentclass stml-l or mmono-s, change
%    13.5pc to 10.5pc.
%\vspace*{13.5pc}
%\begin{center}
%  Dedication text (use \\[2pt] for line break if necessary)
%\end{center}
%\cleardoublepage

%    Change page number to 6 if a dedication is present.
\setcounter{page}{5}

\tableofcontents

%    Include unnumbered chapters (preface, acknowledgments, etc.) here.
\include{}

\mainmatter
%    Include main chapters here.
\include{}

\appendix
%    Include appendix "chapters" here.
\include{}

\backmatter
%    Bibliographies can be prepared with BibTeX using amsplain,
%    amsalpha, or (for author-year style) natbib.
\bibliographystyle{amsalpha}
\bibliography{}

%    See note above about multiple indexes.
\printindex

\end{document}

%%%%%%%%%%%%%%%%%%%%%%%%%%%%%%%%%%%%%%%%%%%%%%%%%%%%%%%%%%%%%%%%%%%%%%%%

%    Templates for common elements in a book; for information specific
%    to this series, see the sample amstext-doc.pdf and .tex files.
%    For additional general information, see the AMS-LaTeX instructions
%    manual, instr-l.pdf, included in the author package, and the amsthm
%    user's guide, linked from http://www.ams.org/tex/amslatex.html .

%    Section headings
\section{}
\subsection{}

%    Exercises grouped in a section
\begin{xcb}
\begin{enumerate}
\item ...
\item ...
\end{enumerate}
\end{xcb}

%    Exercise standing alone in text
\begin{exa}[Optional exercise heading]
% text of exercise
\end{exa}

%    Framed environment for highlighting important (brief) information
\begin{framedthm}{<type of theorem-class element>}
% text
\end{framedthm}

%    Environment for inclusions to be skipped on first reading; note
%    that, even if there is no heading text, the second pair of braces
%    must be present.
\begin{inclusion}{<optional heading text>}
% text, will be set in smaller type
\end{inclusion}

%    Figure insertion; default placement is top; if the figure occupies
%    more than 75% of a page, the [p] option should be specified.
\begin{figure}
\includegraphics{filename}
\caption{text of caption}
\label{}
\end{figure}

%    Ordinary theorem and proof
\begin{theorem}[Optional addition to theorem head]
% text of theorem
\end{theorem}

\begin{proof}[Optional replacement proof heading]
% text of proof
\end{proof}

%    Marker for the end of some important element.  \lozenge is used
%    here only as an illustration; other symbols may also be used.
% text
\xqed{\lozenge}

%    Mathematical displays; for additional information, see the amsmath
%    user's guide, linked from http://www.ams.org/tex/amslatex.html .

%    Numbered equation
\begin{equation}
\end{equation}

%    Unnumbered equation
\begin{equation*}
\end{equation*}

%    Aligned equations
\begin{align}
  &  \\
  &
\end{align}

%    In-chapter bibliography
\renewcommand{\bibname}{References for this chapter}
\begin{inchapterbibliography}{AAA}
\bibitem[]{} text
\end{inchapterbibliography}

%-----------------------------------------------------------------------
% End of amstext-l-template.tex
%-----------------------------------------------------------------------
