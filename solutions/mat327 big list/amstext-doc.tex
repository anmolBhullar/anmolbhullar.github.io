%% filename: amstext-doc.tex
%% version: 0.92
%% date: 2010/03/10
%%
%% American Mathematical Society
%% Technical Support
%% Publications Technical Group
%% 201 Charles Street
%% Providence, RI 02904
%% USA
%% tel: (401) 455-4080
%%      (800) 321-4267 (USA and Canada only)
%% fax: (401) 331-3842
%% email: tech-support@ams.org
%%
%% Copyright 2010 American Mathematical Society.
%%
%% Unlimited copying and redistribution of this file are permitted as
%% long as this file is not modified.  Modifications, and distribution
%% of modified versions, are permitted, but only if the resulting file
%% is renamed.
%%
%% ====================================================================
\documentclass[multixcb]{amstext-l}
\usepackage{url}
\usepackage{mathrsfs}
\DeclareMathAlphabet{\mathzc}{OT1}{pzc}{m}{it}

\theoremstyle{plain}
\newtheorem{theorem}{Theorem}[chapter]
\newtheorem{lemma}[theorem]{Lemma}

\theoremstyle{definition}
\newtheorem{xca}{Exercise}[section]

\newcommand{\cs}[1]{\texttt{\char`\\#1}}
\def\<#1>{$\langle$\textit{#1}$\rangle$}
\newenvironment{exm}{%
  \par
  \begingroup
    \parindent0pt
    \leftskip2\normalparindent
    \obeylines
}{%
    \par
  \endgroup
%  \noindent\ignorespaces
}

\begin{document}

\frontmatter

\makeatletter
\def\currversion{\csname ver@amstext-l.cls\endcsname}
\makeatother

\title{Preparing a Manuscript for the\\ AMSTEXT Book Series}
\author{American Mathematical Society\\[12pt]
        \smaller \currversion}
%        \smaller Version 0.92 beta, March 2010}

\maketitle

\tableofcontents

\mainmatter

\chapter{Introduction}

The \texttt{amstext-l} (``ell'' for \LaTeX, not ``one'') document class
is intended for undergraduate textbooks and other instructional works.
Since this is a new area for the AMS, the document class still lacks
some features that will ultimately prove useful, or even essential.  We
intend to build in more features as they are proposed by authors of
volumes in the series, aiming always for generality.

This brief document is intended to introduce features that are
specific to \texttt{amstext-l}.  It has been produced using
\verb+\documentclass{amstext-l}+ and illustrates this style in
practice.  Features already present in \texttt{amsbook} (on which
\texttt{amstext-l} is based) will not be reiterated in detail;
please refer to the manuals for AMS document classes \cite{INSTR}
and the \texttt{amsthm} package \cite{THM} for that information.

\section{Type sizes and styles}

\subsection{Type sizes}

The default type size for \texttt{amstext-l} is 10pt---the size
used for this introductory manual.  This results from the command
\begin{exm}
\verb+\documentclass{amstext-l}+
\end{exm}
\noindent
An 11pt option is available with
\begin{exm}
\verb+\documentclass[11pt]{amstext-l}+
\end{exm}

\subsection{Sectioning}

Experience with manuscripts prepared for this series
demonstrates that section headings in the ``usual'' style do not
stand out sufficiently well to be spotted easily among closely
packed examples and exercises.  For this reason, a sans serif font
has been substituted for the more traditional serif form, and lower
levels are indented to contrast with the predominantly flush left
theorem-class headings.  Chapter headings are also set in sans serif
to complement the section heading style.

\subsubsection{Second-level sections}

Headings at this level are set in slanted sans serif type.
Subsubsections are not by default listed in the table of contents.

\section{Existing facilities for exercises}

Some facilities that already exist in the basic \texttt{amsbook}
document class are especially relevant for textbooks, but are not
as well documented in \cite{INSTR} as they might be.

\subsection{Exercises grouped in a section}

The \texttt{xcb} environment is intended for exercises that are
grouped in a separate section either at the end of a chapter or
between other sections within a chapter.  These sections are
unnumbered.  The environment requires one argument: the title of
the section:
\begin{exm}
\verb+\begin{xcb}{Sample exercise section}+
\dots
\verb+\end{xcb}+
\end{exm}
\noindent
By default, the end of an \texttt{xcb} section is unmarked, but
an end-of-section marker can be set by specifying the document-class
option \texttt{[multixcb]}.  This marker is most appropriate when
\texttt{xcb} sections are interspersed among other sections.  When
no option is specified, the section heading, like all chapters and
other section headings (numbered and unnumbered), is listed in the
table of contents; when the option is specified, ``Exercises'' does
not appear in the table of contents and the running heads are
unaffected.

The sample exercise section below is set with the \texttt{[multixcb]}
option.

Within an \texttt{xcb} section, the \texttt{enumerate} environment
is used to number the exercises.  The space allowed for exercise
numbers assumes that no more than 99 exercises will be present in
the section.  However, the number of digits can be changed with
the \cs{xcbdigits} command:
\begin{exm}
\verb+\xcbdigits{999}+
\end{exm}
\noindent
This command can be placed either in the preamble or directly before
the line \verb+\begin{xcb}+; it will have no effect if it is placed
within the scope of the \texttt{xcb} environment.

\begin{xcb}{Sample exercise section}

\begin{enumerate}
\item First exercise.
\item Second exercise.
\end{enumerate}

\end{xcb}

\subsection{Individual exercises within main text}

If exercises are desired outside of an \texttt{xcb} block,
it is recommended to set up an individual exercise environment,
\texttt{xca}, as a theorem-class element:
\begin{exm}
\verb+\theoremstyle{definition}+
\verb+\newtheorem{xca}{Exercise}[+\<\textit{counter}>\verb+]+
\end{exm}
\noindent
The \texttt{xca} environment is defined in the \texttt{amstext-l.template}
file, but if you do not use that, you will need to define it yourself.

The following example of an \texttt{xca} exercise was input as
\begin{exm}
\verb+\begin{xca}[Illustrative exercise]+
\texttt{Study the input for this document and suggest improvements.}
\verb+\end{xca}+
\end{exm}

\begin{xca}[Illustrative exercise]
Study the input for this document and suggest improvements.
\end{xca}


%%%%%%%%%%%%%%%%%%%%%%%%%%%%%%%%%%%%%%%%%%%%%%%%%%%%%%%%%%%%%%%%%%%%%%%%

\chapter{New Features}

Several new environments and commands have been defined for
\texttt{amstext-l}:
\begin{itemize}
\item a framed environment for highlighting important
  information, for example theorems;
\item an ``inclusion'' environment for de-emphasizing material
  that is more advanced, or can be skipped over on first reading;
\item a bibliography style suitable for inclusion in a chapter,
  rather than at the end of the book.  (This feature is a candidate
  for inclusion in the next update of \texttt{amsbook}.)
\item a facility to place symbols comparable to the ``QED box'' at
  the end of elements other than the proof environment.  (Only the
  proof environment, where it is automatic, can at present accept
  such an addition in the basic AMS document classes.)
\end{itemize}

\section{Framed environment for highlighting important information}

The framed environment, \verb+framedthm+, is illustrated by two examples.

\begin{framedthm}{theorem}
If $(Q,\sigma)$ is a quiver with automorphism such that the
associated valued quiver $\Gamma=\Gamma(Q,\sigma)$ contains no
loops, then there is a surjective map $\psi\colon
\Phi(Q)\to\Phi(Q,\sigma)$ defined by putting $\psi(\beta):=
\theta(\tilde\sigma(\beta))$, $\beta\in\Phi(Q)$. Moreover, if
$\alpha=\psi(\beta)$ is a real root in $\Phi(Q,\sigma)$, then
$\beta$ must be a real root in $\Phi(Q)$, uniquely determined up
to $\sigma$-conjugacy.
\end{framedthm}

Note that a framed element will not be broken at the end of a
page; such an element falling too close to the bottom of a page
will be moved to the next page.  Since material treated in this
fashion is usually brief, this restriction should not pose a
problem.  Moreover, presentation as a unit on one
page aids comprehension and the ability to locate a particular
item at a later time.

\begin{framedthm}{lemma}
%\label{TAforMQ}
In the tensor algebra $\mathsf T(\mathscr Q)$, the
component $\mathsf T^n(M)$ of grade $n$ decomposes as
$$\mathsf T^n(M)\cong\bigoplus M_{\rho_n}
 \otimes_{\mathzc D_{h\rho_{n-1}}}\cdots\otimes_{\mathzc  D_{h\rho_1}}M_{\rho_1},
$$
where the sum is formed over the set of paths $\rho_n\cdots\rho_1$
of length $n$.
\end{framedthm}

\section{Environment for inclusions to be skipped on first reading}

A left-indented ``inclusion'' environment is provided for text
intended to be skipped on first reading.  It is set one size smaller
than the main text, but may be varied by redefining \cs{inclusionfont}.
One argument is required: a heading, which may be empty.

The input for an inclusion is entered like this:
\begin{exm}
\verb+\begin{inclusion}{+\<optional heading text>\verb+}+
...
\verb+\end{inclusion}+
\end{exm}

One important feature of an inclusion is that it must be able to
span multiple pages.  The second example below should be long
enough to demonstrate this requirement.

\begin{inclusion}{}
Example without a heading.
This prose is set off in a smaller size.  It comprises several lines.
This prose is set off in a smaller size.  It comprises several lines.
This prose is set off in a smaller size.  It comprises several lines.

Here is the beginning of a new paragraph.
This prose is set off in a smaller size.  It comprises several lines.
This prose is set off in a smaller size.  It comprises several lines.
\end{inclusion}

That wasn't very interesting; the next example should be more entertaining.

\begin{inclusion}{Jabberwocky\footnote{Apologies to
  Lewis Carroll \cite{Do}.}}
\obeylines\parindent=0pt
'Twas brillig, and the slithy toves
  Did gyre and gimble in the wabe:
All mimsy were the borogoves,
  And the mome raths outgrabe.
\null
\leavevmode\llap{``}Beware the Jabberwock, my son!
  The jaws that bite, the claws that catch!
Beware the Jubjub bird, and shun
  The frumious Bandersnatch!''
\null
He took his vorpal sword in hand:
  Long time the manxome foe he sought---
So rested he by the Tumtum tree,
  And stood awhile in thought.
\null
And, as in uffish thought he stood,\par\nobreak
  The Jabberwock, with eyes of flame,
Came whiffling through the tulgey wood,
  And burbled as it came!
\null
One, two! One, two! And through and through
  The vorpal blade went snicker-snack!
He left it dead, and with its head
  He went galumphing back.
\null
\leavevmode\llap{``}And, has thou slain the Jabberwock?
  Come to my arms, my beamish boy!
O frabjous day! Callooh! Callay!''
  He chortled in his joy.
\null
'Twas brillig, and the slithy toves
  Did gyre and gimble in the wabe:
All mimsy were the borogoves,
  And the mome raths outgrabe.
\end{inclusion}

\section{``In-chapter'' bibliography}

It is sometimes appropriate to include a brief reference list within
a chapter, in addition to, or in place of the main bibliography (which
is always a separate chapter).  Such a bibliography is presented as
an unnumbered section.  The \texttt{inchapterbibliography} environment
produces the desired list.  However, it is usually desirable to change
the title from ``Bibliography'' to something more appropriate.  This
is done by redefining the command \cs{bibname}; remember that
\cs{bibname} must be changed again before the main bibliography
is processed.

This input sets up the format for the local references list.
\begin{exm}
\verb+\renewcommand{\bibname}{References for this chapter}+
\verb+\begin{inchapterbibliography}{Do}+
\verb+...+
\verb+\end{inchapterbibliography}+
\end{exm}
\noindent
The coding of references is the same as for the main bibliography.

\renewcommand{\bibname}{References for this chapter}
\begin{inchapterbibliography}{Do}

\bibitem[Do]{Do} Charles Lutwidge Dodgson [Lewis Carroll],
  \textit{Through the Looking-Glass and What Alice Found There}, 1872.

\end{inchapterbibliography}

\section{End-of-element markers}

Especially for pedagogical clarity, textbook authors sometimes use
markers (similar to the {\qedsymbol} at the end of a proof) to
indicate the end of other elements, such as definitions, examples,
remarks, and the like.  The command \cs{xqed} is provided for this
purpose; it takes one argument: the command for the desired symbol.

The input\par\nobreak
\begin{exm}
\verb+\xqed{\lozenge}+
\end{exm}
\noindent
places a lozenge flush right at the end of this paragraph.
\xqed{\lozenge}

The command \cs{qedhere}, which makes it possible to force the
QED symbol up onto the last line of a display or list that ends
a proof, is not easily adapted to non-proof environments, although
consideration is being given to include this in future releases.

If such a situation occurs in practice, you can request assistance
by e-mail from the AMS Publications Technical Group at
\texttt{tech-support@ams.org}; please identify the AMSTEXT series
in your message, and include a small, self-contained, {\LaTeX}able
file that contains the problem material and enough surrounding
text to allow for valid experimentation.


%%%%%%%%%%%%%%%%%%%%%%%%%%%%%%%%%%%%%%%%%%%%%%%%%%%%%%%%%%%%%%%%%%%%%%%%

\renewcommand{\bibname}{Bibliography}
\begin{thebibliography}{INSTR}

\bibitem[INSTR]{INSTR} \textit{Instructions for Preparation of
  Papers and Monographs:\ AMS-\LaTeX, version~\textup{2.20}},
  Amer. Math. Soc., Providence, RI, 2004.
  \url{ftp://ftp.ams.org/pub/author-info/documentation/amslatex/instr-l.pdf}

\bibitem[THM]{THM} \textit{Using the \texttt{amsthm} package,
  version~\textup{2.20}}, Amer. Math. Soc., Providence, RI, 2004.
  \url{ftp://ftp.ams.org/pub/tex/doc/amscls/amsthdoc.pdf}

\end{thebibliography}

\end{document}
