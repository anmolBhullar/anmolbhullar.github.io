\documentclass[12pt]{article}
\usepackage[a4paper]{geometry}
\usepackage{mhchem}
\usepackage{amsfonts}
\usepackage{amsthm}
\usepackage{amsmath}
\usepackage{amssymb}
\usepackage{nicefrac}

\usepackage[lastexercise]{exercise}

\begin{document}
\section{Smooth Manifolds}

\begin{ExerciseList}
  \Exercise
 	Show that equivalent definitions of locally Euclidean spaces are obtained if, instead of requiring $U$ to be homeomorphic to an open subset of $\mathbb{R}^n$, we require it to be 
  	homeomorphic to an open ball in $\mathbb{R}^n$, or to $\mathbb{R}^n$ itself. 
  \Answer
  	Let $\phi: U\to\tilde{U}$ where $p\in U\subset M$ and $\tilde{U}\subset\mathbb{R}^n$. Since $\tilde{U}$ is open,
        we can find an open ball around $\phi(p)$ (denoted $B_r(\phi(p))$ for some $r>0$) such that $B_r(\phi(p))\subset\tilde{U}$.
        We know $p\in \phi^{-1}(B_r(\phi(p)))$ and is open since $\phi$ and $\phi^{-1}$ is continuous. Therefore, the map
        $\phi': \phi^{-1}(B_r(\phi(p))) \to B_r(\phi(p))$ is a homeomorphism. Using this along with the fact that open balls
        are of course open sets in $\mathbb{R}^n$, we obtain that requiring $U$ to be homeomorphic to open subsets of $\mathbb{R}^n$
        or open balls of $\mathbb{R}^n$ makes no difference.\\
        
        Next, consider a ball $B_r(x)$ centered at $x\neq 0$ of radius $r>0$. The function $f: B_r(x)\to B_r(0)$ translates this
        ball to the origin via $z\mapsto z-x$. This map is clearly bijective and linear (so it is differentiable and so, continuous)
        with a continuous inverse, thus it is a homeomorphism. Therefore, it suffices to consider to balls centered at the origin.
        Now, consider the map $g: B_r(0) \to B_1(0)$ defined by $z\mapsto \nicefrac{z}{r}$ which is clearly still a homeomorphism.
        Now, consider a map $\pi: B_1(0)\to\mathbb{R}^n$ defined by $x\mapsto (\tan(\nicefrac{\pi|x|}{2})x)$.$\hfill\blacksquare$
  
  \Exercise
  	Show that any topological subspace of a Hausdorff space is Hausdorff, and any finite product of Hausdorff spaces is Hausdorff.
  \Answer
  	Let $A$ be a subspace of a Hausdorff space $T$. Let $x\neq y\in A$, since $x,y\in T$, then there exist open sets $U,V$
        in $T$ such that $x\in U$, $y\in V$ and $V\cap U=\emptyset$. Since $x\in U$ and $x\in A$, then $x\in U\cap A$ and similarly,
        for $y\in V\cap A$ and since $U\cap A\subset U\cap A$ and similarly for $V$, we have that $U\cap A \cap V \cap A = \emptyset$.
        Therefore, since $U\cap A$ and $V\cap A$ are open sets in $A$, we have the existence of two disjoint sets which contain
        $x$ and $y$ so that $A$ is also Hausdorff. A similar proof follows for finite products except we take the product of
        the disjoint sets i.e. if $U_1\times\hdots\times U_n$ is our space, then take disjoint sets from each $U_i$ and product
        them together.$\hfill\blacksquare$
  
  \Exercise
  	Show that any topological subspace of a second countable space is second countable, and any finite product of second countable spaces is second countable.
  \Answer
  	 Let $A$ be a subspace of the second countable space $T$. Since a basis of $A$ can be given by $B_i
        = \{A \cap T_i: T_i\in\{T_i\}\}$, we can simply let $\{T_i\}$ be our countable basis of $T$. Thus, $A$ is second countable.
        Note if $U_1\times\hdots\times U_n$ is a finite product of second countable space, its basis is given by
        $B = \{U_{1_i}\times\hdots\times U_{n_i}: i\in\mathbb{N},$ $U_{k_i}$ is the $i$th basis element of $U_k\}$. 
        From this, choosing the appropriate basis yields the fact that $U_1\times\hdots\times U_n$ is second countable.$\hfill\blacksquare$
  
  \Exercise
  	Show that $\mathbb{P}^n$ is Hausdorff and second countable, and is therefore a topological $n$-manifold.
  \Answer
  	Let $x\neq y\in \mathbb{P}^n$. We show there exists $U,V\in\mathbb{P}^n$ such that $U\cap V=\emptyset$ with $x\in U$ and $y\in V$. Since $x\neq y$, we have that $\varphi_i[x]\neq \varphi_i[y]$
	for any $1\leq i\leq (n+1)$. Note $\varphi_i[x]$ and $\varphi_i[y]$ are points in $\mathbb{R}^n$ and so we can abuse the fact that $\mathbb{R}^n$ is Hausdorff. There exist open $U'\subset\mathbb{R}^n$
	 and $V'\subset\mathbb{R}^n$ disjoint sets that contain $\varphi_i[x]$ and $\varphi_i[y]$ respectively. Using the fact that $\varphi_i$ is a homeomorphism, we get that $U := \phi_i^{-1}(U')$ and 
	 $V := \varphi_i^{-1}(V')$ are disjoint and they contain $x$ and $y$ respectively. Furthermore, $U$ and $V$ are open since their images are open. Thus, $\mathbb{P}^n$ is Hausdorff.\\
	 
	 It is left to show that $\mathbb{P}^n$ is second countable. We know that all of the $U_i$'s for $i = 1,\hdots,n+1$ cover $\mathbb{P}^n$. Let $\{X_i\}_{i=1}^{n+1} = \{\phi_i(U_i)\}_{i=1}^{n+1}$.
	 Each $X_i\subset\mathbb{R}^d$ which is second countable, so each $X_i$ as a subspace is second countable. Let $\{Y_{i_j}\}_{j=1}^{\infty}$ be a countable basis for $X_i$.
	 We claim,
	 \[ M := \bigcup_{i=1}^{n+1} \{\phi_i^{-1}(Y_{i_j}): j=1,2,3\hdots\}\]
	 is a countable basis of $\mathbb{P}^n$. Clearly $M$ is countable because each $\cup_{j=1}^{\infty} \phi_i^{-1}(Y_{i_j})$ is countable and the finite union of countable sets is countable. Let
	 $O$ be an open set of $\mathbb{P}^n$. Then note $O = \cup_{i=1}^{n+1} (U_i\cap O)$. Note, each $\phi_i(U_i\cap O)$ can be written as a countable union of elements of $\{Y_{i_j}\}$ and so
	 $U_i\cap O$ can be written as a countable union of elements of $\{\phi_i^{-1}(Y_{i_j}): j=1,2,3,\hdots\}$. Thus, we can write each $U_i\cap O$ as a countable union of elements of $M$ and so
	 we can write $\cup_{i=1}^{n+1} (U_i\cap O)$ as a countable union of elements of $M$. Since $M$ only contains open sets of $\mathbb{P}^n$ (i.e. cannot generate a topology bigger than $M$), 
	 we have that $M$ is a countable basis of $\mathbb{P}^n$ as wanted.$\hfill\blacksquare$
	 
  \Exercise
  	Let $M$ be a topological manifold. Two smooth atlases for $M$ determine the same maximal smooth atlas if and only if their union is a smooth atlas.
   \Answer
   	Let $\mathcal{X},\mathcal{Y}$ be two smooth maximal smooth atlases containing $\mathcal{A}$ and $\mathcal{B}$ respectively.
	Let $\mathcal{A}$ and $\mathcal{B}$ be two smooth atlases for $M$ such that $\mathcal{A} \cup \mathcal{B}$ is a smooth atlas for $M$. Since $\mathcal{A}\subset\mathcal{X}$, it follows
	$(\mathcal{A}\cup\mathcal{B})\cap\mathcal{X}\neq\emptyset$. Thus, if $\mathcal{X}$ is to be the unique maximal smooth atlas determined by $\mathcal{A}$, 
	we must have that $\mathcal{A}\cup\mathcal{B}\subset\mathcal{X}$. For if this were not the case, we would have the existence of a chart smoothly compatible with $\mathcal{A}$ which is not
	in $\mathcal{X}$. Similarly, we obtain $\mathcal{A}\cup\mathcal{B} \subset \mathcal{Y}$. From the uniqueness of both $\mathcal{X}$ and $\mathcal{Y}$, we obtain $\mathcal{X}=\mathcal{Y}$ i.e.
	$\mathcal{A}$ and $\mathcal{B}$ determine the same smooth atlas. Now, suppose instead $\mathcal{A}$ and $\mathcal{B}$ determine the same maximal smooth atlas $\mathcal{X}$. Let
	$(U,\phi)\in\mathcal{A}$ and $(V,\psi)\in\mathcal{B}$. Since $(V,\psi)\in\mathcal{X}$, it follows $(U,\phi)$ and $(V,\psi)$ are smoothly compatible. Similarly, the converse holds i.e. every chart
	in $\mathcal{A}$ is smoothly compatible with charts in $\mathcal{B}$. Thus, $\mathcal{A}\cup\mathcal{B}$ is a smooth atlas.$\hfill\blacksquare$
	
   \Exercise
   	If $k$ is an integer between 0 and min$(m,n)$, show that the set of $m\times n$ matrices whose rank is at least $k$ is an open submanifold of M$(m\times n,\mathbb{R})$.

   \Exercise
   	By identifying $\mathbb{R}^2$ with $\mathbb{C}$ in the usual way, we can think of the unit circle $\mathbb{S}^1$ as a subset of the complex plane. An \textit{angle function} on
	a subset $U\subset\mathbb{S}^1$ is a continuous function $\theta:U\to\mathbb{R}$ such that $e^{i\theta(p)} = p$ for all $p\in U$. Show that there exists an angle function on an
	open subset $U\subset\mathbb{S}^1$ if and only if $U\neq\mathbb{S}^1$. For any such angle function, show that $(U,\theta)$ is a smooth coordinate chart for $\mathbb{S}^1$ with its
	standard smooth structure.
	
   \Exercise
   	Let $0<k<n$ be integers, and let $P,Q\subset\mathbb{R}^n$ be the subspaces spanned by $(e_1,\hdots,e_k)$ and $(e_{k+1},\hdots,e_n)$, respectively, where $e_i$ is the $i$th standard
	basis vector. For any $k$-dimensional subspace $S\subset\mathbb{R}^n$ that has the trivial intersection with $Q$, show that the coordinate representation $\phi(S)$ constructed in the
	example on Grassmannian manifolds is the unique $(n-k)\times k$ matrix $B$ such that $S$ is spanned by the columns of the matrix $\bigl( \begin{smallmatrix} I_k \\ B\end{smallmatrix}\bigr)$, where
	$I_k$ denotes the $k\times k$ identity matrix.
   	
\end{ExerciseList}

\end{document}