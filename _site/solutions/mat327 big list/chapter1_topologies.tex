\chapter{Topologies}

\begin{ExerciseList}
	\Exercise[difficulty=1] Fix $a<b\in\RR$. Show explicitly that the %
    open interval $(a,b)$ is open in $\RR_{\text{usual}}$. Show %
    explicitly that the interval $[a,b)$ is not open in %
    $\RR_{\text{usual}}$

	\Answer First, we show that $(a, b)$ is open in %
    $\mathbb{R}_{\text{usual}}$. To do this: we have to show that %
    for any point $x\in (a,b)$, there exists a neigbourhood $N(x,\e) %
    \subseteq (a,b)$.\\
    Thus, choose a point $x\in(a,b)$. By density of $\RR$, there %
    exists $x_1\in(a,x)$ and $x_2\in(x,b)$. Let $\e =$ min$(x_1,x_2).%
    $ Then $N(x,\e) \subseteq (a,b)$.\\
    Next, we show that $[a,b)$ is not open. To do this, we show that %
    $\exists$ no $\e>0$ such that $N(a,\e) \subseteq [a,b)$. Note %
    that any $x\in (a-\e,a]$ would not be in $(a,b)$ for any $\e>0$, %
    so $N(a,\e)\not\subseteq [a,b)$.

    \Exercise[difficulty=1] Let $X$ be a set and $\mathcal{B} = %
    \{ \{x\}: x\in X\}$. Show that the only topology on $X$ that %
    contains $\B$ as a subset is the discrete topology.

    \Answer To show that only $X_{\text{discrete}}$ has the property %
    $\B\subseteq X_{\text{discrete}}$, consider the set:
    \[\T := \{ \{x\}: x\in X\}\]
    For $\T$ to be a topology, it must be closed under finite %
    intersections so we must put $\emptyset$ in $\T$. Furthermore, %
    $\T$ must be closed under the union of an arbitrary collection %
    elements of $X$. If this is to be true, then $\T$ must contain %
    every subset of $X$ since any arbritrary subset can be written %
    as the union of all of its elements. Thus an arbitrary subset %
    $A\subseteq X$ must be in $\T$. This implies %
    $\T = X_{\text{discrete}}$.

    \stepcounter{Exercise}

    \Exercise[difficulty=1] Let $(X, \T_{\text{co-countable}})$ be %
    an infinite set with the co-countable topology. Show that %
    $\T_{\text{co-countable}}$ is closed under countable %
    intersections but not necessairly arbitrary ones.

    \Answer Given a countable indexing set $I$, we have to show %
    \[\bigcap_{\alpha\in I} U_{\alpha} \in \T_{\text{co-countable}}\]
    Equivalently, we can show $X \setminus (\cap U_{\alpha})$ %
    is countable. By DeMorgan's Law: $X\setminus(\cap_{\alpha})=
    \cup (X\setminus U_{\alpha})$. Since the right hand side %
    is the countable union of countable sets, it is countable.
    Thus, $\cap U_{\alpha} \in \T_{\text{co-countable}}$. To show %
    that an arbitrary intersection of elements is not open, it %
    suffices to state that the intersection of arbitrary many %
    elements does not necessairly form a countable set.

    \Exercise[difficulty=1] Let $(X, \T)$ be a topological space, %
    and let $A\subseteq X$ be a set with the property that for all %
    $x\in A,\;\exists$ an open set $U_x\in\T$ such that $x\in U_x %
    \subseteq A$. Show that $A$ is open.

    \Answer Let $I$ be a set which indexes the elements of $A$. Then:
    \[\bigcup_{\alpha\in I} \{x_{\alpha}\} = A\]
    Similarly, since $x_{\alpha} \in U_{x_{\alpha}}$ and %
    $U_{x_{\alpha}} \subseteq A$, then:
    \[\bigcup_{\alpha\in I} U_{x_{\alpha}} = A\]
    Since this is the union of arbitrary elements of $\T$, then %
    $A\in\T$.

    \Exercise[difficulty=1] Let $(X,\T)$ be a topological space, and %
    let $f: X\to Y$ be injective. Is $\T_f := \{f(U): U\in\T\}$ a %
    topology on $Y$? Is it necessairly a topology on the range of $f$?

    \Answer Lost solution : (

    \Exercise[difficulty=1] Let $X$ be a set and let $\T_1$ and %
    $\T_2$ be two topologies on $X$. Is $\T_1\cup\T_2$ a topology %
    on $X$? What about $\T_1\cap\T_2$? Is yes, prove it. If not, %
    provide a counterexample.

    \Answer Let $\T_1 := \T_{\text{usual}}$, $\T_2 := \T_7$ on $\RR$. 
    Consider:
    \[(6,7) \cap [6.5, 7] \qquad\text{both in}\T_1\cup\T_2)\]
    but this intersection yields the interval: $[6.5,7)$
    which is not open in either $\T_1$ or $\T_2$. So $\T_1\cup\T_2$ %
    is not necessairly a topology. Now, we show that $\T_1\cap\T_2$ %
    is a topology:\\
    \begin{align*}
        \emptyset\in\T_1,\emptyset\in\T_2\implies\emptyset%
        \in\T_1\cap\T_2 \\
        X\in\T_1, X\in\T_2 \implies X\in\T_1\cap\T_2
    \end{align*}
    Furthermore, given, $U,V\in\T_1,T_2$. Then $U\cap V$ is in both %
    $\T_1$ and $\T_2$ so it follows $U, V\in\T_1\cap\T_2$ and that %
    since $U,V\in\T_1$, then $U\cap V\in\T_1$. Similarly for $\T_2$ %
    so then $U\cap V\in\T_1\cap\T_2$. Thus, the finite intersection %
    of open sets in $\T_1\cap\T_2$ is in $\T_1\cap\T_2$. Now it is %
    left to show that $\cup_{\alpha\in I} V_{\alpha} \in %
    \T_1\cap\T_2$ given that for all $\alpha\in I$, $V_{\alpha} \in %
    \T_1\cap\T_2$. If every $V_{\alpha}\in\T_1\cap\T_2$, then it %
    follows $\cup V_{\alpha}\in T_1$ and similarly for $\T_2$.
    Thus $\T_1\cap\T_2$ is a topology.

    \Exercise[difficulty=1] Let $X$ be an infinite set. Show that %
    there are infinitely many topologies on $X$.

    \Answer Define $\T_{\text{co-k}} := \{U\subseteq X: %
    |U^c|\leq k\}$. Since $\T_{\text{co-k}}$ is a topology for all %
    $k\in\NN$, it follows that if $|X|=\infty$, then there are %
    infinite topologies on $X$.

    \Exercise[difficulty=1] Let $\{\T_{\alpha} : \alpha\in I\}$ be %
    a collection of topologies on a set $X$, where $I$ is some %
    indexing set. Prove that there is a unique finest topology that %
    is refined by all the $\T_{\alpha}'s$. That is, prove that %
    there is a topology $\T$ on $X$ such that:
    \begin{enumerate}
        \item $\T_{\alpha}$ refines $\T$ for every $\alpha\in I$
        \item If $\T^{'}$ is another topology that fulfills (a), %
            then $\T$ is finer than $\T^{'}$
    \end{enumerate}

    \Answer Claim: $\T = \cap \T_{\alpha}$. By Ex 7, $\T$ is a %
    topology, and $\T_{\alpha}$ refines $\T$ by construction of %
    $\T$. Suppose $T^{'}$ is another topology of $X$ such that %
    all $\T_{\alpha}$ refine $\T^{'}$ and that $\T{'}$ refines %
    $\T$. Then, there exists $x\in\T^{'}$ such that $x\not\in\T$.
    But then $x\not\in \cap \T_{\alpha}$ so there exists %
    $T_{\alpha_0}$ such that $x\not\in\T_{\alpha_0}$. Thus $\T$ %
    is not refined by $\T^{'}$ and then $\T^{'}=\T$ which implies %
    $\T$ is unique.

    \Exercise[difficulty=1] This extends exercise 6. Show with %
    examples that the assumption that $f$ is injective is necessary.
    That is, give an example of a topological space $(X,\T)$ and %
    a non-injective function $f:X\to Y$ such that $\T_f$ is a %
    topology and another example where $\T_f$ is not.

    \Answer An example of a non-injective function which is not a %
    topology is given by mapping all of the irrationals of $\RR$ %
    to themselves but mapping all rationals to 0. Since there is %
    no open set in $\RR$ which maps to 0, we arrive that $\T_f$ %
    is not a topology. An example of a non injective function which %
    is a topology is given by:

    \Exercise[difficulty=3] Working in $\RR_{\text{usual}}$: 
    \begin{enumerate}
        \item Show that every non empty open set contains a %
            rational number
        \item Show that there is no uncountable collection of %
            pairwise disjoint open subsets of $\RR$.
    \end{enumerate}

    \Answer Let $U$ be such a set. For any $x\in U$, $\exists \e>0$ %
    such that $N(x,\e)\subseteq U$, but this is impossible since %
    by density of $\QQ$, there exists a $y\in(x,x+\e)$. Then, no %
    such $U$ exists. Now, we prove (b). We know the set of all %
    intervals $(a,b)$ where $a,b\in\QQ$ is countable. Suppose %
    $\{X_{\alpha}:\alpha\in I\}$ is such a set. If we show %
    $\theta:=\{X_{\alpha}:\alpha\in I\}$, $|\theta|\geq|%
    \{(a,b):a,b\in\QQ\}|$ then we are done. Since we construct %
    a set $\theta_2$ where for every $X_{\alpha}\in\theta$, %
    $\theta_2$ contains two rational open subsets of $X_{\alpha}$.
    Then $|\theta|\leq|\theta_2|$ implying that $\theta_2$ is at %
    most countable.
\end{ExerciseList}
