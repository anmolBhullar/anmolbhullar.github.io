\documentclass[a4paper, 11pt]{book}

\newcommand*{\plogo}{\fbox{$\mathcal{DB}$}}

\usepackage[utf8]{inputenc}
\usepackage{stix}
\usepackage[english]{babel}
\usepackage{amsfonts}
\usepackage{amsthm}
\usepackage{amsmath}
\usepackage{amssymb}
\usepackage{nicefrac}
\usepackage{commath}
\usepackage{dirtytalk}

\usepackage{graphicx}

\newtheorem{theorem}{Theorem}
\newtheorem{es}{Examples}
\newtheorem*{example}{Example}

\newcommand\restr[2]{{% we make the whole thing an ordinary symbol
    \left.\kern-\nulldelimiterspace % automatically resize the bar with \right
    #1 % the function
    \vphantom{\big|} % pretend it's a little taller at normal size
    \right|_{#2} % this is the delimiter
}}

\theoremstyle{definition}
\newtheorem{definition}{Definition}[section]
 
\theoremstyle{remark}
\newtheorem*{remark}{Remark}

\newtheorem{corollary}{Corollary}[theorem]
\newtheorem{lemma}[theorem]{Lemma}

\newtheorem{prop}{Proposition}

\newcommand{\inter}[1]{int(#1)}

\begin{document}

    \begin{titlepage} 
        \raggedleft

        \rule{1pt}{\textheight} % Vertical line
        \hspace{0.05\textwidth} % Whitespace between the vertical line and title page text
        \parbox[b]{0.75\textwidth}{ % Paragraph box for holding the title page text, adjust the width to move the title page 
            % left or right on the page
                            
            {\Huge\bfseries MAT1847 Lecture \\[0.5\baselineskip] Notes }\\[2\baselineskip] % Title
            {\large\textit{Lecture notes on a course Holomorphic Dynamics}}\\[4\baselineskip] % Subtitle or further description
            {\Large\textsc{A. Wortschöpfer}} % Author name, lower case for consistent small caps
                                                    
            \vspace{0.5\textheight} % Whitespace between the title block and the publisher
                                                            
            {\noindent The Publisher~~\plogo}\\[\baselineskip] % Publisher and logo
        }
    \end{titlepage}

    \chapter{Riemann Surfaces}

    The next three sections are meant to be an overview of background material required for this course.

    \section{Simply Connected Surfaces}

    \begin{definition}[Riemann Surface]
        A \textbf{Riemann surface} is a connected, paracompact, Hausdorff topological space $X$ equipped with an open covering $\{U_i\}$ 
        and a collection of homeomorphisms $f_i: U_i\to\mathbb{C}$ such that there exist analytic (i.e. holomorphic) $g_{ij}$ 
        such that
        \[ f_i \circ f_j^{-1} = g_{ij} \]
        where we have the assumption $U_i\cap U_j\neq 0$. We refer to the collection $\{(U_i,f_i)\}$ as the \textbf{chart} of 
        the Riemann surface. An equivalent way defining a Riemann surface is as follows:
        A Riemann surface is a connected complex analytic manifold of complex dimension 1.
    \end{definition}

    \begin{example}
        The complex plane $\mathbb{C}$ is the most basic Riemann surface. $\mathbb{C}$ is not only connected but also
        \textit{simply} connected and is also clearly Hausdorff and paracompact. The chart is given by $\{(\mathbb{C},i)\}$
        where $i: \mathbb{C}\to\mathbb{C}$ is the identity function. In fact, any non-empty open set $U$ of $\mathbb{C}$ (for example
        $\mathbb{H} = \{z\in\mathbb{C}: $Im$(z)>0\}$) is also a Riemann surface.
    \end{example}

    \begin{example}[Riemann sphere]
        The Riemann sphere is another example of a \textit{simply} connected Riemann surface. This sphere is not a subset
        of $\mathbb{C}$ nor is it biholomorphic to $\mathbb{C}$. This is the extended complex plane (i.e. $\mathbb{C}\cup\{\pm\infty\}$).
        Geometrically, this can be idealized using \textit{stereographic projection}:\\
        Identify the $xy-$plane (in $\mathbb{R}^3$) with $\mathbb{C}$. Let $\mathbb{S}$ be the sphere sitting in $\mathbb{R}^3$ centered
        at $(0,0,\nicefrac{1}{2})$ and of radius $\nicefrac{1}{2}$. Also, denote $\mathbb{N}$ to be the North pole of $\mathbb{S}$.
        Then, given any point $W = (X,Y,Z)$ on $\mathbb{S}-\mathbb{N}$, the line joining $N$ and $W$ intersects the $xy$-plane at
        a single point $w$. We say that $w$ is the \textbf{stereographic projection} of $W$. The inverse map can also be derived
        similarly, therefore we have a bijective mapping between $\mathbb{S}-\mathbb{N}$ and $\mathbb{C}$. As the point $w$ goes to
        infinity in $\mathbb{C}$ (i.e. $|w|\to\infty$) the corresponding point $W$ on $\mathbb{S}$ comes arbitrarily close to 
        $\mathbb{N}$. Thus, we can define $\mathbb{N}$ to be the point at infinity. Therefore, we can identify the extended complex 
        plane with the sphere $\mathbb{S}$.
    \end{example}

    \begin{remark}
        We can show that the projection mapping is not only a bijective mapping but a \textit{diffeomorphic} mapping. This makes
        showing that transition maps are holomorphic not too hard.
    \end{remark}

    \begin{definition}[Conformal]
        A function $f:\mathbb{C}\to\mathbb{C}$ is \textbf{conformal} if it preserves angles.
    \end{definition}

    \begin{remark}
        A function $f:\mathbb{U}\to\mathbb{C}$ is \textbf{holomorphic} if
        \begin{enumerate}
            \item the Cauchy-Riemann equations hold i.e.
                \[ \overline{\partial}{f} = 0\qquad\text{and}\qquad \nicefrac{\partial{f}}{\partial{\overline{z}}} = 0 \]
            \item \[ \lim_{h\to 0} \frac{f(z+h)-f(z)}{h} = f'(z)\quad\text{exists} \]
            \item For all $z_0\in U$, there exists a convergent power series such that $f(z) = \sum_{n=0}^{\infty} a_nz^n$ for
                $z$ in a neighbourhood of $z_0$.
        \end{enumerate}
    \end{remark}

    \begin{remark}
        A holomorphic function $f:V\to\mathbb{C}$ is \textbf{conformal} if the derivative $f'(z)$ never vanishes i.e. $f'(z_0)\neq 0$
        for all $z\in V$. (note $f': V\to\mathbb{C}$ also and $f'(z)\neq 0$ implies that $f$ is locally injective). We can see this
        through:
        \begin{align*}
            f'(z_0) &= \lim_{h\to0} \frac{f(z+h)-f(z)}{h} \\
            \implies f(z+h) &\approx f(z) + f'(z)h
        \end{align*}
        Recall that multiplying by a number (non-zero and complex) is a composition of rotation and dilation so $f$ is conformal.
    \end{remark}

    There are three main \textit{simply} connected Riemann surfaces that are distinct (general case is uncountable).

    \begin{theorem}[Uniformization Theorem]
        Any simply connected Riemann surface is conformally isomorphic either
        \begin{enumerate}
            \item to the open disk $\mathbb{D}\subset\mathbb{C}$ consisting of all $|z|^2 < 1$ or,
            \item to the plane $\mathbb{C}$
            \item to the Riemann sphere $\hat{\mathbb{C}}$
        \end{enumerate}
    \end{theorem}

    To see that these are indeed inequivalent surfaces, we have to talk about the following:

    \begin{theorem}[Liouville]
        Every entire, bounded function is constant.
    \end{theorem}

    \begin{corollary}
        $\mathbb{C}$ is not biholomorphic to $\mathbb{D}$.
    \end{corollary}
    \begin{proof}
        Let $f$ be a holomorphic function $f:\mathbb{C}\to\mathbb{D}$. Then, $f$ is bounded since $D$ is bounded. By Liouville's
        theorem, we have that $f$ is constant. Since constant functions are clearly not bijective, we have that $f$ is not
        biholomorphic. Therefore, there exist no biholomorphic maps between $\mathbb{C}$ and $\mathbb{D}$. 
    \end{proof}

    It is clear that $\hat{\mathbb{C}}$ is not biholomorphic to any of the other Riemann surfaces on the list above since it is
    compact and the two other surfaces are not. Here is another way to arrive at the same conclusion:

    \begin{remark}
        There are no non-constant biholomorphic maps $f:\hat{\mathbb{C}}\to\mathbb{C}$.
    \end{remark}
    \begin{proof}
        It is clear that $f(\hat{\mathbb{C}})$ is compact since $\hat{\mathbb{C}}$ is and $f$ is continuous. By the open mapping
        theorem, if $f$ is non-constant, $f(\hat{\mathbb{C}})$ is open. However, there are no open compact subsets of $\mathbb{C}$.
        Therefore, $f$ is constant as wanted.
    \end{proof}

    \begin{remark}
        $\mathbb{D}\hookrightarrow \mathbb{C} \hookrightarrow \hat{\mathbb{C}}$.
    \end{remark}

    \begin{corollary}
        Every Riemann surface is the quotient of either $\mathbb{D},\mathbb{C}$, or $\hat{\mathbb{C}}$ by a discrete group $\Gamma$
        of automorphisms of $\mathbb{D},\mathbb{C},$ or $\hat{\mathbb{C}}$.
    \end{corollary}

    \begin{definition}[Conformally Isomorphic]
        Two Riemann surfaces $S$ and $S'$ are \textbf{conformally isomorphic} (or \textbf{biholomorphic}) if and only if there
        exists a homeomorphism from $S$ to $S'$ which is holomorphic in terms of the respecting local uniformizing parameters i.e.
        the transition maps. Equivalently, we can say that a \textbf{biholomorphism} is a bijective and holomorphic map (this forces
        the inverse to be holomorphic as well).\\
        A \textbf{conformal automorphism} is a conformal isomorphism $f:U\to U$ ($U$ a complex domain or Riemann surface). 
    \end{definition}



\end{document}
