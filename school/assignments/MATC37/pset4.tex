\documentclass[12pt]{article}

\usepackage{answers}
\usepackage{setspace}
\usepackage{graphicx}
\usepackage{enumitem}
\usepackage{multicol}
\usepackage{mathrsfs}
\usepackage[margin=1in]{geometry} 
\usepackage{amsmath,amsthm,amssymb}
 
\newcommand{\N}{\mathbb{N}}
\newcommand{\Z}{\mathbb{Z}}
\newcommand{\C}{\mathbb{C}}
\newcommand{\R}{\mathbb{R}}

\DeclareMathOperator{\sech}{sech}
\DeclareMathOperator{\csch}{csch}
 
\newenvironment{theorem}[2][Theorem]{\begin{trivlist}
\item[\hskip \labelsep {\bfseries #1}\hskip \labelsep {\bfseries #2.}]}{\end{trivlist}}
\newenvironment{definition}[2][Definition]{\begin{trivlist}
\item[\hskip \labelsep {\bfseries #1}\hskip \labelsep {\bfseries #2.}]}{\end{trivlist}}
\newenvironment{proposition}[2][Proposition]{\begin{trivlist}
\item[\hskip \labelsep {\bfseries #1}\hskip \labelsep {\bfseries #2.}]}{\end{trivlist}}
\newenvironment{lemma}[2][Lemma]{\begin{trivlist}
\item[\hskip \labelsep {\bfseries #1}\hskip \labelsep {\bfseries #2.}]}{\end{trivlist}}
\newenvironment{exercise}[2][Exercise]{\begin{trivlist}
\item[\hskip \labelsep {\bfseries #1}\hskip \labelsep {\bfseries #2.}]}{\end{trivlist}}
\newenvironment{solution}[2][Solution]{\begin{trivlist}
\item[\hskip \labelsep {\bfseries #1}]}{\end{trivlist}}
\newenvironment{problem}[2][Problem]{\begin{trivlist}
\item[\hskip \labelsep {\bfseries #1}\hskip \labelsep {\bfseries #2.}]}{\end{trivlist}}
\newenvironment{question}[2][Question]{\begin{trivlist}
\item[\hskip \labelsep {\bfseries #1}\hskip \labelsep {\bfseries #2.}]}{\end{trivlist}}
\newenvironment{corollary}[2][Corollary]{\begin{trivlist}
\item[\hskip \labelsep {\bfseries #1}\hskip \labelsep {\bfseries #2.}]}{\end{trivlist}}
 
\begin{document}
 
% --------------------------------------------------------------
%                         Start here
% --------------------------------------------------------------
 
\title{MATC37 Assignment 4}
\author{Anmol Bhullar}
 
\maketitle


\begin{problem}{1}
    Given a collection of Lebesgue measurable sets $F_1,\hdots,F_n\subset\R^d$, construct another collection
    $F_1^*,\hdots,F_N^*$ of Lebesgue measurable sets with $N = 2^n - 1$ such that $\cup_1^n F_k = \cup_1^N F_j^*$, the
    $F_j^*$'s are disjoint and $F_k = \cup_{F_j^*\subset F_k} F_j^*$.
\end{problem}


\begin{solution}{}

    The collection $A:= \{F_1^{'} \cap \hdots \cap F_n^{'}: F_i^{'} = F_i$ or $F_i^{'} = F_i^c\}$ is clearly of
    cardinality $2^n$ since in each element of $A$, there are intersections of $n$ elements from 
    $\{F_1,\hdots,F_n,F_1^c,\hdots,F_n^c\}$ where each element of the intersection
    can be one of two choices $F_i^c$ or $F_i$. Remove $F_1^c \cap F_2^c \cap \hdots \cap F_n^c$ from $A$ so 
    that $A$ then has cardinality of $2^n-1$. We claim that $A$ is the desired collection of Lebesgue measurable 
    sets. First, of all it is clear that each element of $A$ is measurable since it is the intersection of
    measurable sets with intersections of complements of a measurable set (which is still measurable).\\
    
    First, we show that the union of all elements in $A$ is equal to $\cup_1^n F_k$.\\
    Choose $x\in F_k\subseteq \cup_1^n F_k.$ If $x\in F_k$, then for all $1\leq i\leq n$, $x\in F_i$ or $x\in F_i^c$
    (recall $B \cup B^c = \R^d$ for any $B\subseteq\R^d$). Thus, we can find an element in $A$ such that
    $x\in F_1^{'} \cap F_2^{'} \cap \hdots \cap F_n^{'}$ with the property that for each $F_i^{'}$, if $x\in F_i$,
    then $F_i^{'}=F_i$ and similarly for the complement case. Therefore, $x\in\cup A$.\\
    Now, choose $x\in F_j^{*}\subseteq \cup A$. Write $F_j^{*} = F_1^{'}\cap\hdots\cap F_n^{'} = F_j^{*}$. 
    Since we removed the set $F_1^c \cap \hdots \cap F_n^c$ from $A$, there exists some $i$ such that
    $F_i^{'} = F_i$ and so $x\in F_i$. But then since $F_i\in\cup_1^n F_k$, we have that $x\in\cup_1^n F_k$.\\
    Therefore, $\cup_1^n F_k = \cup A$ as wanted.\\
    
    Now, we show $A$ is a collection of pairwise disjoint elements. Pick $F_j^{*}$ and $F_k^{*}$ in $A$ such that $j\neq k$.
    Write $F_j^{*} = a_1 \cap \hdots \cap a_n$ and $F_k^{*} = b_1 \cap \hdots \cap b_n$ where for each $a_i, b_i$, we have that
    $a_i = F_i^c$ or $a_i = F_i$ and similarly for each $b_i$. If $F_j^{*}\neq F_k^{*}$, then by this construction of $F_j^*,F_k^*$, there
    exists some $1\leq i\leq n$ such that $a_i\neq b_i$. Thus, if $x\in F_k^{*}$, then $x\in b_i$ and consequently $x\not\in a_i$
    so that $x\not\in a_1\cap \hdots\cap a_n$. In particular, for every $x\in F_k^{*}$, $x\not\in F_j^{*}$ and a quick repetition
    of this argument shows that if $x\in F_j^{*}$, then $x\not\in F_k^{*}$ implying that $F_j^{*}\cap F_k^{*}=\emptyset$ as wanted.
    Thus, all the $F_j^{*}$ (i.e. elements of $A$) are disjoint.\\
    
    It is left to show $F_k = \cup_{F_j^{*}\subset F_k} F_j^{*}$ for all $k$. For any $F_j^{*}$ in $A$, $F_j^{*}\subset F_k$
    if and only if the $k$th intersecting element of $F_j^{*}$ is equal to $F_k$ i.e. if we can write
    $F_j^{*} = F_1^{'} \cap \hdots \cap F_n^{'}$, then $F_k^{'} = F_k$. Choose some $x\in F_k$, we find some $F_i^*\in A$ such that
    $F_k\subseteq F_i^{*}$. Consider 
    $F_i^{*} = F_1^{'} \cap F_2^{'} \cap \hdots \cap F_n^{'}$ 
    where $F_k^{'} = F_k$ and for other indices $i$,
    the choice of $F_i^{'}$ is arbitrary. If $x\in F_i^{*}$, we are done. So assume $x\not\in F_i^{*}$, then $x\not\in F_t^{'}$ for some
    $1\leq t\leq n$ but $t\neq k$. For these choices of $t$, pick a new element from $A$ where $F_k^{'}$ is the same but
    we set $F_t^{'}$ to its complement. Then $x$ is an element of this new choice of element of $A$. This implies 
    $F_k\subseteq\cup_{F_j^*\subset F_k} F_j^*$. The subset containment for the other direction follows directly from the
    fact that $\cup_{F_j^*\subset F_k} F_j^*$ is a union of sets contained in $F_k$. Thus, $\cup_{F_j^*\subset F_k} F_j^* = F_k$
    as wanted.
\end{solution}

\begin{problem}{2}

    Let $\phi:\R^d\to\R$ be a simple function, and let $\phi = \sum_1^n \hat{a_i}\chi_{\hat{E_i}}$ be its canonical
    representation.
    
    \begin{enumerate}
        \item Prove that if $\phi = \sum_1^N a_k\chi_{E_k}$ is another representation of the simple function $\phi$, where
        the $E_k$'s are disjoint finite measure sets, but the $a_k$'s are not necessarily distinct or nonzero, then
        $\sum_1^N a_km(E_k) = \sum_1^n \hat{a_i}m(\hat{E_i})$.
        
        \item Prove that if $\phi = \sum_1^N a_k\chi_{E_k}$ is another representation of the simple function $\phi$, where
        the $E_k$'s are finite measure sets but not necessarily disjoint and the $a_k$'s are not necessarily distinct or
        nonzero, then $\sum_1^N a_km(E_k) = \sum_1^n \hat{a_i}m(\hat{E_i})$. 
    \end{enumerate}
\end{problem}

\begin{solution}
    sLet $\phi = \sum_{k=1}^N a_k\chi_{E_k}$ where the collection $E_1,\hdots,E_k$ of Lebesgue measurable sets are disjoint but
    not all $a_k$'s are distinct and some may be zero. If some $a_i$ are zero, then we can simply remove the set $E_i$ and the constant
    $a_i$ from our summation since it does not impact the sum. Thus, we have a new collection $E_1,\hdots,E_M$ of disjoint Lebesgue
    measurable sets (for $M\leq N$) such that $\phi = \sum_{k=1}^M a_k\chi_{E_k}$. Next, since every $E_k$ is disjoint from $E_j$
    ($j\neq k$), then $\phi(x)$ for $x\in\R^d$ is in some specific $E_i$ for some $i$ i.e. $\phi(x) = a_i\chi_{E_i}$. Now, suppose
    that there is some $a_i = a_j$ for $i\neq j$. Thus, even if $y\in E_j$ and $x\in E_i$, we have that $\phi(x) = \phi(y)$.
    We can then simplify our summation by unioning $E_i$ and $E_j$ (call it $E_{i'}$) with constant $a_{i'}=a_i=a_j$. Thus, we have
    \[ \phi = \sum_{k=1}^{M-2} a_k\chi_{E_k} + a_{i'}\chi_{E_{i'}} \]
    For aesthetic purposes, we can just write $\phi = \sum_{k=1}^{M-1} a_k\chi_{E_k}$. Repeat this process until every $a_k$
    is distinct. Thus, we obtain for some $J\leq M$: $\phi = \sum_{k=1}^J a_k\chi_{E_k}$. Note,
    \[ \sum_{i=1}^N a_im(E_i) = \sum_{k=1}^J a_km(E_k) \]
    where the summation in the left side is the summation over the old not necessarily non-zero, distinct constants and the right
    side is the new simplified summation over the new collection of distinct, non-zero constants. The equality holds because
    if $a_j = a_i$ as discussed above, then $a_{i'}m(E_{i'}) = a_{i'}m(E_i + E_j) = a_im(E_i) + a_jm(E_j)$.
    where the apostrophe is again used to signify we are referring to the indices from the new simplified representation of $\phi$.
    Thus, the equality above holds. Note, then $\phi = \sum_1^J a_k\chi_{E_k}$ is a representation where all constants are
    non-zero and distinct and the sets $E_k$ are all disjoint. Thus, this is a canonical representation of $\phi$. However,
    canonical representations of $\phi$ are clearly unique since we would have $a_i = \hat{a_j}$ (up to reordering of indices) and
    $E_i = \phi^{-1}\{a_i\}$ which is exactly how the collection $\hat{E_1},\hdots,\hat{E_n}$ is defined. Since these are then the same
    collection (up until reordering), we have that,
    \[ \sum_{i=1}^N a_im(E_i) = \sum_{k=1}^J a_km(E_k) = \sum_{e=1}^n \hat{a_e}m(\hat{E_e}) \]
    as wanted.\\


    For the second part, we know $E_1,\hdots, E_N$ are Lebesgue measurable sets (otherwise $m(E_k)$ is meaningless). By question 1,
    we know there exists some other collection of Lebesgue measurable sets $E_1^{'},\hdots,E_F^{'}$ where $F = 2^{N} -1$ and has all of
    the properties question 1 requires. We can then write $\phi = \sum_{i=1}^F g_i m(E_i^{'})$ for constants $g_i$ which are yet to
    be defined.\\

    We know $E_k = \cup_{E_j^{'}\subset E_k} E_j^{'}$ so $m(E_k) = m(\cup_{E_j^{'}\subset E_k} E_j^{'})$ and this is equal to
    $\sum_{j:E_j^{'}\subset E_k} m(E_j^{'})$ because of countable additivity and disjoint property of the new collection. Thus,
    \[ \phi = \sum_{k=1}^{N} a_km(E_k) = \sum_{k=1}^{N} a_k\Big{(}\sum_{j: E_j^{'}\subset E_k} m(E_j^{'})\Big{)} 
        = \sum_{k=1}^N \Big{(}\sum_{j: E_j^{'}\subset E_k} a_km(E_j^{'})\Big{)}\]
    In the right side of the equation above, we are summing over finitely many indices, thus, we can rewrite this as (for some $R$)
    \[ \sum_{j=1}^R a_jm(E_j^{'}) \]
    where this new summation is not over disjoint $E_j^{'}$'s. However, if for some $k\neq j$, $E_j^{'}\cap E_j^{'}\neq\emptyset$,
    then $E_j^{'} = E_k^{'}$. Thus, we can write this as $a_jm(E_j^{'}) + a_km(E_k^{'}) = (a_j + a_k)m(E_j^{'})$ 
    since $m(E_k^{'}) = m(E_j^{'})$. Thus, we can again rewrite the sum above as (for some $R'$)
    \[ \phi = \sum_{j=1}^{R'} a_jE_j^{'} \]
    where then all $E_1^{'},\hdots,E_{R'}^{'}$ are all disjoint but the constants $a_j$ are not and may not be all non-zero.
    By the previous part of this question, this is enough to imply that
    \[ \sum_{j=1}^{R'} a_jm(E_j^{'}) = \sum_{i=1}^n \hat{a_i}m(\hat{E_i}) \]

\end{solution}
\pagebreak

\begin{problem}{3}
    Let $E\subset\R$ be a measurable set wtih $m(E)<\infty$. Recall from class that if $f:E\to\R$ is a measurable function
    with $|f|\leq M$, then we defined $\int_E f := \int_E \phi_n$ where $\phi_n$ is any sequence of simple functions which
    converges point wise to $f$ and satisfies $|\phi_n|\leq M$.
    
    \begin{enumerate}
        \item Prove that if $f,g:E\to\R$ are bounded measurable functions with $f\leq g$ then $\int_E f\leq \int_E g$
        \item Prove that $f:E\to\R$ is a bounded Lebesgue measurable function then $|\int_E f|\leq \int_E |f|$
    \end{enumerate}
\end{problem}

\begin{solution}

    For the first part, take some sequence of simple functions $\phi_n \to f$ and $\psi_n \to g$. We first, want to show that
    for some $N>0$, we have for all $n>N$ that $\phi_n \leq \psi_n$. To show this, suppose this is not the case. Then for all $N$,
    there exists some $n>N$ such that $\phi_n > \psi_N$. Then for all $x\in E$, we have $\phi_n(x) > \psi_N(x)$. However, this is
    a contradiction because of the following: Not considering the case where $f=g$ for all $x\in E$, we have that for some
    $x\in E$, $f(x) < g(x)$. Choose any positive $\epsilon$ less than $[g(x)-f(x)]/2$. Then, since $\phi_n \to f$, 
    there exists some $N_1>0$ such that for all $n_1>N_1$, we have $|\phi_{n_1}(x) - f(x)|<\epsilon$ and 
    similarly for $\psi_n \to g$, we have for some $N_2>0$ such that for all $n_2>N_2$, we have $|\psi_{n_2}(x) - g(x)|<\epsilon$. 
    Choose $N = \max\{N_1,N_2\}$. Then, for all $n\geq N$:
    \[ |\phi_n(x) -f(x)| < \epsilon \qquad |\psi_n(x) - g(x)| < \epsilon \]
    which implies from our choice of $\epsilon$ that $\psi_{N}(x) > \phi_n(x)$ which is absurd. Thus, either $f(x) = g(x)$ for all $x$
    or there exists some $N>0$ such that for all $n>N$, we have $\phi_n \leq \psi_n$. \\
    If the former is true, then the question is trivial, so assume the latter. Since each $\phi_n$ and $\psi_n$ are simple functions,
    from a result in class that, \[ \int_E \phi_n \leq \int_E \psi_n \]
    Combining this with the fact that this inequality holds for all $n>N$, we have:
    \[ \lim_{n\to\infty} \int_E \phi_n \leq \lim_{n\to\infty} \int_E \psi_n \]
    which is enough to imply $\int_E f \leq \int_E g$ from our definition of lebesgue integration for bounded measurable functions.\\
    
    Let $\phi_n$ be a sequence of functions so that $\phi_n \to f$. Then by a simple $\epsilon-N$ proof, we know $|\phi_n|\to |f|$ since
    \[ ||\phi_n(x)|-|f(x)|| \leq |\phi_n(x) - f(x)|\]
    by the reverse triangle inequality. Therefore, if $\lim_{n\to\infty} \int_E \phi_n = \int_E f$, we know that
    $\lim_{n\to\infty} \int_E |\phi_n| = \int_E |f|$. Additionally, note for simple functions we have that:
    \[ |\int_E \phi_n| \leq \int_E |\phi_n| \]
    Thus,
    \[ |\int_E f| = \lim_{n\to\infty} |\int_E \phi_n| \leq \lim_{n\to\infty} \int_E |\phi_n| = \int_E |f| \]
    as wanted.
    
\end{solution}
\pagebreak

\begin{problem}{4}
    Let $f:E\to\R$ be a bounded measurable function defined on a measurable domain $E\subset\R^d$ with $m(E)<\infty$.
    \begin{enumerate}
        \item Show that if $g:E\to\R$ is bounded and $g(x)=f(x)$ for a.e. $x\in E$, then $\int_E g = \int_E f$.
        \item Show that if $f\geq 0$ and $\int_E f = 0$, then $f(x) = 0$ for a.e. $x\in E$.
    \end{enumerate}
\end{problem}

\begin{solution}
    FFor the first part, define $h(x) := f(x) - g(x)$. Then $h(x) = 0$ for a.e. $x\in E$ by choice of $f$ and $g$.
    If $\phi_n \to f$ and $\psi_n \to g$, then 
    \[ h(x) = f(x) - g(x) = (\lim_n \phi_n(x)) - (\lim_n \psi_n(x)) = \lim_n (\phi_n - \psi_n)(x) \]
    so that $\phi_n - \psi_n \to h$. In particular, by Lemma 1.2 in Chapter 2, we have that since $h$ is 0 a.e. for every $x\in E$,
    then:
    \[ \lim_n \int_E (\phi_n - \psi_n) \to 0 \quad \implies \quad \int_E h = 0 \]
    Since $\phi_n$ and $\psi_n$ are simple functions, we know that $\int_E (\phi_n - \psi_n) = \int_E \phi_n - \int_E \psi_n$ so
    in particular,
    \[ \lim_n [(\int_E \phi_n) - (\int_E \psi_n)] \to 0 \quad \implies \lim_n \int_E \phi_n - \lim_n \int_E \psi_n \to 0\]
    so that in particular, $\int_E f - \int_E g = 0$ so that $\int_E f = \int_E g$ as wanted.\\

    For the second part, we know $f$ is non-negative so if we write $E = E' \cup E^2$ where $E^2$ is the set where
    $f$ is non-zero and $E'$ is the set where $f$ is identically zero, then we have by proposition 1.3 (chapter 2) that 
    $\int_E f = \int_{E'} f + \int_{E^2} f$ (since $E' = \{f = 0\}$ and $E^2 = \{f > 0\}$ are clearly disjoint).
    Clearly, we have $\int_{E'} f = 0$ since $f$ is exactly zero on $E'$ (if one wishes to be rigorous in this, 
    then choose the sequence of identically zero simple functions, these converge to $f$ on $E'$ and all have integral zero). 
    Thus, we obtain $\int_{E^2} f = 0$ since $\int_E f = 0$. Find some sequence of functions $\{\phi_n\} \to f$ on $E^2$ such that
    each $\phi_n$ is non-negative, and the sequence is increasing (possible by Theorem 4.1 in Chapter 1). Since $f$ is non-zero
    (and thus, positive) on $E^2$, if we assume $m(E^2)>0$, then by the Lemma below, we obtain: $\int_{E^2} f > 0$. This is absurd
    as we have already established $\int_{E^2} f = 0$. Thus $m(E^2) = 0$. This implies $f(x) = 0$ for a.e. $x\in E$. As a side
    note, we used proposition 1.3 without showing that $f$ is supported by a set of finite measure. However, this is proven later
    when we show $m(E^2) = 0$ without the assumption of proposition 1.3 holding.
\end{solution}
\pagebreak

\begin{problem}{5}
    Compute the following limits and justify the calculations:
    \begin{enumerate}
        \item $\lim_{n\to\infty} \int_0^1 n\sin(x/n)/x dx$
        \item $\lim_{n\to\infty} \int_0^{\infty} (1-x/n)^nx^2 dx$
    \end{enumerate}
\end{problem}

\begin{solution}
    LLet $\{n\sin(x/n)/x\}_{n=1}^{\infty}$ be a sequence of functions on $0\leq x\leq1$. This converges, by the following computation:
    \[
        \lim_{n\to\infty} \frac{n\sin(x/n)}{x} = \lim_{n\to\infty} \frac{\sin(x/n)}{x/n}
    \]
    which clearly goes to $0/0$. Applying L'hopital's rule, we get:
    \[ \lim_{n\to\infty} \frac{n\sin(x/n)}{x} = \lim_{n\to\infty} \frac{\cos(x/n)(d/dn)(x/n)}{d/dn(x/n)} = \cos(0) = 1 \]
    Thus, $\{n\sin(x/n)/x\}_n \to \mathbb{1}$ where $\mathbb{1}$ is the constant function mapping to 1. Note also that each
    function in the sequence is bounded and has finite support since $\sin(x/n)/(x/n) \leq 1/(x/n) \leq 1$ (since $E = [0,1]$).
    Thus, we can say by the bounded convergence theorem that,
    \[ \int_0^1 \lim_{n\to\infty} \frac{n\sin(n/x)}{x} = \int_0^1 1 = 1\]
    We know $\int_0^1 = 1$ since the Riemann integral and the Lebegue integral agree on closed intervals 
    Thus, they agree on $[0,1]$ and since the Riemann integral tells us $\int_0^1 1 = 1$, we have 
    that $\int_0^1 1 = 1$ in the Lebesgue integral so that,
    \[ \lim_{n\to\infty} \int_0^1  \frac{n\sin(x/n)}{x} = 1 \]
    as wanted.\\

    We show that $\int_0^{\infty} (1-x/n)^nx^2dx$ does not exist, even in the Lebesgue integral sense. Note that if $n$ is odd,
    we have that for all $x\geq n$, $(1-x/n)^n$ goes to $-\infty$. Since we are taking $n$ to $\infty$, we know that for a large
    enough $n$, $(1-x/n)^nx^2$ would then also go to $-\infty$. Thus, this function does not have finite support so it's
    Lebesgue integral is not defined. Similarly, for $n$ even, we have that for all $x>n$, $(1-x/n)^n$ goes to $+\infty$, and so
    for large enough $n$, $(1-x/n)^nx^2$ would also go to $+\infty$ and so it also does not have finite support. Thus, its
    Lebesgue integral is also not defined. Therefore, the expression $\lim_{n\to\infty} \int_0^{\infty} (1-x/n)^nx^2dx$ cannot
    be computed as it is absurd.
\end{solution}
\pagebreak

\begin{problem}{6}
    Let $f:\R^d\to\R$ be an integrable function such that $\int_E f(x)dx \geq 0$ for every measurable set $E$
    \begin{enumerate}
        \item Prove that $f(x)\geq 0$ for a.e. $x$
        \item Prove that if we assume in addition that $f$ is continuous, then $f(x)\geq 0$ for all $x$.
    \end{enumerate}
\end{problem}

\begin{solution}
    FFor the first part, the proof is done in a somewhat similar style to question 4.2. Thus, write $E = E' \cup E^2$ where
    $f$ is non-negative on $E'$ and negative on $E^2$. Note that $f$ is measurable on $E'$ and $E^2$ because $f$ is measurable on
    $\{f \geq 0\}$ and $\{f < 0\}$. Thus, we have that $0 \leq \int_E f = \int_{E'} f + \int_{E^2} f$. Note since $f$ is non-negative
    on $E'$, it follows that $0\leq \int_{E'} f$ so we must have that $0\leq \int_{E^2} f$. Intuitively, we must have that
    $m(E^2) = 0$ since if it were not, then $f$ is negative on every $x\in E^2$ but $\int_{E^2} f\geq 0$ which is absurd.
    Assume $m(E^2)\neq 0$. We know that $f$ is negative on $E^2$ so that $-f$ is positive on $E^2$. By the lemma, it follows
    that $\int_{E^2} -f > 0$. By linearity, $-(\int_{E^2} f) > 0$ so that $\int_{E^2} f < 0$ which is absurd. Thus, $m(E^2) = 0$
    so that $f\geq 0$ for a.e. $x\in E$.\\

    We know $m(E^2) = 0$ so in particular, $E^2$ does not contain any open neighbourhoods (as they are all of positive Lebesgue measure).
    Having noted this, now assume $f$ is continuous on $E$ and so it is continuous on $E^2$. Assume $x\in E^2$ exists. We note
    that $E^2$ is open since $E^2 = \{f < 0\} = f^{-1}((-\infty,0))$ (we are using the topological definition of continuity here).
    Since $E^2$ is open, we then have the existence of an open neighbourhood centered around $x$ entirely contained in $E^2$. This
    is a contradiction as discussed above. Thus, we have that $E^2 = \emptyset$, so in particular, $f(x)\geq 0$ for all $x\in E$.
\end{solution}

\begin{lemma}
    IIf $f$ is positive on some measurable domain $F$ (with $m(F)>0$), then $\int_F f > 0$.
\end{lemma}
\begin{proof}
    Let $F_n = F \cap \{f > 1/n\}$. Then $\cup_n F_n = F$ since as $n\to\infty$, we know $\{f > 1/n\}\to \{f>0\} = F$
    since $f$ is positive on $F$. We know $m(F)>0$ so it follows for some $n$, $m(F_n)>0$. For this $n$, we have that
    if $x\in F_n$, then $f(x) \geq 1/n$ just by definition of $\{f > 1/n\}$. Thus, $\int_{F_n} f \geq 1/n\cdot m(F_n) > 0$.
    Since $F_n\subseteq F$ with both $F_n$ and $F$ being measurable, then by monotonicity $\int_F f \geq \int_{F_n} f > 0$ as wanted.
\end{proof}


\end{document}
