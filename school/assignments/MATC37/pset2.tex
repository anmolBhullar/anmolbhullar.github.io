\documentclass{article}
% \usepackage{tikz}
% \usetikzlibrary{cd}
\usepackage[utf8]{inputenc}
\usepackage[english]{babel}
\usepackage{amsfonts}
\usepackage{amsthm}
\usepackage{amsmath}
\usepackage{amssymb}
\usepackage{nicefrac}

\newtheorem{theorem}{Theorem}
\newtheorem{es}{Examples}

\newcommand{\inter}[1]{int(#1)}
\newcommand{\norm}[1]{\left\lVert#1\right\rVert}

\title{MATC37: Introduction to Real Analysis\\
    Assignment 2}
\author{Anmol Bhullar - 1002678140}

\begin{document}
    \maketitle


    \textbf{P1.}\\

    Let $m_*$ be the (Lebesgue) outer measure on $\mathbb{R}^d$ as defined in lecture.
    \begin{enumerate}
        \item Show that $m_*(\mathbb{R}^d) = \infty$.
        \item Give an example of an unbounded set $A\subseteq\mathbb{R}^d$ with $m_*(A) < \infty$.
    \end{enumerate}

    \textbf{Solution.}
    \begin{enumerate}
        \item Choose a arbitrary countable open cover $\{Q_i\}$ of $\mathbb{R}^d$. Since each $Q_i\subseteq\mathbb{R}^d$, it follows
            \[ \bigcup_{n=1}^{\infty} Q_i \subseteq \mathbb{R}^d \]
            which implies $m_*(\cup_{n=1}^{\infty} Q_i) \leq m_*(\mathbb{R}^d)$ by monotonicity. Also by the definition of an open 
            cover, we have that
            \[ \mathbb{R}^d \subseteq \bigcup_{n=1}^{\infty} Q_i \]
            which again implies by monotinicity, the result: $m_*(\mathbb{R}^d)\leq m_*(\cup_{n=1}^{\infty} Q_i)$. Therefore,
            \[ m_*(\bigcup_{n=1}^{\infty} Q_i) = m_*(\mathbb{R}^d) \]
            In particular, the outer measure of $\mathbb{R}^d$ is equal to the outer measure of \textit{any} of its open covers.
            Therefore, choose the open cover $\{B_n(0): n\in\mathbb{N}\}$ where $B_n(0)$ is the open cube of side length $n$
            centered at 0. Suppose by contradiction,
            \[ m_*(\bigcup_{n=1}^{\infty} B_n(0)) = k\in\mathbb{R}^{\geq 0} \]
            Then, there exists some $i\in\mathbb{N}$ such that $i^d > k$, note $B_i(0)\in\{B_n(0)\}$ and $B_i(0)$ clearly has
            measure $i^d$, therefore, by monotonicity, $i^d = m_*(B_i(0)) \leq m_*((\cup_n B_n(0))) = k$ so that $i^d < k < i^d$
            which is a contradiction, therefore no such $k$ exists, thus
            \[ m_*(\bigcup_{n=1}^{\infty} B_n(0)) = \infty \implies m_*(\mathbb{R}^d) = \infty \]
        \item It is clear that $\mathbb{Q}$ is unbounded and $\mathbb{Q}$ is countable. In class we proved that all countable
            sets have outer measure 0. Therefore, $\mathbb{Q}$ is an example of an unbounded set which has the property
            \[ m_*(\mathbb{Q}) = 0 < \infty \]
    \end{enumerate}

    \textbf{P2.} \\

    Let $m_*$ be the (Lebesgue) outer measure on $\mathbb{R}^d$.
    \begin{enumerate}
        \item Compute $m_*(\mathbb{Q}^d\cap [0,1]^d)$.
        \item Compute $m_*([0,1]^d$\textbackslash$\mathbb{Q}^d)$.
    \end{enumerate}

    \textbf{Solution.}
    \begin{enumerate}
        \item Note $\mathbb{Q}^d\subseteq[0,1]^d$ is not true. With this consider:
            \[ m_*(\mathbb{Q}^d\cap[0,1]^d) \leq m_*(\mathbb{Q}^d) = 0 \]
            by monotonicity and the fact that $\mathbb{Q}$ is countable so finite cross products of $\mathbb{Q}$ must be countable
            as well.
        \item We have shown above that $m_*([0,1]^d\cap\mathbb{Q}^d) = 0$, then, note:
            \begin{align*}
                1 = m_*([0,1]^d) &= m_*(([0,1]^d\cap\mathbb{Q}^d)\cap([0,1]^d\backslash\mathbb{Q}^d)) \\
                &\leq m_*([0,1]^d \cap \mathbb{Q}^d) + m_*([0,1]^d\backslash\mathbb{Q}^d) \\
                &= 0 + m_*([0,1]^d\backslash\mathbb{Q}^d)\\
                &:= n
            \end{align*}
            so that $1\leq n$. But, by monotonicity,
            \[ n = m_*([0,1]^d\backslash\mathbb{Q}^d) \leq m_*([0,1]^d) = 1 \]
            Therefore, $1 = n$, and $m_*([0,1]^d\backslash\mathbb{Q}^d) = 1$.
    \end{enumerate}

    \textbf{P3.} \\

    Let $m_*$ be the (Lebesgue) outer measure on $\mathbb{R}^d$ and let $A\subseteq[0,1]^d$
    \begin{enumerate}
        \item Prove or disprove: If $\overline{A} = [0,1]$ then $m_*(A) = 1$.
        \item Prove or disprove: If $m_*(A) = 1$ then $\overline{A} = [0,1]^d$.
    \end{enumerate}

    \textbf{Solution.}
    \begin{enumerate}
        \item This is false. Let $A = \mathbb{Q}^d\cap [0,1]^d$. Because $\overline{\mathbb{Q}^d} = \mathbb{R}^d$, it follows
            $\overline{A} = [0,1]^d$. However, as noted in the solution of question 2 above, $m_*(A) = 0 \neq 1$.
        \item Note $[0,1]^d = A\cup([0,1]^d\backslash A)$. Choose $x\in [0,1]^d\backslash A$ such that $x$ is not a limit point
            of $A$ (if none exist, then we are done since $\overline{A} = A \cup \partial{A}$ and since $[0,1]^d$ is closed,
            no limit points of $A$ would exist outside of it). Since $x$ is not a limit point, by definition, there exists some
            $r>0$ such that the open ball $B_r(x)$ does not intersect $A$ and we can always decrease our $r$ so that $B_r(x)$
            is completely contained in $[0,1]^d$ (the only case this fails is if $x$ is the boundary point of $[0,1]^d$ but then
            we can simply choose some other $y\in B_r(x)$ so that $y\neq x$ and $y$ is not a boundary point of $[0,1]^d$). Also,
            by decreasing $r$, we can ensure that $d(A,B_r(x)) > 0$. Because of this, we have that $m_*(A\cup B_r(x))
            = m_*(A) + m_*(B_r(x)) = 1 + \pi r^2 > 1$. Furthermore, since $B_r(x)\subseteq [0,1]^d$ (by construction of $r$)
            and $A\subseteq [0,1]^d$ (given), we have that $A\cup B_r(x)\subseteq [0,1]^d$ so by monotonicity,
            $m_*(A\cup B_r(x)) = m_*(A) + m_*(B_r(x)) \leq m_*([0,1]^d) = 1$ so that $1 > 1$ which is a contradiction. Therefore,
            no such $x\in [0,1]^d\backslash A$ exists and all $x$ are limit points of $A$. Therefore, $[0,1]^d$ consists of points
            of $A$ and limit points of $A$ and since $[0,1]^d$ is closed, we have that $\overline{A} = [0,1]^d$ (the closed
            condition allows to conclude there are no limit points of $A$ outside of $[0,1]^d$).
    \end{enumerate}

    \textbf{P4.} \\

    Let $m_*$ the (Lebesgue) outer measure on $\mathbb{R}^d$.
    \begin{enumerate}
        \item For $A\subseteq\mathbb{R}^d$ and $v\in\mathbb{R}^d$ consider the translated set $A + v := \{a+v: a\in A\}$.
            Prove that $m_*(A+v) = m_*(A)$.
        \item For $A\subseteq\mathbb{R}^d$ and $\delta > 0$ consider the dilated set $\delta\cdot A := \{\delta\cdot a: a\in A\}$.
            Prove that $m_*(\delta\cdot A) = \delta^dm_*(A)$.
    \end{enumerate}

    \textbf{Solution.}
    \begin{enumerate}
        \item Denote $n := m_*(A) = \inf\{\sum_{j=1}^{\infty} |Q_j|\}$ for closed cubes $Q_j$ such that $\{Q_j\}$ covers $A$.
            By definition of infimum, for all $\epsilon>0$, there exists some countable cover $\{Q_j\}$ of $A$ such that
            \[ \sum_{j=1}^{\infty} |Q_j| < n + \epsilon \]
            We now prove that $|Q_j| = |Q_j + v|$ for any $v\in\mathbb{R}^d$. It is clear that $Q_j+v$ is a linear transformation
            of $Q_j$ (in particular, a translation), thus the sidelengths of $Q_j+v$ remain similar to $Q_j$. 
            Thus, it follows $|Q_j| = |Q_j+v|$. Thus, the above inequality becomes:
            \[ \sum_{j=1}^{\infty} |Q_j + v| < n + \epsilon \]
            so that for all $\epsilon>0$, there exists some countable cover $\{Q_j+v\}$ such that the above inequality holds.
            Now, we claim that $\{Q_j+v\}$ is a cover of $A+v$. Pick any $x\in A+v$, then we can write $x=y+v$ for some $y\in A$.
            Since $\{Q_j\}$ covers $A$, there exists some $k\in\mathbb{N}$ such that $y\in Q_k$, then $x=y+v\in Q_k+v$. Since
            $Q_k+v \in \{Q_j+v\}$, the cover $\{Q_j+v\}$ covers $A+v$. Thus, by definition of infimum:
            \[ n = m_*(A+v) \qquad\text{so that}\qquad m_*(A+v) = m_*(A) \]
            as wanted.
        \item We know $m_*(A) = \inf\{\sum_{n=1}^{\infty} |Q_j|\}$, so then by definition of infimum, we have that for all
            $\epsilon>0$, there exists some countable cover $\{Q_j\}$ of $A$ such that:
            \[ \sum_{j=1}^{\infty} |Q_j| < m_*(A) + \epsilon \]
            Now for any $\delta>0$, it follows:
            \[ \delta^d(\sum_{j=1}^{\infty} |Q_j|) < \delta^dm_*(A) + \delta^d\epsilon \implies \sum_{j=1}^{\infty} \delta^d|Q_j|
                < \delta^dm_*(A) + \delta^d\epsilon\]
            Now, we show that $\delta^d|Q_j| = |\delta\cdot Q_j|$. Note, $\delta\cdot Q_j = [\delta a, \delta a]^d$ where
            $Q_j = [a,a]^d$ by definition of an open cube in $\mathbb{R}^d$. Therefore, $|\delta\cdot Q_j| = (\delta a)^d =
            \delta^d (a^d) = \delta^d |Q_j|$. Since $\delta^d>0$ is fixed, we have obtained the result that for all
            $\nicefrac{\epsilon}{\delta^d} > 0$, there exists some open countable cover $\{\delta\cdot Q_j\}$ of $\delta\cdot A$
            such that
            \[ \sum_{j=1}^{\infty} |\delta\cdot Q_j| < \delta^dm_*(A) + \epsilon \]
            which shows that $\delta^d m_*(A) = \inf\{\sum_{j=1}^{\infty} |\delta Q_j|\}$.
    \end{enumerate}

    \textbf{P5.} \\

    Suppose $\{l_k\}_{k\geq 1}$ is a sequence of positive numbers such that \\$\sum_{k=1}^{\infty} 2^{k-1}l_k < 1$. Consider the
    fat Cantor-like set $\hat{C}$ where at the $k$-th stage of the construction one removes $2^{k-1}$ centrally situated open intervals
    each of length $l_k$. Compute $m_*(\hat{C})$.\\

    \textbf{Solution.}\\

    If at each iteration $C_k$ we are removing $2^{k-1}$ centrally situated open intervals, each of length (read: measure) $l_k$, 
    then since $\hat{C} = \cap_k \hat{C_k}$, it is clear that within $\hat{C}$, we have removed $\sum_{k=1}^{\infty} 2^{k-1}$ 
    centrally situated open intervals (from $[0,1]$) which have a total measure of $\sum_{k=1}^{\infty} 2^{k-1}l_k$. Since we are given 
    $\sum_{k=1}^{\infty} 2^{k-1}l_k < 1$, and $m_*([0,1])<1$, then it is clear:
    \[ m_*(\hat{C}) \leq 1 \]
    Furthermore, we know $m_*(\hat{C}) \geq 0$ since $(l_k)$ is a sequence of \textit{positive} real numbers (in particular, this means
    $l_k\neq 0$ for all $k\in\mathbb{N}$) so that we are removing $\sum_k 2^{k-1}$ open intervals from $\hat{C}$ each of non-zero
    length. Finally, since we are removing measure $\sum_{k=1}^{\infty} 2^{k-1}l_k$ from $[0,1]$ to obtain the measure of $\hat{C}$,
    it follows:
    \[ m_*(\hat{C}) = 1 - \sum_{k=1}^{\infty} 2^{k-1}l_k \]

    \textbf{P6.} \\

    For $A\subseteq\mathbb{R}^n$, the $d$-dimensional Hausdorff exterior measure is defined by
    \[ h_d^*(A) = \lim_{\delta\to 0}\big{(}\inf\Big{\{}\sum_{k=1}^{\infty} (\text{diam}(B_k))^d: \{B_k\}\;\text{covers}\: A,\;
        \text{diam}(B_k) < \delta\Big{\}}\big{)} \]
    \begin{enumerate}
        \item Prove that $h_d^*(A)\in[0,\infty]$ always exists
        \item Prove that $A\subseteq B$, then $h_d^*(A)\subseteq h_d^*(B)$
        \item Prove that if $A_1,A_2,\hdots\subseteq\mathbb{R}^n$, then $h_d^*(\cup_{i=1}^{\infty} A_i) \leq \sum_{i=1}^{\infty} 
            h_d^*(A_i)$
    \end{enumerate}

    \textbf{Solution.}
    \begin{enumerate}
        \item First, the infimum clearly exists since it is non-empty and is bounded by 0 (the sum is always non-negative)
            and $\infty$, in particular, this implies the infimum may take on values in the interval $[0,\infty]$. Denote,
            \[ h_d^A(\delta_1) := \inf\big{\{}\sum_{i=1}^{\infty}\text{diam}(A_i)^d: A\subseteq\cup_i 
                A_i,\text{diam}(A_i)<\delta_1\big{\}} \]
            so that $h_d^A(\delta_1)$ is a real-valued function and in particular, we show that $h_d^A(\delta_1)\geq h_d^A(\delta_2)$
            for $\delta_1\geq\delta_2$ i.e. $h_d^A$ is a monotone (increasing) function. Let $\gamma_1$ and $\gamma_2$ be the
            notation borrowed from 6b. below (this proof of monotonicity and the one below is similar) where $\gamma_1$ now associates
            with $\delta_1$ and $\delta_2$ to $\gamma_2$. Since $\delta_2\leq\delta_1$, we have that any cover $\{B_k\}\in\gamma_2$
            is also in $\gamma_1$. This implies $h_d^A(\delta_1)\geq h_d^A(\delta_2)$ as wanted (the reason is \textit{very} similar
            to the one in the proof of 6b. below). If one needs more clarification, they
            can refer to the proof of 6b as it follows quite similarly in structure and arguments. Therefore, $h_d^A$ is monotone. 
            Now suppose $h_d^A$ is bounded above (it is always bounded below since the infimum is bounded below by 0) by some
            non-negative $M\in\mathbb{R}$, i.e. for all $\delta'>0$, $h_d^A(\delta') \leq M$, then we can apply the bounded
            monotone convergence theorem to get that $0\leq \lim_{\delta'\to 0} h_d^A(\delta') < \infty$ so that
            $0 \leq h_d^*(A) < \infty$. Clearly if $h_d^A$ is not bounded, then $\lim_{\delta'\to 0} h_d^A(\delta') = \infty$
            so that $h_d^*(A) = \infty$. Therefore, $h_d^*(A)$ always exists and takes some value in the closed interval $[0,\infty]$.

        \item Fix $\delta > 0$. Note,
            \begin{align}
                h_d^*(A) &= \lim_{\delta\to 0}\inf\big{\{}\sum_{k=1}^{\infty} \text{diam}(A_k)^d:A\subseteq\cup_k A_k,
                    \text{diam}(A_k)<\delta\big{\}} \\
                h_d^*(B) &= \lim_{\delta\to 0}\inf\big{\{}\sum_{k=1}^{\infty} \text{diam}(B_k)^d:A\subseteq\cup_k B_k,
                    \text{diam}(B_k)<\delta\big{\}}
            \end{align}
            Since we have fixed $\delta>0$, let us denote the set over which we take the infimum in (1) by $\gamma_1$ and the other,
            by $\gamma_2$. Now, since $A\subseteq B$, then every cover $\{B_k\}\in\gamma_2$ is also in $\gamma_1$. Therefore,
            $\gamma_2\subseteq \gamma_1$ and in particular, inf$(\gamma_2) \geq \inf(\gamma_1)$ since if the opposite is true, then
            we can simply choose some cover $\{B_k\}\in\gamma_2$ such that $\sum_k$ diam$(B_k)^d < \inf(\gamma_2) + \epsilon$ where
            $\epsilon = \nicefrac{\inf(\gamma_1) - \inf(\gamma_2)}{2}$. However, this cover $\{B_k\}$ is also a cover of $A$, so
            we have that $\sum_k$ diam$(B_k)^d < \inf(\gamma_2) + \epsilon < \inf(\gamma_1)$ and this contradicts the notion of
            an infimum. Therefore, for any $\delta>0$, we have that $\inf(\gamma_2)\geq\inf(\gamma_1)$. This being true for every
            $\delta>0$ naturally implies that $\lim_{\delta\to 0} \inf(\gamma_2) \geq \lim_{\delta\to 0} \inf(\gamma_1)$ i.e.
            $h_d^*(B)\geq h_d^*(A)$ as wanted.

        \item If for any $A_i$, $h_d^*(A_i) = \infty$, then the sum on the right is also infinite so that in particular the inequality
            holds. Therefore, assume for no $i\in\mathbb{N}$, $h_d^*(A_i) = \infty$. Similarly, if the infinite sum diverges, the
            inequality is also clearly true (nothing is bigger than $\infty$!), therefore, assume the sum converges. In order
            to prove countable subadditivity, we instead prove the equivalent statement, for all $\epsilon>0$,
            \[ h_d^*(\bigcup_{i=1}^{\infty} A_i) \leq \big{(}\sum_{i=1}^{\infty} h_d^*(A_i)\big{)} + \epsilon \]
            Let $\{A_{i,k}\}_k$ be the collection of countably (in the finite case, simply let the rest be $\emptyset$) many
            subsets of $\mathbb{R}^d$ such that:
            \[ A_i \subseteq \bigcup_{k=1}^{\infty} A_{i,k} \]
            and,
            \[ \sum_{k=1}^{\infty}\:\text{diam}(E_{i,k})^d < h_d^*(A_i) + 2^{-i}\epsilon \]
            We know such a $\{A_{i,k}\}_k$ always exists since it is always possible to cover a set (in this case $A_i$ for every
            $i$) and since $h_d^*(A_i)$ always exists, then the limit (of infimums) must exist, in particular, this must mean
            that for every $\epsilon>0$, there exists some $\sum_{k=1}^{\infty}$ diam$(E_{i,k})^d - h_d^*(A_i) < \epsilon$ (sequence
            definition) and we can rearrange this inequality to obtain the inequality above.\\
            Unioning over every index $i,k$, we obtain that:
            \[ \bigcup_{i=1}^{\infty} A_i \subseteq \bigcup_{i,k=1}^{\infty} A_{i,k} \]
            In particular, this means:
            \[ \sum_{i,k=1}^{\infty}\:\text{diam}(A_{i,k})^d < \sum_{i=1}^{\infty} h_d^*(A_i) + 2^{-i}\epsilon = \sum_{i=1}^{\infty} 
                h_d^*(A_i) + \epsilon \]
            which is as we wanted.
    \end{enumerate}

    \textbf{P7.} \\

    Suppose that $A\subseteq E\subseteq B$, where $A$ and $B$ are measurable sets of finite exterior measure. Prove that if
    $m_*(A) = m_*(B)$, then $E$ is measurable.\\

    \textbf{Solution.}\\

    We can write $E = A \cup (E\backslash A)$. It is given that $A$ is measurable, and so we prove $(E\backslash A)$ is measurable.
    Note, the set of Lebesgue measurable sets is a $\sigma$-algebra, hence $A^c$ is measurable and since $B$ is measurable,
    so is $B\cap A^c = B\backslash A$. Now, we show that $m_(B-A) = 0$. To see this, recall $m_*(B) = m_*(A)$ and so
    $m(A) = m(B)$, and the countable additive property of Lebesgue measurable sets:
    \begin{align*}
        B &= A \cup (B\backslash A) \\
        \implies m(B) &= m(A\cup B\backslash A) \qquad(\text{measure preserves equalities })\\
        \implies m(B) &= m(A) + m(B\backslash A) \\
        \implies m(B) &= m(B) + m(B\backslash A) \\
        \implies m(B\backslash A) &= 0
    \end{align*}
    as wanted. Since $A\subseteq E\subseteq B$, then $E\backslash A\subseteq B\backslash A$ and by monotonicity, we have
    of Lebesgue measurable sets, we obtain: $m(E\backslash A) \leq 0 \implies m(E\backslash A)$ since $m$ is always
    positive definite. It might be worthwhile to note that measure (specifically Lebesgue measure) preserves equalities because
    the cover of the set on the left side works as a cover for the set on the right, and from this we can argue that the infimum
    must be equal because we end up taking the infimum over the same set (of non-negative real numbers) for both sides.
\end{document}
