\documentclass{article}
% \usepackage{tikz}
% \usetikzlibrary{cd}
\usepackage[utf8]{inputenc}
\usepackage[english]{babel}
\usepackage{amsfonts}
\usepackage{amsthm}
\usepackage{amsmath}
\usepackage{amssymb}
\usepackage{nicefrac}

\newtheorem{theorem}{Theorem}
\newtheorem{es}{Examples}
\newtheorem{lemma}{Lemma}

\newcommand{\inter}[1]{int(#1)}
\newcommand{\norm}[1]{\left\lVert#1\right\rVert}

\title{MATC37: Introduction to Real Analysis\\
    Assignment 3}
\author{Anmol Bhullar - 1002678140}

\begin{document}
    \maketitle

    \textbf{P1.}\\

    Let $f: \mathbb{R}^d\to\mathbb{R}$ be measurable. Prove that (a) $-f$ is measurable and (b) $f^2$ is measurable.\\

    \textbf{Solution.}

    Suppose $f$ is measurable. We want to show $-f$ is measurable, thus, it suffices to show for any arbitrary
    $a\in\mathbb{R}$, $\{-f < a\}$ is measurable. Note:
    \begin{align*}
        \{-f < a\} = \{x\in\mathbb{R}^d: -f(x) < a\} = \{x\in\mathbb{R}^d: f(x) > -a\} = \{f > -a\} 
    \end{align*}
    Since $f$ is measurable, we know $\{f > -a\}$ is measurable by a proposition covered in class. Therefore, $\{-f < a\}$
    is measurable which lets us conclude that $-f$ is measurable.\\

    Suppose $f$ is measurable. We want to show $f^2$ is measurable, thus, it suffices to show for any arbitrary $a\in\mathbb{R}$,
    $\{f^2 > a\}$ is measurable. If $a\geq 0$, note:
    \begin{align*}
        \{f^2 > a\} = \{x\in\mathbb{R}^d: f^2(x) > a\} &= \{x\in\mathbb{R}^d: \pm f(x) > \sqrt{a}\} \\
            &= \{x\in\mathbb{R}^d: f(x) > \sqrt{a}\} \cup \{x\in\mathbb{R}^d: -f(x) > \sqrt{a}\} \\
            &= \{f > \sqrt{a}\} \cup \{f < -\sqrt{a}\}
    \end{align*}
    which we know is measurable because $f$ is measurable so both $\{f > \sqrt{a}\}$ and $\{f < -\sqrt{a}\}$ are measurable
    and thus their union must be as well. Now suppose $a < 0$. Since $f^2$ is an even power, it is a positive valued function.
    This means there exist no $x\in\mathbb{R}^d$ so that $f^2(x) < 0$. From this, we can conclude $\{f^2 > a\} = \{f^2 \geq 0\}$.
    We have already shown from the case above that $\{f^2 \geq 0\}$ is measurable. Therefore, for all $a\in\mathbb{R}$,
    we have that $\{f^2 > a\}$ is measurable from which we conclude that $f^2$ is measurable as wanted.$\hfill\blacksquare$

    \newpage

    \textbf{P2.}

    \begin{enumerate}
        \item[(a)] Prove that the function $f:\mathbb{R}\to\mathbb{R}$ defined by $x\mapsto\sin{(1+e^x)}+\nicefrac{1}{1+x^2}$.
        \item[(b)] Prove that every monotone function $f:\mathbb{R}\to\mathbb{R}$ is measurable.
    \end{enumerate}

    \textbf{Solution.}\\

    (a) $\sin{1+e^x}$ is clearly a composition of continuous functions and $1/(1+x^2$ is also clearly continuous,
    thus their sum must be continuous as well. All continuous functions are measurable by Property 2 on page 29 of the class
    textbook. Therefore, $f$ is measurable.\\

    (b) Every montone function has a countable number of discontinuities. Call this set $E$. Since $E$ is countable,
    we have $m(E)$ exists and is equal to 0. Note $f$ is continuous, so its restriction to the set $E^c$ is also continuous.
    Therefore, $f|_{E^c}$ is measurable by the same logic in (a). Furthermore, note for all $a\in\mathbb{R}$:
    \[ \{f > a\} = \{f|_{E^c} > a\} \cup \{f|_{E} > a\} \]
    $\{f|_E > a\}$ is measurable since clearly, $\{f|_E > a\} \subseteq E$ so that $m_*(\{f|_E > a\}) = 0$ and in particular,
    it is Lebesgue measurable. Thus, $\{f > a\}$ is measurable which is enough to imply that $f$ is measurable.
    $\hfill\blacksquare$\\

    \textbf{P3.}\\

    Let $f:\mathbb{R}^d\to\mathbb{R}$.
    \begin{enumerate}
        \item[(a)] Prove that $f$ is measurable if and only if for all $q\in\mathbb{Q}$, the set $\{f < q\}$ is measurable.
        \item[(b)] Prove that $f$ is measurable if and only if for every compact set $C\subseteq\mathbb{R}$, the pre-image
            $f^{-1}(C)$ is measurable.
    \end{enumerate}

    \textbf{Solution.}

    (a) $f$ is measurable implies for all $a\in\mathbb{R}$, $\{f < a\}$ is measurable. In particular, this holds for all
    $a\in\mathbb{Q}$ so the forward direction is proven. Now, assume for all $q\in\mathbb{Q}$, the set $\{f < q\}$ is measurable.
    Assume $b\not\in\mathbb{Q}$. $\overline{\mathbb{Q}}=\mathbb{R}$ implies that $b$ is a limit point of $\mathbb{Q}$ so that
    in particular, there exists a sequence $(b_n)_1^{\infty}\subseteq\mathbb{Q}$ such that $b_n\to b$. Without loss of
    generalization, we may assume $(b_n)$ is strictly increasing ($(b_n)$ is convergent so a monotone subsequence exists
    either way). From this, we have 
    \[ \{f < b_1\} \subseteq \{f < b_2\} \subset \hdots \subseteq \{f < b_n\} \subseteq \hdots \]
    and in particular, $\{f < b_n\} \nearrow \{f < b\}$ and since from assumption, we have each $\{f < b_n\}$ is measurable, by
    Corollary 3.3 on Page 20 of the class textbook, we have that $m(\{f < b\})$ exists so that $\{f < b\}$ is measurable.
    Therefore, for all $b\in\mathbb{R}$, $\{f < b\}$ is measurable from which we conclude that $f$ is measurable.\\

    \newpage

    (b) $f$ is measurable and so from Property 1 on Page 28 of the class textbook, we have that the pre-image of every
    closed set is measurable. Every compact set is closed by Heine-Borel, so in particular, from the measurability of $f$,
    we have that the pre-image of every compact set is measurable. Thus the forward direction is proven. Now, assume
    the pre-image of every compact set is measurable. Define $E_1 = [-1,0]$, $E_2 = [-2,0]$, $\hdots$, $E_n = [-n, 0]$.
    The pre-image of each $E_n$ is measurable (since each $E_n$ is compact by Heine-Borel),
    and so the countable union $\cup_n f^{-1}(E_n)$ is measurable.
    However, $\cup_n f^{-1}(E_n) = \{f\leq 0\}$ so that $\{f\leq 0\}$ is measurable. We can repeat this proof for not just
    0 but any $a\in\mathbb{R}$, and therefore, we get that $\{f\leq a\}$ is measurable which is enough to imply that $f$
    is measurable.$\hfill\blacksquare$\\

    \textbf{P4.}\\

    Let $\phi:[0,1]\to[0,1]$ be the Cantor-Lebesgue function. Consider the function $\psi:[0,1]\to[0,2]$ defined by
    $\psi(x) = \phi(x) + x$.

    \begin{enumerate}
        \item[(a)] Show that $\psi$ is continuous, strictly increasing and onto.
        \item[(b)] Show that $\psi$ maps Borel sets to Borel sets, and conclude that there exists some subset $A\subseteq C$ of
            the Cantor set which is not a Borel set.
    \end{enumerate}

    \textbf{Solution.}

    (a) $\psi$ is continuous because it is the sum of two continuous functions $\phi$ and $x\mapsto x$. Let $x_1 < x_2$ be real numbers.
    We have since $\phi$ is monotone so that $\phi(x_1)\leq \phi(x_2)$ from which, we have $\phi(x_1) + x_1\leq \phi(x_2) + x_1$.
    Since $x_1<x_2$, we obtain $\phi(x_1) + x_1\leq \phi(x_2)+x_1 < \phi(x_2) + x_2$ so that $\psi(x_1)<\psi(x_2)$ as wanted. Thus,
    $\psi$ is strictly increasing. It is onto because of IVT, since $\psi(0) = 0$ and $\psi(1) = 2$ and the domain of $\psi$
    is an interval, and thus, 
    so is the image of its domain (IVT) and the two endpoints of the interval in the domain determine the endpoints
    of the interval in the image. Therefore, $\psi([0,1]) = [0,2]$ as wanted. Note we can apply IVT since we have a continuous
    strictly increasing function.\\

    (b) Define $\Gamma := \{B: \psi(B)$ is a Borel set$\}$. First, we show $\Gamma$ is a $\sigma$-algebra. Since the image of $\Gamma$
    is a Borel $\sigma$-algebra, there must exist some set $E\in\Gamma$ such that $\psi(E) = [0,2]$. Since $\psi$ is onto, we must
    have that $E = [0,1]$, and since $E\in\Gamma$, we have that $[0,1]\in\Gamma$. Now, let $E$ be any arbitrary set in $\Gamma$. We
    show $[0,1]\backslash E\in\Gamma$. Note $\psi(E)$ is a Borel set, so $[0,2]\backslash\psi(E)$ must be one as well. By definition
    of $\Gamma$, there exists some $F$ such that $\psi(F) = [0,2]\backslash\psi(E) = \{x\in[0,1]: \psi(x)\not\in\psi(E)\}$. Thus,
    by definition, $F$ and $E$ are disjoint, for if they were not, the intersecting element $y$ would map to $\psi(E)$ and $\psi(F)$
    contradicting the disjoint property of the two sets. Note also that $E\cup F = [0,1]$ since $F$ is comprised of elements of
    $[0,1]$ that are not mapped to $\psi(E)$. Therefore, $[0,1]\backslash E = F$ and since $F\in\Gamma$, we have that the complement
    of any set in $\Gamma$ is in $\Gamma$. It is left to prove the countable union of sets in $\Gamma$ is in $\Gamma$. Let
    $\cup_n E_n$ be a countable union of sets in $\Gamma$. We have that each $\psi(E_n)$ is a Borel set and since the collection
    of all Borel sets is a $\sigma$-algebra, we have that $\cup_n \psi(E_n)$ is a Borel set. By definition, there exists some
    set $F\in\Gamma$ such that $\psi(F) = \cup_n \psi(E_n)$. Note $\psi(\cup_n E_n) = \cup_n \psi(E_n)$ so we can just choose
    $F = \cup_n E_n$ but then $\cup_n E_n$ is in $\Gamma$ as wanted. Thus, the countable union of sets in $\Gamma$ is in $\Gamma$.
    Therefore, $\Gamma$ is a $\sigma$-algebra.\\

    Now, we show that $\Gamma$ contains all open sets (w.r.t domain of $\psi$) so that we get that the collection of Borel sets
    in the domain of $\psi$ is a subset of $\Gamma$. Since $\Gamma$ maps to a Borel algebra in the image, we will obtain for free
    that all Borel sets are mapped to Borel sets. Thus, take some open set $U$ of $[0,1]$, we can write $U$ as a 
    countable union of open intervals, so let $U = \cup_n U_n$. From the discussion of IVT in (a), we obtain that
    each $\psi(U_n)$ is an interval clearly contained in $[0,2]$. Note $\cup_n \psi(U_n) = \psi(\cup_n \psi_n)$ so that $\psi(U)$
    is open (countable union of open sets is open). Since $\psi(U)$ is open, we have that $\psi(U)\in\psi(\Gamma)$ i.e.
    $\psi(U)$ is a Borel set (all open sets are Borel sets). Therefore, there exists some set $G$ of $[0,1]$ so that $\psi(G)
    = \psi(U)$. Choosing $U = G$, we see that $U\in\Gamma$ as wanted. Therefore $\Gamma$ contains all open sets as wanted.\\

    Thus, $\Gamma$ is a sigma algebra which contains or is equal to the collection of all Borel sets in $[0,1]$, and since
    $\psi(\Gamma)$ is a collection of Borel sets in $[0,2]$, we have that $\psi$ maps Borel sets to Borel sets as wanted. The
    next part of this proof is simple. Noting that $[0,1]\backslash C$ ($C$ being the Cantor set) is percisely the set of line
    segments where each line segment is given by the placement of the middle third being removed and the length of the
    line segment being removed, i.e. the line segment depends on the construction of $C$ so we can can say the line segment
    is given by some mapping $x\mapsto C(x)$ which depends on the construction of $C$. $\phi$ is constant on this line segment
    so $\psi$ on this line segment is given by a mapping $x\mapsto x + a$ for some constant $a$. Therefore, we can characterize
    $\psi$ as a set of translations of line segments on $[0,1]\backslash C$. Since Lebesgue measure is translation invariant, we have
    $m([0,1]\backslash C) = m(\psi([0,1]\backslash C)) = 1$. Since $m(\psi([0,1])) = m([0,2]) = 2$, and
    $m(\psi([0,1]\backslash C \cup C)) = 2$, by applying countable additivity (they are disjoint sets clearly), we have that
    $m(C) = 1$. Therefore, by Vitali, there exists a non-measurable subset of $\psi(C)$ (the measure being non-zero is important since
    all sets of outer measure 0 are measurable). Let this set be $W$. Since it is non-measurable, it must also be non-Borel (recall
    $\mathcal{B}\subset\mathcal{L}$). Since
    $\psi$ only maps Borel sets to Borel sets and $W$ is not Borel, we must have that $A := \psi^{-1}(W)$ is non-Borel. Since
    $W\subseteq\psi(C)$, we must have that $A\subseteq C$. Therefore, we have the existence of a set $A$ which is a subset of the
    Cantor set which is not Borel.

    \newpage

    \textbf{P5.}\\

    Let $f:[0,1]\to\mathbb{R}$ be defined by $f = \chi_{\mathbb{Q}\cap[0,1]}$.
    \begin{enumerate}
        \item[(a)] Show that $f$ is not Riemann integrable.
        \item[(b)] Construct a sequence $f_n:[0,1]\to\mathbb{R}$ of Riemann integrable functions such that $\lim_n f_n(x) = f(x)$
            for all $x\in[0,1]$.
    \end{enumerate}
    \textbf{Solution.}

    (a) We use the hint given in 6c. It says that a bounded function is Riemann integrable if and only if its set of discontinuities
    has measure zero. Since $f$ is given by a characteristic function which is either 0 or 1, $f$ is clearly bounded by $0$ and $1$.
    Additionally, it is clear for all $x\not\in\mathbb{Q}$, $f(x)$ is not continuous since by density of rationals, there exists
    some rational $y$ in any open interval containing $x$ and $y$. From this, we have that $|f(x) - f(y)| = 1$. Thus, for
    $\epsilon = \nicefrac{1}{2}$, for all $\delta>0$, if $0<|x-y|<\delta$, we have that $|f(x)-f(y)|\geq\nicefrac{1}{2}$ implying
    $f$ is not continuous at $x$. We can get a similar result for all $x\in\mathbb{Q}$ by applying the same technique. Thus,
    $f$ is discontinuous everywhere and so the set of discontinuities is $[0,1]$ which clearly has positive measure. Therefore,
    $f$ is not Riemann integrable.\\

    (b) Let $(q_n)_1^{\infty}$ be a sequence of rationals in $[0,1]$. Define $Q_n = \{q_1,\hdots,q_n\}$. Using this, define
    $f_n := \chi_{Q_n}$. Clearly $f_n$ is discontinuous only at a finite number of points so it is still Riemann integrable
    and $\lim_n f_n = f$ follows clearly from the construction of $f_n$ as $(q_n)_1^{\infty} = \mathbb{Q}\cap[0,1]$.$\hfill\blacksquare$\\

    \textbf{P6.}
    \begin{enumerate}
        \item[(a)] Show that for all $n\geq 1$ and all $x\in[0,1]$ we have $0\leq f_n(x)\leq 1$ and $f_{n+1}(x)\leq f_n(x)$, and
            use this to infer that $f(x) = \lim_n f_n(x)$ exists.
        \item[(b)] Show that the function $f$ is discontinuous at every point of $\widetilde{C}$.
        \item[(c)] Conclude that the function $f$ is not Riemann integrable.
    \end{enumerate}

    \textbf{Solution.}

    (a) We are given $0\leq F_n(x)\leq 1$ for all $x\in[0,1]$, for all $n\in\mathbb{N}$, thus $f_n(x)$ is the multiplication of
    \textit{positive} real numbers less than 1. Therefore, it must be the case that $0\leq f_n(x)\leq 1$. Similarly,
    note $f_n(x) = F_1(x)\cdot\hdots\cdot F_n(x) \leq F_1(x)\cdot\hdots\cdot F_n(x)\cdot F_{n+1}(x) = f_{n+1}(x)$
    since $0\leq F_{n+1}(x)\leq 1$. Therefore, $f_n(x)\geq f_{n+1}(x)$. By the bounded monotone convergence theorem,
    $\lim_n f_n(x)$ converges to a real number (in particular, a positive real number less than 1). 
    Define a function $f:[0,1]\to[0,1]$ defined by $f(x) = \lim_n f_n(x)$. \\

    \newpage

    (b) Choose an arbitrary point $x\in\widetilde{C}$ and any open interval $U$ containing $x$. Refer to the Lemma below to obtain
    the fact that there exists some point $y$ in $[0,1]$ distinct from $x$ so that $f(y)=0$ and $y\in U$. Therefore, we obtain that
    if we fix $x\in\widetilde{C}$ choose any $\epsilon < 1$, for all $\delta >0$, if $0<|x-y|<\delta$, we have that
    $|f(x)-f(y)| = 1 > \epsilon$ implying that $f$ is continuous at $x$ as wanted.\\

    (c) We are given that $\widetilde{C}$ has positive measure and so since $f$ is clearly bounded by 0 and 1, it is a bounded function
    whose set of discontinuities is of positive measure implying that $f$ is not Riemann integrable.$\hfill\blacksquare$

    \begin{lemma}
        Let $(a,b)\subset [0,1]$ for real $a,b$. We show there exists some $(c,d)\subseteq (a,b)$ such that $(c,d)$ is removed
        at some iteration of $\widetilde{C}$. Furthermore, we have that for any $x\in\widetilde{C}$ and an open interval $(a,b)$
        containing $x$, there exists a point $y$ in $(a,b)$ such that $f(y) = 0$.
    \end{lemma}
    \begin{proof}
        Suppose not. Then $(a,b)\cap(\widetilde{C})^c = \emptyset$ since if there exists $x\in (a,b)\cap(\widetilde{C})^c$, then
        $x$ is contained in $(a,b)$ which is open and is contained in some removed open interval $I_n$ in the construction of 
        $\widetilde{C}$. Since $x$ lies in the intersection of two open sets, there clearly exists some open interval $I$ such that
        $I \subseteq (a,b)\cap (I_n)$. Define $I := (c,d)$ for some $c<d$ real so we are done as $I$ was removed when $I_n$ was
        removed at some iteration of $\widetilde{C}$. But this cannot be since we assumed that such a $(c,d)$ cannot exist. Therefore,
        $(a,b)\cap(\widetilde{C})^c=\emptyset$. Thus, $(a,b)\subseteq\widetilde{C}$ since $(a,b)\subset [0,1]$. But $\widetilde{C}$
        cannot contain any proper intervals (i.e. a non-empty open interval) since that would imply $\widetilde{C}$ is dense at
        some point. Therefore, $(a,b)\cap(\widetilde{C})^c\neq\emptyset$ so that there exists $x\in(a,b)\cap(\widetilde{C})^c$
        which as we have proved implies that the first part of this lemma is true.\\

        Choose $x\in\widetilde{C}$ and an open interval $(a,b)$ containing $x$. Then there exists some interval $(c,d)$ which
        is a subset of $(a,b)$ which was removed at some iteration of $\widetilde{C}$. Let $y$ be the center of $(c,d)$. By
        construction of $f$, there exists some $F_n$ for some $n$ such that $F_n(y) = 0$ which implies $f(y) = 0$. 
    \end{proof}
    
    
    

\end{document}
