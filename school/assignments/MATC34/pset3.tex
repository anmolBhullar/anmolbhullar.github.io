\documentclass[12pt]{article}
\usepackage{amsmath} 
\usepackage{amsthm} % Theorem Formatting
\usepackage{amssymb}    % Math symbols such as \mathbb
\usepackage{graphicx} % Allows for eps images
\usepackage[dvips,letterpaper,margin=1in,bottom=0.7in]{geometry}
\usepackage{tensor}
 % Sets margins and page size
\usepackage{amsmath}

\renewcommand{\labelenumi}{(\alph{enumi})} % Use letters for enumerate
% \DeclareMathOperator{\Sample}{Sample}
\let\vaccent=\v % rename builtin command \v{} to \vaccent{}
\usepackage{enumerate}
\renewcommand{\v}[1]{\ensuremath{\mathbf{#1}}} % for vectors
\newcommand{\gv}[1]{\ensuremath{\mbox{\boldmath$ #1 $}}}
% for vectors of Greek letters
\newcommand{\uv}[1]{\ensuremath{\mathbf{\hat{#1}}}} % for unit vector
\newcommand{\abs}[1]{\left| #1 \right|} % for absolute value
\newcommand{\avg}[1]{\left< #1 \right>} % for average
\let\underdot=\d % rename builtin command \d{} to \underdot{}
\renewcommand{\d}[2]{\frac{d #1}{d #2}} % for derivatives
\newcommand{\dd}[2]{\frac{d^2 #1}{d #2^2}} % for double derivatives
\newcommand{\pd}[2]{\frac{\partial #1}{\partial #2}}
% for partial derivatives
\newcommand{\pdd}[2]{\frac{\partial^2 #1}{\partial #2^2}}
% for double partial derivatives
\newcommand{\pdc}[3]{\left( \frac{\partial #1}{\partial #2}
 \right)_{#3}} % for thermodynamic partial derivatives
\newcommand{\ket}[1]{\left| #1 \right>} % for Dirac bras
\newcommand{\bra}[1]{\left< #1 \right|} % for Dirac kets
\newcommand{\braket}[2]{\left< #1 \vphantom{#2} \right|
 \left. #2 \vphantom{#1} \right>} % for Dirac brackets
\newcommand{\matrixel}[3]{\left< #1 \vphantom{#2#3} \right|
 #2 \left| #3 \vphantom{#1#2} \right>} % for Dirac matrix elements
\newcommand{\grad}[1]{\gv{\nabla} #1} % for gradient
\let\divsymb=\div % rename builtin command \div to \divsymb
\renewcommand{\div}[1]{\gv{\nabla} \cdot \v{#1}} % for divergence
\newcommand{\curl}[1]{\gv{\nabla} \times \v{#1}} % for curl
\let\baraccent=\= % rename builtin command \= to \baraccent
\renewcommand{\=}[1]{\stackrel{#1}{=}} % for putting numbers above =
\providecommand{\wave}[1]{\v{\tilde{#1}}}
\providecommand{\fr}{\frac}
\providecommand{\RR}{\mathbb{R}}
\providecommand{\NN}{\mathbb{N}}
\providecommand{\CC}{\mathbb{C}}
\providecommand{\ZZ}{\mathbb{Z}}
\providecommand{\seq}{\subseteq}
\providecommand{\e}{\epsilon}

\theoremstyle{definition}
\newtheorem{prop}{Proposition}
\newtheorem{thm}{Theorem}[section]
\newtheorem{axiom}{Axiom}[section]
\newtheorem{p}{Problem}[section]
\usepackage{cancel}
\newtheorem*{lem}{Lemma}
\theoremstyle{definition}
\newtheorem*{dfn}{Definition}
 \newenvironment{s}{%\small%
        \begin{trivlist} \item \textbf{Solution}. }{%
            \hspace*{\fill} $\blacksquare$\end{trivlist}}%
% ***********************************************************
% ********************** END HEADER *************************
% ***********************************************************

\begin{document}

{\noindent\Huge\bf  \\[0.5\baselineskip] {\fontfamily{cmr}\selectfont  %
Problem Set 3}         }\\[2\baselineskip] % Title
{ {\bf \fontfamily{cmr}\selectfont MATC34: Complex Variables}\\ {\textit{\fontfamily{cmr}%
\selectfont October 13, 2017, 1002678140}}}
{\large \textsc{Anmol Bhullar}} % Author name
\\[1.4\baselineskip]

\begin{p}
    Compute the integrals. All curves are considered with counterclockwise orientation. We use the notation
    $C(z_0,r) := \{z\in\CC: \abs{z-z_0} = r\}$, while $[a,b]$ denotes the segment in $\CC$ with endpoints $a,b$.
    \begin{enumerate}
        \item \[ \int_S z^3+2z\:dz\: \text{where}\: S := \{z\in\CC: \abs{z}=1\: \text{and Im}(z)\geq 0\}. \]
        \item \[ \int_P z-4z^2\:dz\: \text{where}\: P := [0,1] \cup [1,1+i] \]
        \item \[ \int_Q z-4z^2\:dz\: \text{where $Q$ is the square of side length 2 centered at the origin.} \]
        \item \[ \int_{C(0,1)} e^{z+z^2}\:dz. \]
        \item \[ \int_{C(2,1)} \log{z}\:dz \]
    \end{enumerate}
\end{p}
\begin{s}
    \begin{enumerate}
        \item Note that $z^3+2z$ is a polynomial. Thus it is holomorphic and its primitive exists and is equal to
            $\frac{z^4}{4} + z^2$. Thus, we can apply FTOC. To find the end points of $S$, notice that
            $S$ is equal to the upper unit semicircle and we are given positive orientation on $S$
            which means $S$ is a curve moving from $z=1$ to $z=-1$. Thus:
            \[ \int_S z^3+2z\:dz = \big{[}\frac{z^4}{4}+z^2\big{]}_1^{-1} = (\frac{1}{4}+1) - (\frac{1}{4}+1) = 0\]
        \item Again we have that $z-4z^2$ is a polynomial implying that it is holomorphic and its primitive exists which is equal
            to $\frac{z^2}{2}-\frac{4}{3}z^3$. Thus, we can apply FTOC. Looking at the definition of $P$, we see that the end points
            are $0$ and $1+i$. Thus:
            \begin{align*}
                \int_P z-4z^2\:dz &= \int_{[0,1+i]} z-4z^2\:dz \\
                &= \big{[}\frac{z^2}{2} - \frac{4}{3}z^3\big{]}_0^{1+i} \\
                &= (\frac{(1+i)^2}{2}-\frac{4}{3}(1+i)^3) \\
                &= \frac{8}{3} - \frac{5i}{3}
            \end{align*}
        \item Note that $z-4z^2$ is a polynomial so it is holomorphic on every $z\in\mathbb{C}$ which implies that it is holomorphic
            on $Q$ and its interior. Thus by corrollary 1.2 (Complex Analysis, pg. 36), we have that:
            \[ \int_Q z-4z^2\:dz = 0 \]
        \item Since $e^z$ and $z+z^2$ are holomorphic everywhere, thus their composition must be as well so $e^z\circ (z+z^2) 
            = e^{z+z^2}$ is holomorphic everywhere. Furthermore, note that $C(0,1)$ is a closed curve so by Cauchy's theorem,
            we have that:
            \[ \int_{C(0,1)} e^{z+z^2}\:dz = 0 \]
        \item Note that $\log{z}$ is holomorphic everywhere except at $z=0$. Since $0\not\in C(2,1)$, then $\log{z}$ is holomorphic
            on the interior of $C(2,1)$. Since $C(2,1)$ is a closed curve, then by Cauchy's theorem, we have that:
            \[ \int_{C(2,1)} \log{(z)}dz = 0 \]
    \end{enumerate}
\end{s}


\begin{p}
    Compute for each $n\in\ZZ$ the integral
    \[ \int_{C(0,1)} z^n\:dz \]
    For which values of $n$ is it zero?
\end{p}
\begin{s}
    Consider the parameterization $z:[0,2\pi]\to\mathbb{C}$ where $z(t) = e^{it} = \cos{t}+i\sin{t}$. Note that this
    parameterizes $C(0,1)$. By applying a change of variables, we get:
    \begin{align*}
        \int_{C(0,1)} z^n\:dz = \int_0^{2\pi} (z(t))^nz'(t)\:dt &= \int_0^{2\pi} i(e^{nit})e^{it}\:dt \\
        &= i\int_0^{2\pi} e^{it(n+1)}\:dt
    \end{align*}
    If $n=-1$, then $e^{it(n+1)}=1 \implies$ (by FTOC) $i\int_0^{2\pi} 1dt = i[1]_0^{2\pi} = 2\pi i$. 
    So, suppose $n\neq -1$.
    Since $e^{it(n+1)}$ is holomorphic everywhere and $-\frac{ie^{i(n+1)t}}{n+1}$ is its primitive. So:
    \begin{align*}
        i\int_0^{2\pi} e^{(n+1)it}dt &= i\big{[}-i\frac{e^{i(n+1)t}}{n+1}\big{]}_0^{2\pi} \\
        &= (-i)(i)\big{[}\frac{e^{(n+1)it}}{n+1}\big{]}_0^{2\pi} \\
        &= (1)\big{(}\frac{e^{(n+1)i(2\pi)}}{n+1} - \frac{e^{(n+1)i(0)}}{n+1}\big{)} \\
        &= \big{(}\frac{e^{(n+1)2\pi i} - e^0}{n+1}\big{)} \\
        &= \frac{e^{(n+1)2\pi i}-1}{n+1}
    \end{align*}
    which is equal to 0 if and only if $e^{(n+1)2\pi i} = 1$. So to find which values of $n\in\mathbb{Z}$,
    is the integral $\int_{C(0,1)} z^n\:dz = 0$,
    we need to find which values of $n$ solves the equation $e^{(n+1)2\pi i} = 1$. Consider:
    \begin{align*}
        e^{(n+1)2\pi i} &= \cos{((n+1)2\pi)} + i\sin{((n+1)2\pi)}
    \end{align*}
    Since $\cos{(z)}$ is periodic with period $2\pi$, then $1 = \cos{(2\pi)} = \cos{((m+1)2\pi)}$ if $m\in\mathbb{Z},m\neq -1$. 
    Similarly, $\sin{(2\pi)} = \sin{((m+1)2\pi)} = 0$. Thus:
    \[ e^{(n+1)2\pi i} = (1) + i(0) = 1 \]
    so that $\int_{C(0,1)} z^n\:dz = 0$ for all $n\in\mathbb{Z}$ but not at $n=-1$ since we established earlier that it is
    equal to $2\pi i$.
\end{s}

\begin{p}
    Consider the function
    \[ g(z) := \sum_{n=1}^{\infty} (-1)^{n+1}\frac{z^n}{n} \]
    \begin{enumerate}
        \item Compute the radius of convergence $R$, showing that it is positive.
        \item Compute the power series for $g'(z)$.
        \item Use this to show that for any $x$ real with $\abs{x} < 1$ one has
            \[ g(x) = \log{(1+x)} \]
    \end{enumerate}
    Thus, $g(z)$ defines the function $\log{(x+1)}$ for complex values.
\end{p}
\begin{s}
    First, we compute the $R$ for $g(z)$. Recall that:
    \[ \frac{1}{R} = \limsup_{n\to\infty} \abs{a_n}^{\frac{1}{n}} \]
    Since $a_n = \frac{(-1)^{n+1}}{n}$, and $\limsup_{n\to\infty} \abs{a_n}^{\frac{1}{n}} = \lim_{n\to\infty} \abs{a_n}^{\frac{1}{n}}$ (if
    the limit on the right hand exists but we claim this does and show it does exist by explicitly computing it), then:
    \begin{align*}
        \frac{1}{R} = \lim_{n\to\infty} \abs{a_n}^{\frac{1}{n}} &= \lim_{n\to\infty} \abs{\frac{(-1)^{n+1}}{n}}^{\frac{1}{n}}\\
        &= \lim_{n\to\infty} \frac{1}{(n)^{\frac{1}{n}}} \\
        &= \frac{1}{\lim_{n\to\infty} (n)^{\frac{1}{n}}} \\
        &= \frac{1}{1} = 1
    \end{align*}
    so $\limsup{n\to\infty}\abs{a_n}^{\frac{1}{n}} = 1 \implies R = 1$ implying the $g(z)$ has positive radius of convergence.
    By theorem 2.6 (Pg. 16, Complex Analysis), $g^{'}(z) = \sum_{n=2}^{\infty} (-1)^{n+1}z^{n-1}$.\\
    Consider for real $x\in\mathbb{R}$:
    \[ \sum_{n=2}^{\infty} (-x)^{n-1} = \sum_{n=1}^{\infty} (-x)^n = \frac{1}{1 - (-x)} \qquad (\text{geometric series}) \]
    Furthermore,
    \[ \frac{1}{1+x} = \sum_{n=2}^{\infty} (-x)^{n-1} = \sum_{n=2}^{\infty} (-1)^{n-1}(x)^{n-1} = \sum_{n=2}^{\infty} 
        (-1)^{n+1}(x)^{n-1} \]
    so that $g'(x) = \sum_{n=2}^{\infty} (-1)^{n+1}x^{n-1} = \frac{1}{1+x}$. Note that $\frac{d}{dx}(\log(1+x)) = \frac{1}{1+x}$
    so we have that: 
    \[\frac{d}{dx}(\log{(1+x)}) = \frac{d}{dx}(g(x)) \]
    Since $x\in\RR$, this simply implies that:
    \[ g(x) = \log{(1+x)} + C\]
    Note that $g(0) = 0$ so if we want $\log{(1+x)}+C$ to agree at $x=0$, we must set $C = 0$. Thus:
    \[ g(x) = \log{(1+x)} \]
\end{s}

\newpage

\begin{p}
    Consider the function
    \[ h(z) = \frac{e^z-1}{z} \]
    \begin{enumerate}
        \item By writing $h(z)$ as a power series and computing its radius of convergence, prove that it is a holomorphic
            function for each $z\in\CC$.
        \item Compute
            \[ \int_{C(1,1)} \frac{e^z}{z}dz.\]
        \item Compute
            \[ \int_{C(0,1)} \frac{e^z-1}{z}dz.\]
        \item Compute
            \[ \int_{C(0,1)} \frac{e^z}{z}dz. \]
    \end{enumerate}
\end{p}
\begin{s}
    \begin{enumerate}
        \item We know that:
            \begin{align*}
                &e^z = 1 + \frac{z}{1} + \frac{z^2}{2} + \frac{z^3}{6} + \hdots \\
                \implies &e^z - 1 = \frac{z}{1} + \frac{z^2}{2} + \frac{z^3}{6} + \hdots \\
                \implies &\frac{e^z-1}{z} =  1 + \frac{z}{2} + \frac{z^2}{6} + \hdots
            \end{align*}
            which implies that $h(z) = 1 + \frac{z}{2} + \frac{z^2}{6} + \hdots + \frac{z^n}{(n+1)!} + \hdots$\\
            so that $h(z) = \sum_{n=1}^{\infty} \frac{z^{n-1}}{n!}$. Now we compute this power series' radius of convergence. Similarly
            to 3(a), we know that:
    We know that
    \[ \frac{1}{R} = \limsup_{n\to\infty} \abs{a_n}^{\frac{1}{n}} = \lim_{n\to\infty} \abs{a_n}^{\frac{1}{n}} \]
    if the limit in the right most hand exists. We claim this limit exists by computing it explicitly. We know $a_n = \frac{1}{n!}$ so:
    \begin{align*}
        \lim_{n\to\infty} \abs{\frac{1}{n!}}^{\frac{1}{n}} &= \lim_{n\to\infty} (\frac{1}{n!})^{\frac{1}{n}}\quad (n!\:\text{is positive for}\:n>0)\\
        &= \lim_{n\to\infty} \frac{1}{(n!)^{\frac{1}{n}}} \\
    \end{align*}
    In lecture, we computed the limit $\lim_{n\to\infty} (n!)^{\frac{1}{n}}$ and showed it is equal to 0. Thus:
    \[ \lim_{n\to\infty} \frac{1}{(n!)^{\frac{1}{n}}} = \frac{1}{0} = \infty \]
    Thus, $R = \infty$ so the power series converges for every $z\in\mathbb{C}$ implying that $h(z)$ is holomorphic for every $z\in\mathbb{C}$.
        \item Note that $0\not\in$ Int$(C(1,1))$ where Int$(U)$ is the interior of a closed curve $U$. Furthermore, $\frac{e^z}{z}$ is
            holomorphic for every $z\in\mathbb{C}$ but undefined at $z=0$. Thus, we have that $\frac{e^z}{z}$ is holomorphic on the interior
            of $C(1,1)$ and since $C(1,1)$ is a closed curve, we can apply Cauchy's theorem to get that:
            \[ \int_{C(1,1)} \frac{e^z}{z}dz = 0 \]
        \item We have already established in 3(a) that $\frac{e^z-1}{z}$ is holomorphic on all of $\mathbb{C}$, thus it is holomorphic
            on the interior of $C(0,1)$. Combining this with that fact that $C(0,1)$ is a closed curve, we get by Cauchy's theorem that:
            \[ \int_{C(0,1)} \frac{e^z-1}{z}dz = 0 \]
        \item Note that $\frac{e^z-1}{z} = \frac{e^z}{z} - \frac{1}{z}$ so that by 3(c), we have $\int_{C(0,1)} (\frac{e^z}{z} - \frac{1}{z})dz = 0$.
            Since we have already computed the integral of $\int_{C(0,1)} \frac{dz}{z}$ in lecture and showed that it was equal to $2\pi i$. Thus, we
            obtain that $\int_{C(0,1)} \frac{e^z}{z}dz = 2\pi i$.
    \end{enumerate}
\end{s}

\begin{p}
    Let
    \[ f(z) := \sum_{n=0}^{\infty} a_nz^n \]
    be a function defined by a power series, with positive radius of convergence. Prove that for any $n$ the $n^{th}$ derivative
    at zero satisfies
    \[ f^{(n)}(0) = a_nn! \]
\end{p}
\begin{s}
    Note that $f(z)$ is a power series. By corollary 2.7, we know that power series are infinitely differentiable in their
    disc of convergence (which exists for $f(z)$ since we are given the disc has positive radius).
    Consider the predicate:
    \[ P(k):\qquad f^{(k)}(z) = \sum_{n=k}^{\infty} (n)(n-1)\hdots(n-k+1)a_nz^{n-k}\]
    We prove this is true using induction. \\
    Consider the base case $k=1$. Then by theorem 2.6, we know:
    \[ f^{(1)}(z) = \sum_{n=1}^{\infty} na_nz^{n-1} \]
    which follows the form given by $P(1)$ so the base case holds. Now assume for some $n>k>1$, $P(k)$ holds. Then we show
    that $P(k+1)$ holds. Consider:
    \begin{align*}
        f^{(k+1)}(z) = \frac{d}{dz}(f^{(k)}(z)) &= \frac{d}{dz}(\sum_{n=k}^{\infty} (n)\hdots(n-k+1)a_nz^{n-k})  \\
        &= \sum_{n=k+1}^{\infty} (n)\hdots(n-k)a_nz^{n-(k+1)}\qquad\text{(By theorem 2.6)} \\
    \end{align*}
    Thus $P(k+1)$ holds. Then $P(n)$ holds which implies that (note, we let $i := n$ and switch to $i$ in the sum to avoid confusion between
    the $n$ in the power series and the $n$ in the predicate):
    \[ f^{(n)}(z) = \sum_{i=n}^{\infty} (i)\hdots(i-n+1)a_iz^{i-n} = a_nn! + \sum_{i=n+1}^{\infty} (i)\hdots(i-n)a_iz^{i-n} \]
    which implies that:
    \[ f^{(n)}(0) = a_nn! + \sum_{i=n+1}^{\infty} (i)\hdots(i-n+1)a_i(0)^{i-n} = a_nn! + (0) = a_nn!\]
    as wanted.
\end{s}


\end{document}
