\documentclass[12pt]{article}
\usepackage{amsmath} 
\usepackage{amsthm} % Theorem Formatting
\usepackage{amssymb}    % Math symbols such as \mathbb
\usepackage{graphicx} % Allows for eps images
\usepackage[dvips,letterpaper,margin=1in,bottom=0.7in]{geometry}
\usepackage{tensor}
 % Sets margins and page size
\usepackage{amsmath}

\renewcommand{\labelenumi}{(\alph{enumi})} % Use letters for enumerate
% \DeclareMathOperator{\Sample}{Sample}
\let\vaccent=\v % rename builtin command \v{} to \vaccent{}
\usepackage{enumerate}
\renewcommand{\v}[1]{\ensuremath{\mathbf{#1}}} % for vectors
\newcommand{\gv}[1]{\ensuremath{\mbox{\boldmath$ #1 $}}}
% for vectors of Greek letters
\newcommand{\uv}[1]{\ensuremath{\mathbf{\hat{#1}}}} % for unit vector
\newcommand{\abs}[1]{\left| #1 \right|} % for absolute value
\newcommand{\avg}[1]{\left< #1 \right>} % for average
\let\underdot=\d % rename builtin command \d{} to \underdot{}
\renewcommand{\d}[2]{\frac{d #1}{d #2}} % for derivatives
\newcommand{\dd}[2]{\frac{d^2 #1}{d #2^2}} % for double derivatives
\newcommand{\pd}[2]{\frac{\partial #1}{\partial #2}}
% for partial derivatives
\newcommand{\pdd}[2]{\frac{\partial^2 #1}{\partial #2^2}}
% for double partial derivatives
\newcommand{\pdc}[3]{\left( \frac{\partial #1}{\partial #2}
 \right)_{#3}} % for thermodynamic partial derivatives
\newcommand{\ket}[1]{\left| #1 \right>} % for Dirac bras
\newcommand{\bra}[1]{\left< #1 \right|} % for Dirac kets
\newcommand{\braket}[2]{\left< #1 \vphantom{#2} \right|
 \left. #2 \vphantom{#1} \right>} % for Dirac brackets
\newcommand{\matrixel}[3]{\left< #1 \vphantom{#2#3} \right|
 #2 \left| #3 \vphantom{#1#2} \right>} % for Dirac matrix elements
\newcommand{\grad}[1]{\gv{\nabla} #1} % for gradient
\let\divsymb=\div % rename builtin command \div to \divsymb
\renewcommand{\div}[1]{\gv{\nabla} \cdot \v{#1}} % for divergence
\newcommand{\curl}[1]{\gv{\nabla} \times \v{#1}} % for curl
\let\baraccent=\= % rename builtin command \= to \baraccent
\renewcommand{\=}[1]{\stackrel{#1}{=}} % for putting numbers above =
\providecommand{\wave}[1]{\v{\tilde{#1}}}
\providecommand{\fr}{\frac}
\providecommand{\RR}{\mathbb{R}}
\providecommand{\NN}{\mathbb{N}}
\providecommand{\seq}{\subseteq}
\providecommand{\e}{\epsilon}

\newtheorem{prop}{Proposition}
\newtheorem{thm}{Theorem}[section]
\newtheorem{axiom}{Axiom}[section]
\newtheorem{p}{Problem}[section]
\usepackage{cancel}
\newtheorem*{lem}{Lemma}
\theoremstyle{definition}
\newtheorem*{dfn}{Definition}
 \newenvironment{s}{%\small%
        \begin{trivlist} \item \textbf{Solution}. }{%
            \hspace*{\fill} $\blacksquare$\end{trivlist}}%
% ***********************************************************
% ********************** END HEADER *************************
% ***********************************************************

\begin{document}

{\noindent\Huge\bf  \\[0.5\baselineskip] {\fontfamily{cmr}\selectfont  %
Problem Set 2}         }\\[2\baselineskip] % Title
{ {\bf \fontfamily{cmr}\selectfont MATC63: Differential Geometry}\\ {\textit{\fontfamily{cmr}%
\selectfont October 5, 2017}}}
{\large \textsc{Anmol Bhullar | 1002678140}} % Author name
\\[1.4\baselineskip]

\begin{p}
    Compute $\mathbf{I}$ (i.e. $\mathbf{E},\mathbf{F}$ and $\mathbf{G}$) for the following parameterizated curve:
    \[ \text{the torus:}\:\:\mathbf{x}(u,v) = ((a+b\cos{u})\cos{v},(a+b\cos{u})\sin{v},b\sin{u})\quad (0 < b < a) \]
\end{p}
\begin{s}
    First, we calculate $\mathbf{E}$:
    \begin{align*}
        \mathbf{E} &= I_p(\mathbf{x}_u,\mathbf{x}_u) = \mathbf{x}_u\cdot\mathbf{x}_u \\
        &= (-b\sin{u}\cos{v},-b\sin{u}\sin{v},b\cos{u})\cdot(-b\sin{u}\cos{v},-b\sin{u}\sin{v},b\cos{u}) \\
        &= (-b)\sin{u}\cos{v}(-b)\sin{u}\cos{v} + (-b)\sin{u}\sin{v}(-b)\sin{u}\sin{v} + (b)\cos{u}(b)\cos{u} \\
        &= b^2\sin^2{u}\cos^2{v} + b^2\sin^2{u}\sin^2{v} + b^2\cos^2{v} \\
        &= b^2\sin^2{u}(\cos^2{v}+\sin^2{v}) + b^2\cos^2{v} \\
        &= b^2\sin^2{u} + b^2\cos^2{v} \\
        &= b^2
    \end{align*}
    Then, we calculate $\mathbf{G}$:
    \begin{align*}
        \mathbf{G} &= I_p(\mathbf{x}_v,\mathbf{x}_v) = \mathbf{x}_v\cdot\mathbf{x}_v \\
        &= (-(a+b\cos{u})\sin{v},(a+b\cos{u})\cos{v},0)\cdot(-(a+b\cos{u})\sin{v},(a+b\cos{u})\cos{v},0) \\
        &= (a+b\cos{u})^2\sin^2{v} + (a+b\cos{u})^2\cos^2{v} \\
        &= (a+b\cos{u})^2(\sin^2{v} + \cos^2{v}) \\
        &= (a+b\cos{u})^2
    \end{align*}
    Finally, we calculate $\mathbf{F}$:
    \begin{align*}
        \mathbf{F} &= I_p(\mathbf{x}_u,\mathbf{x}_v) = \mathbf{x}_u\cdot\mathbf{x}_v \\
        &= (-b\sin{u}\cos{v},-b\sin{u}\sin{v},b\cos{u})\cdot(-(a+b\cos{u})\sin{v},(a+b\cos{u})\cos{v},0) \\
        &= (b)(a+b\cos{u})\sin{u}\cos{v}\sin{v} - (b)(a+b\cos{u})\sin{u}\cos{v}\sin{v} + 0 \\
        &= 0
    \end{align*}
    So, we get that:
    \begin{align*}
        I_p = \begin{pmatrix} b^2 & 0 \\ 0 & (a+b\cos{u})^2 \end{pmatrix}
    \end{align*}
\end{s}
\newpage

\begin{p}
    Find the surface area of the following parameterized surface:
    \[ \text{a zone of the sphere:}\: \mathbf{x}(u,v) = a(\sin{u}\cos{v},\sin{u}\sin{v},\cos{u})\quad0\leq u_0\leq u\leq u_1\leq\pi,\: 0\leq v\leq 2\pi \]
\end{p}
\begin{s}
    We can calculate the surface area via the integral:
    \[ \int_U \sqrt{EG-F^2}dudv \]
    Thus, first we find $\mathbf{E}$:
    \begin{align*}
        \mathbf{E} &= I_p(\mathbf{x}_u,\mathbf{x}_u) \\
        &= a^2(\cos{u}\cos{v},\cos{u}\sin{v},-\sin{u})\cdot(\cos{u}\cos{v},\cos{u}\cos{v},-\sin{u}) \\
        &= a^2(\cos^2{u}\cos^2{v} + \cos^2{u}\sin^2{v} + \sin^2{u}) \\
        &= a^2(\cos^2{u}(\cos^2{v}+\sin^2{v}) + \sin^2{u}) \\
        &= a^2(\cos^2{u} + \sin^2{u}) \\
        &= a^2
    \end{align*}
    Now, we find $\mathbf{G}$:
    \begin{align*}
        \mathbf{G} &= I_p(\mathbf{x}_v, \mathbf{x}_v) \\
        &= a^2(\sin^2{u}\sin^2{v} + \cos^2{v}\sin^2{u}) \\
        &= a^2(\sin^2{u}(\sin^2{v}+\cos^2{v})) \\
        &= a^2\sin^2{u}
    \end{align*}
    Finally, we have $\mathbf{F}$:
    \begin{align*}
        \mathbf{F} &= I_p(\mathbf{x}_u,\mathbf{x}_v) \\
        &= a^2(\cos{v}\cos{u}, \cos{u}\sin{v},-\sin{u})\cdot(-\sin{u}\sin{v},\cos{v}\sin{u},0) \\
        &= a^2(-\sin{u}\sin{v}\cos{v}\cos{u} + \cos{u}\sin{v}\cos{v}\sin{u}) \\
        &= 0
    \end{align*}
    Thus: $\sqrt{EG-F^2} = \sqrt{a^2\cdot a^2\sin^2(u) - 0^2} = \sqrt{a^4\sin^2{u}} = a^2\sin{u}$ which implies that:
    \[ \int_U \sqrt{EG-F^2}dudv = \int_0^{2\pi}\int_{u_0}^{u_1} a^2\sin{u}dudv = a^2\int_0^{2\pi} (-\cos{u_1}+\cos{u_0})dv = 2\pi a^2(\cos{u_0}-\cos{u_1}) \]
\end{s}

\newpage
\begin{p}
    Check that the parameterization $\mathbf{x}(u,v)$ is conformal if and only if $E = G$ and $F = 0$.
\end{p}
\begin{s}
    Recall the definition of a conformal map (Docormo, 226): \\

    A parameterization $\mathbf{x}: S \to \overline{S}$ is called a \textbf{conformal map} if for all $p\in S$, and all $v_1,v_2\in T_p(S)$ we have
    \[ \langle d_{\phi_p}(v_1),d_{\phi_p}(v_2)\rangle = \lambda^2(p)\langle v_1,v_2\rangle \]
    where $\lambda^2$ is a a nowhere-zero differentiable function on $S$.\\

    First, assume that $\mathbf{x}$ is conformal. We prove that $E=G$ and $F = 0$.

    Let $e_1, e_2$ be the two standard vectors in $\RR^2$. Note that these two vectors are orthogonal to each other and $d_p(e_1) = \mathbf{x}_u$ and 
    $d_p(e_2) = \mathbf{x}_v$. Note that:
    \[ \mathbf{x}_u \cdot \mathbf{x}_v = \lambda e_1\cdot e_2\]
    by our assumption that $\mathbf{x}$ is conformal. Since $e_1, e_2$ are orthogonal, then:
    \[ \mathbf{x}_u \cdot \mathbf{x}_v = 0 \implies \mathbf{F} = 0 \]
    Furthermore, we know that $d_p(e_1+e_2) = \mathbf{x}_u + \mathbf{x}_v$ and $d_p(e_1-e_2) = \mathbf{x}_u - \mathbf{x}_v$ and
    we know that by our assumption of $\mathbf{x}$ being conformal that: $(\mathbf{x}_u+\mathbf{x}_v)\cdot(\mathbf{x}_u-\mathbf{x}_v) = \lambda(e_1+e_2)\cdot(e_1-e_2)$.
    But, $e_1+e_2$ and $e_1-e_2$ are orthogonal to each other so:
    \[ (\mathbf{x}_u + \mathbf{x}_v)\cdot(\mathbf{x}_u - \mathbf{x}_v) = 0 \implies \mathbf{x}_u\mathbf{x}_u - \mathbf{x}_v\mathbf{x}_v = 0 
        \implies \mathbf{E} = \mathbf{G} \]
    as wanted. \\

    Now we assume $\mathbf{E} = \mathbf{G}, \mathbf{F} = 0$, and try to prove that $\mathbf{x}$ is conformal.
    Let $(v_1,v_2),\:(w_1,w_2)\in uv\:\text{plane}$. Then, note:
    \begin{align*}
        \overline{U} &= v_1\mathbf{x}_u + v_2\mathbf{x}_v \in T_p(S) \\
        \overline{V} &= w_1\mathbf{x}_u + w_2\mathbf{x}_v \in T_p(S)
    \end{align*}
    Now consider $\langle \overline{U}, \overline{V}\rangle$:
    \begin{align*}
        \langle \overline{U}, \overline{V} \rangle &= (v_1\mathbf{x}_u + v_2\mathbf{x}_v)\cdot(w_1\mathbf{x}_u+w_2\mathbf{x}_v) \\
        &= (v_1\mathbf{x}_u, w_1\mathbf{x}_u + w_2\mathbf{x}_v) + (v_2\mathbf{x}_v, w_1\mathbf{x}_u + w_2\mathbf{x}_v) \\
        &= (v_1\mathbf{x}_u, w_1\mathbf{x}_u) + (v_1\mathbf{x}_u, w_2\mathbf{x}_v) + (v_2\mathbf{x}_v, w_1\mathbf{x}_u) + (v_2\mathbf{x}_v, w_2\mathbf{x}_v)
    \end{align*}
    Using the bi-linearity of this inner product we can say this is equal to:
    \begin{align*}
        v_1w_1(\mathbf{x}_u,\mathbf{x}_u) + v_1w_2(\mathbf{x}_u,\mathbf{x}_v) + v_2w_1(\mathbf{x}_v,\mathbf{x}_u) + v_2w_2(\mathbf{x}_v,\mathbf{x}_v)
    \end{align*}
    The middle two terms are equal to 0 since by our assumption $\mathbf{F} = \mathbf{x}_v\cdot\mathbf{x}_u = \mathbf{x}_u\cdot\mathbf{x}_v = 0$.
    Furthermore, we can simplify the left over terms using $\mathbf{E} = \mathbf{x}_u\mathbf{x}_u = \mathbf{x}_v\mathbf{x}_v = \mathbf{G}$
    \[ \langle \overline{U}, \overline{V} \rangle = (v_1w_1 + v_2w_2)(\mathbf{x}_u,\mathbf{x}_u) = \mathbf{E}(v\cdot w) \]
    which is enough to imply that $\mathbf{x}$ is a conformal mapping.
\end{s}

\begin{p}
    Derive a formula for the stereographic projection of the 2-sphere.
\end{p}
\begin{s}
    Let $(p_1,p_2,p_3)$ be a point in $S^2-\{N\}$. Then the line from $N$ going through $(p_1,p_2,p_3)$ is given by:
    \[ \phi: t \mapsto (0,0,1) + t(p_1,p_2,p_3-1) \]
    Thus, $\phi(t) = 0$ when $t(p_3 - 1) + 1 = 0 \implies t = \frac{-1}{p_3-1}$. Plugging this back into the parameterization of the line, we obtain
    another function $\mathbf{x}^{-1}: \RR^3 \to \RR^2$ given by:
    \[ \mathbf{x}^{-1}(p_1,p_2,p_3) = (\frac{-p_1}{p_3-1}, \frac{-p_2}{p_3-1}) \]
    Now, for the other direction, to find a function $\mathbf{x}: \RR^2 \to \RR^3$, consider a line $\gamma(t) = (0,0,1) + t(x,y,-1)$. This is a line
    between any point sitting on the $z=0$ plane and the north pole of the sphere, thus to find when this line intersects the sphere, we can just find when
    \[ \sqrt{(tu)^2 + (tv)^2 + (-t+1)^2} = 1 \]
    which holds true whenever $t = 0$ or $t = \frac{2}{1+x^2+y^2}$. Thus:
    \begin{align*}
        \mathbf{x}(u,v) &= (0,0,1) + t(u,v,-1) \\
        &= \frac{2}{1+u^2+v^2}(u,v,-1) + (0,0,1) \\
        &= (\frac{2u}{1+u^2+v^2}, \frac{2v}{1+u^2+v^2}, \frac{-2}{1+u^2+v^2} + 1) \\
        &= (\frac{2u}{1+u^2+v^2}, \frac{2v}{1+u^2+v^2}, \frac{-1+u^2+v^2}{1+u^2+v^2})
    \end{align*}
    as wanted.
\end{s}

\end{document}
