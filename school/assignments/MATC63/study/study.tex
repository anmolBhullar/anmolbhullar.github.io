\documentclass{article}
\usepackage[utf8]{inputenc}
\usepackage[english]{babel}
\usepackage{amsfonts}
\usepackage{amsthm}
\usepackage{amsmath}

\newtheorem{theorem}{Theorem}
\newtheorem{es}{Examples}

\begin{document}
    \section*{Curves 1.1}
    We say $f$ is \textit{smooth} if $f$ is $C^k$ for every integer $k$.\\\\
    A \textit{parameterized} curve is a $C^n$ (or smooth) map $\alpha: I \to \mathbb{R}^n$ for some interval $I = (a,b)$
        or $[a,b]$ in $\mathbb{R}$.\\\\
    We say $\alpha$ is \textit{regular} parameterized curve if it is a parameterized curve and if $\alpha^{'}\neq 0$ 
    for all $t\in I$.\\\\
    The velocity vector $\alpha'(t)$ is tangent to the curve at $\alpha(t)$ and its length $||\alpha'(t)||$, is the 
    speed of the particle.\\\\
    The \textit{arclength} of a curve $\alpha$ from $a$ to $t$ is given by $s(t) = \int_a^t ||\alpha'(t)||du$.\\\\
    The curve $\alpha$ is \textit{arclength parametrized} if $s(t) = t$ for all $t$. Equivalently, if $||\alpha'(t)|| = 1$.\\\\
    \textbf{Examples of curves.}\\\\
    Given points $\vec{P}$ and $\vec{Q}$, the parameterization of a line going from $\vec{P}$ to $\vec{Q}$ is given 
    by: $\alpha(t) = P + t(Q-P)$ where $t\in\mathbb{R}$ and $0\leq t\leq 1$.\\\\
    The parameterization of the ellipse is given by $\alpha(t) = (a\cos(t),b\sin(t))$ where $t\in\mathbb{R}$ and 
    $0\leq t\leq 2\pi$.\\\\
    The \textit{cuspidal cubic} parameterization is given by $\alpha(t) = (t^2,t^3)$ and the \textit{nodal cubic} is
    given by $\beta(t) = (t^2-1, t^3-t)$.\\\\
    The \textit{twisted cubic} (in $\mathbb{R}^3$) is given by $\alpha(t) = (t,t^2,t^3)$ where $t\in\mathbb{R}^3$. It's
    projections in the $xy-,xz-$ and $yz-$ coordinate planes are given by $y=x^2$, $z=x^3$ and $z^2=y^3$ (cuspidal cubic).\\\\
    The \textit{cycloid} is given by $\alpha(t) = a(t-\sin{(t)}, 1-\cos{(t)})$ where $t\in\mathbb{R}$.\\\\
    The \textit{helix} (in $\mathbb{R}^2$) is given by $\alpha(t) = (a\cos{(t)},a\sin{(t)},bt)$.\\\\
    The \textit{catenary} is given by the graph of $f(x) = C\cosh{(\frac{x}{C})}$, for any constant $C>0$ or by the
    parametrization $\alpha(t) = (t,\cosh(t))$.\\\\
    The \textit{tractrix} is given by $\beta(t) = (t-tanh(t),sech(t))$ for all $t\geq 0$.\\\\
    \newpage
    \textbf{Important properties of} cosh and sinh.
    \[ \cosh(t) = \frac{e^t + e^{-t}}{2},\qquad \sinh(t) = \frac{e^t - e^{-t}}{2},\qquad tanh(t) = \frac{\sinh(t)}{\cosh(t)},
        \qquad sech(t) = \frac{1}{\cosh(t)}\]
    \begin{align*}
        \cosh^2(t) - \sinh^2(t) = 1,\qquad tanh^2(t) + sech^2(t) = 1 \\
        \sinh'(t)  = \cosh(t),\qquad \cosh'(t) = \sinh(t),\qquad tanh'(t) = sech^2(t)\\
        sech'(t) = -tanh(t)\cdot sech(t).
    \end{align*}

    \section*{Curves Continued}
    Let $f:(a,b)\to\mathbb{R}^3$ be differentiable. Then $||f(t)||$ is constant for all $t$ if and only if $f(t)\cdot f'(t)=0$.\\\\
    Assume that the curve $\alpha$ is arclength parameterized. Then $\mathbf{T}(t) = \alpha'(t)$ is the 
    \textit{unit tangent vector} to the curve. Note that $\mathbf{T}$ has constant length and $\mathbf{T}'$ 
    is orthogonal to $\mathbf{T}$.\\\\
    Assume $\mathbf{T}'\neq 0$, then we define the \textit{principal normal vector} $\mathbf{N}(t) = 
    \frac{\mathbf{T}'}{||\mathbf{T}'||}$ and the \textit{curvature} is defined to be $\mathcal{K}(t) = ||\mathbf{T}'(t)||$. Note,
    that if $\alpha$ is a regular parametrized curve, then $\mathcal{K} = \frac{||\alpha'\times \alpha''||}{||\alpha'||^3}$\\\\
    Assume $\mathcal{K}\neq 0$. We define \textit{binormal vector} to be $\mathbf{B}(t) = \mathbf{T}(t)\times \mathbf{N}(t)$.\\\\
    We define the \textit{torsion} to be $\tau(t) = \mathbf{N}'\cdot \mathbf{B}$.\\\\
    The \textit{Frenet Apparatus} is defined to be the collection $\mathbf{T},\mathbf{N},\mathbf{B},\mathcal{K}$ and $\tau$.\\
    The \textit{Frenet Frame} is given by the formulas:
    \begin{align*}
        \mathbf{T}'(s) &= \mathcal{K}(s)\mathbf{N}(s) \\
        \mathbf{N}'(s) &= -\mathcal{K}(s)\mathbf{T}(s) + \tau(s)\mathbf{B}(s)\\
        \mathbf{B}'(s) &= -\tau(s)\mathbf{N}(s)
    \end{align*}
    The \textit{involute} of a curve $\alpha(t)$ is given by $\beta(t) = \alpha(t) + (c-t)\mathbf{T}(t)$ for some constant $c$.
    The \textit{evolute} of a curve $\alpha(t)$ is given by $\beta(t) = \alpha(t) + 
    \frac{1}{\mathcal{K}}\mathbf{N}(t) + \frac{1}{\mathcal{K}}\cot(\int\tau dt)\mathbf{B}(t)$.\\\\
    Assume $\mathcal{K}(t)\neq 0$ for a curve $\alpha$. Then, its \textit{radius of curvature} is defined as $\rho(s) = 
    \frac{1}{\mathcal{K}(t)}$. Furthermore, we define the \textit{center of curvature} $c$ to be given by the formula:
    $c = \alpha(t) + \rho(t)\mathbf{N}(t)$ where $\mathbf{N}$ is the normal to the curve.\\\\
    If a simple closed curve $C$ has length $L$ and encloses area $A$, then the \textit{isoperimetric inequality} states that
    $L^2 \geq 4\pi A$ where the equality holds if and only if $C$ is a circle.

    \section*{Surfaces}
    Let $U$ be an open set in $\mathbb{R}^2$. A \textit{regular parameterization} of a subset $M\subset\mathbb{R}^3$ is a $C^3$
    injective function $x: U\to M\subset\mathbb{R}^3$ so that $x_u\times x_v\neq 0$. A connected subset
    $M\subset\mathbb{R}^3$ is called a \textit{surface} if each point has a neighbourhood that is regularly parameterized.\\\\
    \textbf{Examples of Surfaces}\\\\
    The graph of a function $f:U\to \mathbb{R}$, $z=f(x,y)$, is parameterized by $x(u,v) = (u,v,f(u,v))$.\\\\
    The \textit{helicoid} is given by $x(u,v) = (u\cos(v),u\sin(v),bv)$ for $u>0$ and $v\in\mathbb{R}$. Note that the
    $u-$curves are called \textit{rays} and the $v-$curves are called \textit{helices}.\\\\
    The \textit{torus} (i.e. the surface of a doughnut) is given by the regular parametrization $x(u,v) = 
    ((a+b\cos(u))\cos(v),(a+b\cos(u))\sin(v),b\sin(u))$ for $0\leq u$ and $v<2\pi$.\\\\
    The standard parametrization of the unit sphere $\sigma$ is given by spherical coordinates $(\phi,\theta)\leftrightarrow
    (u,v)$: $x(u,v) = (\sin(u)\cos(v),\sin(u)\sin(v),\cos(u))$ for $0<u<\pi$ and $0\leq v<2\pi$.\\\\
    Let $I\subset\mathbb{R}$, and let $\alpha(u) = (0,f(u),g(u)), u\in I$, be a regular parametrized plane curve (injective) with
    $f > 0$. Then the \textit{surface of revolution} obtained by rotating $\alpha$ about the $z-$axis is parametrized by
    $x(u,v) = (f(u)\cos(v),f(u)\sin(v),g(u))$ where $u\in I$ and $0\leq v < 2\pi$. Note, the $u-$curves are called
    \textit{profile curves} or \textit{meridians}. The $v-$curves are circles, called \textit{parallels}.\\\\
    Let $I\subset\mathbb{R}$ be an interval, let $\alpha: I\to\mathbb{R}^3$ be a regular parametrized curve, and let
    $\beta: I \to \mathbb{R}^3$ be an arbitrary smooth function with $\beta(u)\neq 0$ for all $u\in I$. We define a parametrized
    surface by $x(u,v) = \alpha(u) + v\beta(u)$ for $u\in I$ and $v\in\mathbb{R}$. This is called a \textit{ruled surface}
    with \textit{rulings} $\beta(u)$ and \textit{directrix} $\alpha$. The cylinder, helicoid and cone are all examples of a
    ruled surface.
    \newpage

    \section*{First Fundamental Form}
    Let $M$ be a regular parametrized surface, and let $P\in M$. Then choose a regular parametrization $x:U\to M\subset\mathbb{R}^3$
    with $P = x(u_0,v_0)$. We define the \textit{tangent plane} of $M$ at $P$ to be the subspace $T_PM$ spanned by $x_u$ and
    $x_v$ evaluated at $(u_0,v_0)$.\\\\
    The \textit{unit normal} $\mathbf{n}$ to a parametrized surface is given by $\mathbf{n} = \frac{x_u\times x_v}{||x_u\times x_v||}$.
    Note $x_u \times x_v$ is a non zero vector orthogonal to the plane spanned by $x_u$ and $x_v$.\\\\
    The \textit{first fundamental form} is $I_P(U,V) = U\cdot V$ for $U,V\in T_PM$. To find $I_P$, we define the equations:
    \begin{align*}
        E &= I_P(x_u,x_u) = x_u\cdot x_u \\
        F &= I_P(x_u,x_v) = x_u\cdot x_v = x_v\cdot x_u = I_P(x_v,x_u) \\
        G &= I_P(x_v,x_u) = x_v\cdot x_v
    \end{align*}
    so that $I_P$ is given by the symmetric matrix:
    \begin{align*} 
        I_P = \begin{pmatrix} E & F \\ F & G \end{pmatrix}
    \end{align*}
    The \textit{surface area} of the parametrized surface $x: U\to M$ is given by the formula:
    $ \int_U ||x_u\times x_v||dudv = \int_U \sqrt{EG-F^2}dudv $ \\\\
    A map is an \textit{isometry} if it preserves distance and angles. A map is \textit{conformal} if it preserves angles.
    Note that a map $x(u,v)$ is conformal if and only if $E=G$ and $F=0$ and $x(u,v)$ is isometric if and only if $E=G=1$ and $F=0$.
    \newpage

    \section*{Second Fundamental Form}

    
\end{document}
