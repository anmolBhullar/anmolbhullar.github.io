\documentclass{article}
\usepackage[utf8]{inputenc}
\usepackage[english]{babel}
\usepackage{amsfonts}
\usepackage{amsthm}
\usepackage{amsmath}
\usepackage{amssymb}

\newtheorem{theorem}{Theorem}
\newtheorem{es}{Examples}

\title{MATC63 A4}
\date{November 30, 2017}
\author{Anmol Bhullar | 1002678140}

\begin{document}
    \maketitle 

    $\S$3.1 \textbf{Question 1a}.\\
    Compute the holonomy around the parallel $u=u_0$ (and indicate which direction the rotation occurs from the viewpoint of an observer away 
    from the surface down the x-axis) on:
    \[ \text{a) The Torus}\: \mathbf{x}(u,v) = ((a+b\cos{u})\cos{v},(a+b\cos{u})\sin{v},b\sin{u}) \]
    \textbf{Answer}.\\
    Let the framing be given by $e_1 = \frac{x_u}{\sqrt{E}}$ and $e_2 = \frac{x_v}{\sqrt{G}}$. We explicitly compute this.
    \begin{align*}
        E &= x_u\cdot x_u \\
        &= ((-b\sin{u})\cos{v},(-b\sin{u})\sin{v},b\cos{u}) \cdot ((-b\sin{u})\cos{v},(-b\sin{u})\sin{v},b\cos{u}) \\
        &= b^2\sin^2(u)\cos^2(v) + b^2\sin^2(u)\sin^2(v) + b^2\cos^2(u) \\
        &= b^2\sin^2(u)[\cos^2(v) + \sin^2(v)] + b^2\cos^2(u) \\
        &= b^2 \implies \sqrt{E} = b
    \end{align*}
    \begin{align*}
        G &= x_v\cdot x_v \\
        &= (a+b\cos(u))^2((-\sin(v)),\cos(v),0)\cdot(-\sin(v),\cos(v),0) \\
        &= (a+b\cos(u))^2((-\sin(v))^2 + (\cos(v))^2 + 0^2) \\
        &= (a+b\cos(u))^2 \implies \sqrt{G} = a+b\cos(u)
    \end{align*}
    so that
    \begin{align*}
        e_1 &= \frac{(-b\sin(u)\cos(v),-b\sin(u)\sin(v),b\cos(u))}{b} \\
        &= (-\sin(u)\cos(v),\sin(u)\sin(v),\cos(u))
    \end{align*}
    \begin{align*}
        e_2 &= \frac{((a+b\cos(u))(-\sin(v),(a+b\cos(u))(\cos(v)),0)}{a+b\cos(u)} \\
        &= (-\sin(v),\cos(v),0)
    \end{align*}
    Now, we check whether this is an orthogonal projection:
    \begin{align*}
        F &= x_u\cdot x_v \\
        &= ((-b\sin{u})\cos{v},(-b\sin{u})\sin{v},b\cos{u})\cdot (a+b\cos{u})(-\sin{v},\cos{v},0)\\
        &= [a+b\cos{u}](b\sin{u}\cos{v}\sin{v} - b\sin{u}\sin{v}\cos{v})\\
        &= [a+b\cos{u}](0)\\
        &= 0
    \end{align*}
    so that $\mathbf{x}$ is an orthogonal projection. By Proposition 1.1 in $\S$3.1, we then obtain that:
    \[ \phi_{12} = \frac{1}{2\sqrt{EG}}(-E_vu' + G_uv') \]
    Noting that $E = b^2 \implies E_u = 0$ and $G = (a+b\cos{u})^2 \implies G_u = -2b(a+b\cos{u})\sin{u}$, we obtain that:
    \[ \phi_{12} = \frac{1}{2b(a+b\cos{u})}[-2b(a+b\cos{u})(\sin{u})] = -\sin{u} \]
    Using Proposition 1.12 in $\S$3.1 and setting $u=u_0=$const., we obtain that the holonomy is equal to:
    \begin{align*}
        -\int_0^{2\pi} \phi_{12}(t)dt = -\int_0^{2\pi} -\sin{u_0}dt = 2\pi\sin{u_0}
    \end{align*}
    which is what we wanted to compute.\hfill$\blacksquare$\\

    $\S$3.1 \textbf{Question 3}.\\
    Calculate the Gaussian curvature of a torus and verify Corollary 1.11.\\

    \textbf{Answer}.\\
    We know from $\S$2.3 that in an orthgonal projection (which the torus is, as we have established in question 1), we have that
    \[ K = -\frac{1}{2\sqrt{EG}}\Big{(}\big{(}\frac{E_v}{\sqrt{EG}}\big{)}_v + \big{(}\frac{G_u}{\sqrt{EG}}\big{)}_u\Big{)} \]
    Note, we computed $\frac{1}{2\sqrt{EG}}$, $G_u$ and $E_v$ in the last question so:
    \begin{align*}
        K &= -\frac{1}{2b(a+b\cos{u})}\Big{(}(0) + (\frac{-2b(a+b\cos{u})\sin{u}}{b(a+b\cos{u})})_u\Big{)} \\
        &= -\frac{1}{2b(a+b\cos{u})}\Big{(}(2\sin{u})_u\Big{)} \\
        &= -\frac{1}{2b(a+b\cos{u})}(-2\cos{u}) \\
        &= \frac{\cos{u}}{b(a+b\cos{u})}
    \end{align*}
    Now, we can compute $\int_M KdA$. First, since we know that $dA = \sqrt{EG-F^2}dudv$, we can just compute the integral:
    \[ \int_0^{2\pi}\int_0^{2\pi} K\sqrt{EG-F^2}dudv \]
    We have already computed the necessary information to solve this integral. Note:
    \begin{align*}
        \int_0^{2\pi}\int_0^{2\pi}K\sqrt{EG-F^2}dudv &= \int_0^{2\pi}\int_0^{2\pi} \frac{\cos{u}}{b(a+b\cos{u})}[b(a+b\cos{u})]dudv \\
        &= 2\pi\int_{u=0}^{u=2\pi}\cos{u}du \\
        &= 2\pi (0) \\
        &= 0
    \end{align*}
    Since we know that the Euler Characteristic of the torus is simply 0, we know the right hand side of Corollary 1.11 is also
    equal to 0. Therefore, Corollary 1.11 holds.\hfill$\blacksquare$\\

    $\S$3.1 \textbf{Question 8(a)}.\\
    Consider the paraboloid $M$ parameterized by \textbf{x}$(u,v) = (u\cos{v},u\sin{v},u^2),0\leq u,0\leq v\leq 2\pi$. Denote
    c$M_r$ to be the portion defined by $0\leq u\leq r$.\\
    (a) Calculate the geodesic curvature of the boundary circle and compute $\int_{\partial{M_r}}\kappa_gds$\\

    \textbf{Answer}.\\
    Since we are only considering the geodesic curvature of the \textit{boundary circle}, we can let $u=u_0=$const. Thus, we can
    define:
    \[ \alpha(v) := \mathbf{x}(u_0,v) = (u_0\cos{v},u_0\sin{v},u_0^2) \]
    It follows that $\alpha'(v) = (-u_0\sin{v},u_0\cos{v},0)$ and:
    \[ |\alpha'(v)| = \sqrt{(-u_0\sin{v})^2 + (u_0\cos{v})^2} = u_0 \]
    so that $\alpha$ is not arclength parameterized. Therefore, define the function $v(s) = \frac{s}{u_0}$. Then if we define
    $\beta(s) = \alpha(v(s))$, we obtain:
    \[ \beta'(s) = (-u_0\sin{v}(\frac{1}{u_0}),u_0\cos{v}(\frac{1}{u_0}),0) = (-\sin{v},\cos{v},0)\]
    which implies that $\beta$ is arclength parameterized. Now, we are ready to compute $\kappa_g$ directly from the definition.
    More specifically, since $\kappa_g = \mathbf{T}' \cdot (n\times \mathbf{T})$, we will compute the right hand side using $\beta$
    to get $\kappa_g$.\\
    Note, $\mathbf{T} = \beta'(s) = (-\sin{(\frac{s}{u_0})},\cos{(\frac{s}{u_0})},0)$ so that $\mathbf{T}' = \beta''(s) =
    (\frac{-1}{u_0}\sin{v},\frac{1}{u_0}\cos{v},0)$. Then, note that $x_u = (\cos{v},\sin{v},2u)$ and $x_v = (-u\sin{v},u\cos{v},0)$
    so that:
    \begin{align*}
        x_u \times x_v &= (-2u^2\cos{v},-2u^2\sin{v},u) \\
        \implies |x_u\times x_v| &= \sqrt{4u^4+u^2} = \sqrt{u^2}\sqrt{4u^2+1} = u\sqrt{4u^2+1}\\
        \implies \frac{x_u\times x_v}{|x_u\times x_v|} &= \frac{(-2u\cos{v},-2u\sin{v},1)}{\sqrt{4u^2+1}}\\
        \implies \mathbf{n}(s) &= \frac{x_u\times x_v}{|x_u\times x_v|} \circ \beta(s) \\
        &= \frac{(-2u_0\cos{(\frac{s}{u_0})},-2u\sin{(\frac{s}{u_0})},1)}{\sqrt{4u_0^2+1}}
    \end{align*}
    We now compute $\mathbf{n}\times\mathbf{T}$:
    \begin{align*}
        \mathbf{n}\times\mathbf{T} &= \frac{(-2u_0\cos{(\frac{s}{u_0})},-2u\sin{(\frac{s}{u_0})},1)}{4u_0^2+1} \times 
        (-\sin{(\frac{s}{u_0})},\cos{(\frac{s}{u_0})},0) \\
        &= \frac{(-\cos{(\frac{s}{u_0})},\sin{(\frac{s}{u_0})},-2u_0)}{\sqrt{4u_0^2+1}}
    \end{align*}
    Finally, it is left to see that:
    \begin{align*}
        \mathbf{T}'\cdot (\mathbf{n}(s) \times \mathbf{T}) &= ((\frac{-1}{u_0})\sin{(\frac{s}{u_0})},\frac{1}{u_0}
    \end{align*}
\end{document}
