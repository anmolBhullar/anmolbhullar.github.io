\documentclass[12pt]{article}
\usepackage{amsmath} 
\usepackage{amsthm} % Theorem Formatting
\usepackage{amssymb}    % Math symbols such as \mathbb
\usepackage{graphicx} % Allows for eps images
\usepackage[dvips,letterpaper,margin=1in,bottom=0.7in]{geometry}
\usepackage{tensor}
 % Sets margins and page size
\usepackage{amsmath}

\renewcommand{\labelenumi}{(\alph{enumi})} % Use letters for enumerate
% \DeclareMathOperator{\Sample}{Sample}
\let\vaccent=\v % rename builtin command \v{} to \vaccent{}
\usepackage{enumerate}
\renewcommand{\v}[1]{\ensuremath{\mathbf{#1}}} % for vectors
\newcommand{\gv}[1]{\ensuremath{\mbox{\boldmath$ #1 $}}}
% for vectors of Greek letters
\newcommand{\uv}[1]{\ensuremath{\mathbf{\hat{#1}}}} % for unit vector
\newcommand{\abs}[1]{\left| #1 \right|} % for absolute value
\newcommand{\avg}[1]{\left< #1 \right>} % for average
\let\underdot=\d % rename builtin command \d{} to \underdot{}
\renewcommand{\d}[2]{\frac{d #1}{d #2}} % for derivatives
\newcommand{\dd}[2]{\frac{d^2 #1}{d #2^2}} % for double derivatives
\newcommand{\pd}[2]{\frac{\partial #1}{\partial #2}}
% for partial derivatives
\newcommand{\pdd}[2]{\frac{\partial^2 #1}{\partial #2^2}}
% for double partial derivatives
\newcommand{\pdc}[3]{\left( \frac{\partial #1}{\partial #2}
 \right)_{#3}} % for thermodynamic partial derivatives
\newcommand{\ket}[1]{\left| #1 \right>} % for Dirac bras
\newcommand{\bra}[1]{\left< #1 \right|} % for Dirac kets
\newcommand{\braket}[2]{\left< #1 \vphantom{#2} \right|
 \left. #2 \vphantom{#1} \right>} % for Dirac brackets
\newcommand{\matrixel}[3]{\left< #1 \vphantom{#2#3} \right|
 #2 \left| #3 \vphantom{#1#2} \right>} % for Dirac matrix elements
\newcommand{\grad}[1]{\gv{\nabla} #1} % for gradient
\let\divsymb=\div % rename builtin command \div to \divsymb
\renewcommand{\div}[1]{\gv{\nabla} \cdot \v{#1}} % for divergence
\newcommand{\curl}[1]{\gv{\nabla} \times \v{#1}} % for curl
\let\baraccent=\= % rename builtin command \= to \baraccent
\renewcommand{\=}[1]{\stackrel{#1}{=}} % for putting numbers above =
\providecommand{\wave}[1]{\v{\tilde{#1}}}
\providecommand{\fr}{\frac}
\providecommand{\RR}{\mathbb{R}}
\providecommand{\NN}{\mathbb{N}}
\providecommand{\seq}{\subseteq}
\providecommand{\e}{\epsilon}

\newtheorem{prop}{Proposition}
\newtheorem{thm}{Theorem}[section]
\newtheorem{axiom}{Axiom}[section]
\newtheorem{p}{Problem}[section]
\usepackage{cancel}
\newtheorem*{lem}{Lemma}
\theoremstyle{definition}
\newtheorem*{dfn}{Definition}
 \newenvironment{s}{%\small%
        \begin{trivlist} \item \textbf{Solution}. }{%
            \hspace*{\fill} $\blacksquare$\end{trivlist}}%
% ***********************************************************
% ********************** END HEADER *************************
% ***********************************************************

\begin{document}

{\noindent\Huge\bf  \\[0.5\baselineskip] {\fontfamily{cmr}\selectfont  %
Assignment 1}         }\\[2\baselineskip] % Title
{ {\bf \fontfamily{cmr}\selectfont MATC63: Differential Geometry}\\ {\textit{\fontfamily{cmr}%
\selectfont September 19, 2017}}}
{\large \textsc{Anmol Bhullar}} % Author name
\\[1.4\baselineskip]

\begin{p}
    Let $\alpha(t) = (\frac{1}{\sqrt{3}}\cos(t)+\frac{1}{\sqrt{2}}\sin(t), \frac{1}{\sqrt{3}}\cos(t), \frac{1}{\sqrt{3}}\cos(t) - \frac{1}{\sqrt{2}}\sin(t))$. Calculate $\alpha'(t), \abs{\abs{\alpha'(t)}}$,
    and reparameterize $\alpha$ by arclength.
\end{p}
\begin{s}
    \begin{align*}
        \frac{d}{dt}(\alpha(t)) &= \frac{d}{dt}((\frac{1}{\sqrt{3}}\cos(t)+\frac{1}{\sqrt{2}}\sin(t), \frac{1}{\sqrt{3}}\cos(t), \frac{1}{\sqrt{3}}\cos(t) - \frac{1}{\sqrt{2}}\sin(t))) \\
        \alpha'(t) &= (\frac{-1}{\sqrt{3}}\sin(t) + \frac{1}{\sqrt{2}}\cos(t), \frac{-1}{\sqrt{3}}\sin(t), \frac{-1}{\sqrt{3}}\sin(t) - \frac{1}{\sqrt{2}}\cos(t))
    \end{align*}
    Now we calculate the norm of $\alpha$:
    \begin{align*}
        \abs{\abs{\alpha'(t)}} &= \sqrt{(\frac{-1}{\sqrt{3}}sin(t) + \frac{1}{\sqrt{2}}\cos(t))^2 +
            (\frac{-1}{\sqrt{3}}\sin(t))^2 + (\frac{-1}{\sqrt{3}}\sin(t) + \frac{-1}{\sqrt{2}}\cos(t))^2} \\
        &= \sqrt{\frac{1}{3}\sin^2(t) - 
            \frac{1}{\sqrt{3}}\sin(t)\cdot\frac{1}{\sqrt{2}}\cos(t) + \frac{1}{2}\cos^2(t) + \frac{1}{3}\sin^2(t) + 
            (\frac{-1}{\sqrt{3}}\sin(t) + \frac{-1}{\sqrt{2}}\cos(t))^2} \\
        &= \sqrt{\frac{2}{3}\sin^2(t) - 
            \frac{1}{\sqrt{3}}\sin(t)\cdot\frac{1}{\sqrt{2}}\cos(t) + \frac{1}{2}\cos^2(t) + \frac{1}{3}\sin^2(t) + 
            \frac{1}{\sqrt{3}}\sin(t)\cdot\frac{1}{\sqrt{2}}\cos(t) + \frac{1}{2}\cos^2(t)} \\
        &= \sqrt{\sin^2(t) - 
            \frac{1}{\sqrt{3}}\sin(t)\cdot\frac{1}{\sqrt{2}}\cos(t) + \cos^2(t) + 
            \frac{1}{\sqrt{3}}\sin(t)\cdot\frac{1}{\sqrt{2}}\cos(t)} \\
        &= \sqrt{\sin^2(t) + \cos^2(t)} \\
        \abs{\abs{\alpha'(t)}} &= 1
    \end{align*}
    Since $\abs{\abs{\alpha'(t)}} = 1$, this implies that $s(t) = t$, which implies that $\alpha$ is already
    arclength parametrized.
\end{s}

\begin{p}
    Reparametrize the catenary by arclength
\end{p}
\begin{s}
    Note, that computing the arc-length of $\alpha(t)$, we obtain $\sinh(b)$ (by the computation in 5(a)).
    So, we want to find isolate $t$ in the equation, $s = \sinh(t)$.
    So, consider:
    \begin{align*}
        s &= \sinh t \\
        2s &= e^t - e^{-t} \\
        2se^t &= e^{2t} - 1 \\
        (e^t)^2 - 2s(e^t) - 1 &= 0 \\
    \end{align*}
    Applying the quadratic formula, we get that:
    \begin{align*}
        e^t &= s \pm \sqrt{s^2 + 1} \\
        e^t &= s + \sqrt{s^2+1}\:\: (e^t\:\: \text{only admits positive values}) \\
        t &= \ln(s + \sqrt{s^2+1})
    \end{align*}
    Thus, we have that $\sinh^{-1}(s) = \ln(s+\sqrt{s^2+1}) = t$. So, we can reparametrize $\alpha$ by
    setting
    \[ \beta(s) = \alpha(t(s)) \]
    where $0 \leq s \leq \sinh(b)$, to get that:
    \begin{align*}
        \beta(s) &= (\sinh^{-1}(s), \cosh(\sinh^{-1}(s))) \\
        \implies \beta'(s) &= (\frac{1}{\sqrt{s^2+1}}, \frac{s}{\sqrt{s^2+1}}\sinh(\sinh^{-1}(s))) \\
        \beta'(s) &= (\frac{1}{\sqrt{s^2+1}}, \frac{s}{\sqrt{s^2+1}}) \\
        \implies \abs{\abs{\beta'(s)}} &= \sqrt{ (\frac{1}{\sqrt{s^2+1}})^2 + (\frac{s}{\sqrt{s^2+1}})^2 }\:\:\:\text{(note: }\: s\geq 0)\\
        \abs{\abs{\beta'(s)}} &= \sqrt{\frac{s^2+1}{s^2+1}} = \sqrt{1} = 1
    \end{align*}
    so that $\beta(s)$ is now arc-length parametrized.
\end{s}

\begin{p}
    Compute the curvature of the following arclength-parametrized curve:
    \[ \alpha(s) = (\sqrt{1+s^2},\: \ln(s+\sqrt{1+s^2})) \]
\end{p}
\begin{s}
    First, we calculate $\alpha'(s) = \mathbf{T}(s)$:
    \begin{align*}
        \alpha'(s) &= (\frac{s}{\sqrt{1+s^2}}, \frac{1+\frac{s}{\sqrt{1+s^2}}}{s+\sqrt{1+s^2}})) \\
        &= (\frac{s}{\sqrt{1+s^2}}, \frac{1}{\sqrt{s^2+1}})
    \end{align*}
    Then, we calculate $\alpha''(s) = \mathbf{T}'(s)$:
    \begin{align*}
        \alpha''(s) &= \Bigg{(}\frac{1}{(s^2+1)^{\frac{3}{2}}}, \frac{-s}{(s^2+1)^{\frac{3}{2}}}\Bigg{)}
    \end{align*}
    Calculating the norm of $\mathbf{T}'(s)$, we get the curvature (WLOG assume that $s \geq 0$ since we know
    $\alpha$ is a regular parametrized curve so it can be reparametrized): 
    \[ \mathcal{K}(s) = \abs{\abs{T'(s)}} = \sqrt{\frac{s^2}{(s^2+1)^3} + \frac{1}{(s^2+1)^3}} = \frac{1}{s^2+1}\]
\end{s}

\begin{p}
    Calculate the Frenet apparatus ($\mathbf{T}, \mathcal{K}, \mathbf{N}, \mathbf{B}$ and $\mathcal{T}$) of the
    following curve:
    \[ \alpha(t) = (t, \frac{t^2}{2}, t\sqrt{1+t^2} + \ln(t+\sqrt{1+t^2})) \]
\end{p}
\begin{s}
    Note that $\alpha(t)$ is not arc-length parametrized, so we will first have to calculate $v(t) = \abs{\abs{\alpha'(t)}}$ in order
    to calculate $\mathbf{T}(t)$. The last argument is quite complicated so we split it into parts and find their
    derivatives separately:
    \begin{align*}
        \frac{d}{dt}(t\sqrt{1+t^2}) &= \sqrt{1+t^2} + \frac{t^2}{\sqrt{1+t^2}} = \frac{2t^2+1}{\sqrt{t^2+1}} \\
        \frac{d}{dt}(\ln(t+\sqrt{1+t^2})) &= \frac{1+\frac{t}{\sqrt{t^2+1}}}{t+\sqrt{t^2+1}} = \frac{1}{\sqrt{t^2+1}} \\
        \implies \alpha'(t) &= (1, t, \frac{1}{\sqrt{t^2+1}} + \frac{2t^2+1}{\sqrt{t^2+1}}) = (1,t,2\sqrt{t^2+1}) \\
        \implies v(t) &= \sqrt{1+t^2+4(t^2+1)} = \sqrt{5+5t^2} = \sqrt{5}\sqrt{1+t^2}
    \end{align*}
    so, we have that:
    \begin{align*}
        \mathbf{T}(t) &= \sqrt{5}\sqrt{1+t^2}(1,t,2\sqrt{t^2+1}) = \sqrt{5}\cdot(\sqrt{1+t^2},\: t\sqrt{1+t^2},\: 
        2t^2+2)
    \end{align*}
    Now,
    \begin{align*}
        \mathcal{K}\mathbf{N} = \frac{d\mathbf{T}}{ds} = \frac{\frac{d\mathbf{T}}{dt}}{\frac{ds}{dt}} 
        &= \frac{1}{v(t)}\frac{d\mathbf{T}}{dt} \\
        &= \frac{1}{\sqrt{5}(t^2+1)}(\frac{-t}{\sqrt{t^2+1}},\:\frac{1}{\sqrt{t^2+1}},\:0)
    \end{align*}
    Thus, we have that:
    \begin{align*}
        \mathcal{K}(t) &= \frac{1}{\sqrt{5}(t^2+1)} \\
        \mathbf{N}(t) &= (\frac{-t}{\sqrt{t^2+1}},\:\frac{1}{\sqrt{t^2+1}}, 0)
    \end{align*}
    We can find $\mathbf{B}(t)$, normally by finding the cross product between $\mathbf{T}$ and 
    $\mathbf{N}$ (respectively):
    \[ \mathbf{B}(t) = \frac{1}{\sqrt{5}}(\frac{-2}{\sqrt{t^2+1}}, \frac{-2t}{\sqrt{t^2+1}}, 1) \]
    and, we calculate the torsion by using the formula $\mathcal{T}(t) = \mathbf{N}'(t)\cdot\mathbf{B}(t)$:
    \[ \mathcal{T}(t) = \frac{2}{\sqrt{5}(t^2+1)} \]
\end{s}

\begin{p}
    Show that the involute of a catenary is a tractrix.
\end{p}
\begin{s}
    The equation for an involute $(X(t), Y(t))$ of a parametrically defined function $t\mapsto (x(t),\:y(t))$ is:
    \begin{align*}
        X(t) &= x(t) - \frac{x'(t)}{\sqrt{(x'(t))^2+(y'(t))^2}}\cdot\int_{a}^{t}\sqrt{(x'(w))^2+(y'(w))^2}dw \\
        Y(t) &= y(t) - \frac{y'(t)}{\sqrt{(x'(t))^2+(y'(t))^2}}\cdot\int_a^t\sqrt{(x'(w))^2+(y'(w))^2}dw
    \end{align*}
    Thus, we just have to subsitute in the proper values. $x(t) = t, y(t) = \cosh(t)$ (from the parameteric 
    representation of the catenary). In an earlier question, we also discovered that the arclength of this
    parametrization is $\sinh(t)$. Now, we have enough information to calculate $X(t)$:
    \begin{align*}
        X(t) &= t - \frac{1}{\sqrt{1^2+\sinh^2(t)}}\cdot\int_a^t\sqrt{(x'(w))^2+(y'(w))^2}dw \\
        &= t - \frac{1}{\cosh(t)}\cdot\sinh(t) \\
        &= t - \tanh(t)
    \end{align*}
    and $Y(t)$:
    \begin{align*}
        Y(t) &= \cosh(t) - \frac{\sinh(t)}{\sqrt{1^2+\sinh(t)}}\cdot\sinh(t) \\
        &= \frac{\cosh^2(t) - \sinh^2(t)}{\cosh(t)} \\
        &= \frac{1}{\cosh(t)} \\
        &= \text{sech}(t)
    \end{align*}
    Thus, the involute of the catenary is given by the parametrization $t\mapsto(t-\tanh(t),$sech$(t))$ which is
    the parametrization of a tractrix as wanted.
\end{s}
\end{document}
