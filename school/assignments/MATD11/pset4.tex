\documentclass{article}
% \usepackage{tikz}
% \usetikzlibrary{cd}
\usepackage[utf8]{inputenc}
\usepackage[english]{babel}
\usepackage{amsfonts}
\usepackage{amsthm}
\usepackage{amsmath}
\usepackage{amssymb}
\usepackage{nicefrac}
\usepackage{mathrsfs}
\usepackage{dirtytalk}
\usepackage{exercise}
\usepackage{csquotes}

\newtheorem{theorem}{Theorem}
\newtheorem{es}{Examples}
\newtheorem{lemma}{Lemma}

\newcommand{\inter}[1]{int(#1)}
\newcommand{\norm}[1]{\left\lVert#1\right\rVert}

\title{MATD11: Functional Analysis\\
    Assignment 4}
\author{Anmol Bhullar - 1002678140}

\begin{document}
    \maketitle

    \begin{Exercise}
        For an operator $T$ in $\mathscr{B}(\mathscr{H})$ where $\mathscr{H}$ is a Hilbert space, show that ker $T$ = (ran $T^*)^{\perp}$, where
        ker $T = \{h:Th=0\}$ and ran $T^* \equiv$ range $T^* = \{T^*h: h\in\mathscr{H}\}$. 
    \end{Exercise}

    \begin{Answer}
        Choose $h\in$ ker $T$. Then $Th = 0$. Note, $h\in$ (ran $T^*)^{\perp}$ if and only if for all $g\in$ ran $T^*$, we have $\langle g, h\rangle = 0$.
        Since $g\in$ ran $T^*$, we can write $T^* k = g$ for some $k\in\mathscr{H}$. Then, consider:
        \[ \langle T^* k, h\rangle = \langle k, (T^*)^* h\rangle = \langle k, Th\rangle = \langle k,0\rangle = 0\]
        so that $h\in$ (ran $T^*)^{\perp}$. Thus, ker $T \subseteq ($ran $T^*)^{\perp}$.\\

        Choose $h\in ($ran $T^*)^{\perp}$. Then $h\in$ ker $T$ if and only if $Th=0$.
        \[ 0 = \langle h, g\rangle = \langle h, T^* k\rangle = \langle (T^*)^* h, k\rangle = \langle Th, k\rangle \]
        so that for all $k\in\mathscr{H}$, we have $\langle Th, k\rangle = 0$. In particular, for the identically zero operator $G$, we have that
        $\langle Th, k\rangle = \langle Gh, k\rangle$, thus $Th = Gh$ by a question on the previous assignment. Thus, $Th = 0$ so that $h\in$ ker $T$ and we
        have set containment both ways. Therefore, ker $T = ($ran $T^*)^{\perp}$.
    \end{Answer}

    \newpage
    \begin{Exercise}
        Show that a diagonal operator on a Hilbert space is an orthogonal projection if and only if its diagonal consists of 0's and 1's.
    \end{Exercise}

    \begin{Answer}
        Suppose $A$ is a diagonal operator and an orthogonal projection. We show its diagonal consists only of 0's and 1's. Since $A$ is
        a diagonal operator, for any $h\in\mathscr{H}$, we write 
        \[ Ah = \sum_1^{\infty} \langle h,e_n\rangle \alpha_n e_n = \sum_1^{\infty} \langle \alpha_n h, e_n\rangle e_n \]
        so that $A(Ah) = \sum_1^{\infty} \langle \alpha_n h,e_n\rangle \alpha_n e_n = \sum_1^{\infty} \langle h,e_n\rangle \alpha_n^2 e_n$.
        Recall, for all orthogonal projections $P$, we have $P^2 = P$. Thus, $A(Ah) = A(h)$, so that we have,
        \begin{equation}
            \sum_1^{\infty} \langle h,e_n\rangle \alpha_n e_n = \sum_1^{\infty} \langle h,e_n\rangle \alpha_n^2 e_n
        \end{equation}
        Thus, $\alpha_n = \alpha_n^2$ for all $n$. Clearly 1 and 0 are the only complex numbers which fulfill this equation. Thus,
        $\alpha_n = 0$ or $\alpha_n = 1$. Thus, the diagonal sequence of $A$ consists of 0's and 1's as wanted.\\

        Suppose $A$ is a diagonal operator and its diagonal sequence consists of 0's and 1's. Thus, $\alpha_n = \alpha_n^2$. From this,
        we get that (1) still holds. Thus, $A(Ah) = Ah$ so that $A^2 = A$. Now, we show $A$ is self-adjoint. Consider for $h,g\in\mathscr{H}$:
        \begin{align*}
            \langle Ah,g\rangle &= \langle \sum_{n=1}^{\infty} \langle h,e_n\rangle \alpha_n e_n,g\rangle  \\
                &= \sum_{n=1}^{\infty} \langle \alpha_n h,e_n\rangle \langle e_n,g\rangle 
        \end{align*}
        By theorem 1.33(f) (recall, we are working in a Hilbert space), 
        we have that the last line above is equal to $\langle \alpha_n h, g\rangle$ which is then
        equal to, $\langle h, \overline{\alpha_n} g\rangle$. Noting that all $\alpha_n$'s are real, we have
        from basic complex number properties that $\alpha_n = \overline{\alpha_n}$. Thus,
        \[ \langle Ah,g\rangle = \langle h,\alpha_n g\rangle\]
        Since $\mathscr{H}$ is Hilbert, we can of course write $g = \sum_{n=1}^{\infty} \langle g,e_n\rangle e_n$ so that,
        \begin{align*}
            \langle Ah, g\rangle &= \langle h,\alpha_n (\sum_{n=1}^{\infty} \langle g,e_n\rangle e_n)\rangle \\
                &= \langle h, \sum_{n=1}^{\infty} \langle g,e_n\rangle \alpha_n e_n\rangle \\
                &= \langle h, Ag\rangle
        \end{align*}
        so that $A$ is also self-adjoint. By problem 2.17, this is enough to imply that $A$ is an orthogonal projection.
    \end{Answer}

    \newpage
    \begin{Exercise}
        Show that if $S$ is the forward shift operator on $\ell^2$, then $S^n\to 0$ in the weak operator topology but $S^n$ does not converge to
        0 in the norm topology or the strong operator topology.
    \end{Exercise}

    \begin{Answer}
        First, note that in $\ell^2$, the adjoint of $S^n$ is $B^n$ where $B$ is the backward shift. 
        We can prove this quickly using that $S^* = B$, note:
        \[ (S^n)^* = S^* \circ S^* \circ \hdots \circ S^* = B \circ B \circ \hdots \circ B = B^n \]
        where we used proposition 2.13 to expand $(S^n)^*$. Then, for $h,g\in\ell^2$, we have:
        \[ \langle S^n h, g\rangle = \langle h, B^n g\rangle \]
        We show that $\lim_{n\to\infty} \norm{B^n g} = 0$. Since $g\in\ell^2$, we can write $g = (x_n)_1^{\infty}$ for $x_n\in\mathbb{C}$
        such that $(\sum_1^{\infty} |x_n|^2)^{1/2} < \infty$ and so, $\norm{B^n g} = (\sum_{i=n}^{\infty} |x_i|^2)^{1/2}$ by definition
        of the backward shift. Thus, in order to prove $\lim_{n\to\infty} \norm{B^ng} = 0$, it suffices to prove the tail of the sum
        $(\sum_{n=N}^{\infty} |x_n|^2)^{1/2}$ goes to zero i.e. $\lim_{N\to\infty} (\sum_{n=N}^{\infty} |x_n|^2)^{1/2} = 0$. 
        We already know this is true since the tail of \textit{any} convergent series goes to zero so we are done. Therefore,
        $\lim_{n\to\infty} \norm{B^n g} = 0$ and using continuity of norms, we have $\norm{\lim_{n\to\infty} B^n g} = 0_{\mathbb{R}}$ so that
        $\lim_{n\to\infty} B^n g = 0_{\ell^2}$. Noting that the inner product is continuous in \textit{both} arguements, we have,
        \[ \lim_{n\to\infty} \langle S^nh,g\rangle = \lim_{n\to\infty} \langle h,B^ng\rangle = \langle h,\lim_{n\to\infty} B^n g\rangle = 0 \]
        so that $S^n \to 0$ in the weak operator topology as wanted.\\

        We show $S^n \to 0$ is not true in the norm operator topology. Note,
        \[ \norm{S^n} = \sup\{\norm{S^n x}_{\ell^2}: \norm{x}_{\ell^2} = 1\} \]
        Writing $x$ as $(x_n)_1^{\infty}$ for $x_n\in\mathbb{N}$, we have $S^nx = (\sum_{i=1}^{\infty} |(S^nx)_i|^2)^{1/2}$. However, the first
        $n$ terms are simply zero, so we have $S^n x = (\sum_{i=n+1}^{\infty} |x|^2)^{1/2}$ and also, since $(S^n x)_n = x_1$,
        $(S^n x)_{n+2} = x_2$ and so on, we have that $S^n x = (\sum_{i=1}^{\infty} |x_i|^2)^{1/2}$ for all $n\in\mathbb{N}$. Thus,
        \[ \norm{S^n} = \sup\{\norm{S^n x}_{\ell^2}: \norm{x}_{\ell^2} = 1\} = \sup\{\norm{x}_{\ell^2}: \norm{x}_{\ell^2} = 1\} = 1 \]
        so that $\norm{S^n} = 1$ for all $n\in\mathbb{N}$ so that in particular, $S^n \to 0$ is not true in the norm operator topology.\\

        We show $S^n \to 0$ is not true in the strong operator topology. In the previous paragraph, we have shown $\norm{S^n x}_{\ell^2} =
        \norm{x}_{\ell^2}$ for all $n\in\mathbb{N}$. Thus, $\norm{S^n x} \to 0$ is not true for all $x\in \ell^2$ (pick an $x\in\ell^2$
        such that $\norm{x}_{\ell^2} > 0$) and so $S^n\to 0$ is not true in the strong operator topology.
    \end{Answer}

    \newpage
    \begin{Exercise}
        Let $\mathscr{H}$ be a Hilbert space. Prove that $U\in\mathscr{B}(\mathscr{H})$ is a self-adjoint unitary operator if and only if
        $U = 2P-I$ for some projection $P\in\mathscr{B}(\mathscr{H})$.
    \end{Exercise}

    \begin{Answer}
        Assume $U$ is a self-adjoint unitary operator. Define the map $P:\mathscr{H}\to\mathscr{H}$ by $P := (U+I)/2$. We show
        $P\in\mathscr{B}(\mathscr{H})$, $P^2 = P$ and $P$ is self-adjoint to get that $P$ is an orthogonal projection. Thus, by solving
        for $U$, we get that $U = 2P - I$ as wanted. Since $\mathscr{B}(\mathscr{H})$ is a vector space and they are closed under
        vector addition and scalar multiplication, we have that since $P = (U+I)/2$ and $U,I\in\mathscr{B}(\mathscr{H})$, then
        $P\in\mathscr{B}(\mathscr{H})$. Now, we show $P^2 = P$. Pick $x\in\mathscr{H})$, then
        \begin{align*}
            P^2(x) = P(\frac{(U+I)x}{2}) &= P(\frac{Ux + Ix}{2}) \\
                &= \frac{U((Ux+Ix)/2) + I((Ux+Ix)/2)}{2} \\
                &= \frac{U^2x/2 + (U\circ Ix)/2 + (I\circ Ux)/2 + I^2x/2}{2}
        \end{align*}
        Noting that since $I$ is the identity map, then $U = U\circ I = I \circ U$ and since $U$ is unitary, we have $U^2 = I$, we get:
        \[ P^2(x) = \frac{Ix/2 + (U\circ Ix)/2 + (U\circ Ix)/2 + Ix/2}{2} = \frac{Ux + Ix}{2} = Px \]
        as wanted. Thus, $P^2 = P$. Now, it is left to show $P$ is self-adjoint. Let $h,g\in\mathscr{H}$. Then,
        \begin{align*}
            \langle Ph,g\rangle &= \langle \frac{Uh + Ih}{2},g\rangle \\
                &= \langle Uh,g/2\rangle + \langle h,g/2\rangle \\
                &= \langle h, U(g/2)\rangle + \langle h,g/2\rangle \\
                &= \langle h,(1/2)(Ug + g)\rangle = \langle h,(Ug+Ig)/2\rangle = \langle h,Pg\rangle
        \end{align*}
        where in the second last line, we use the fact that $U$ is self-adjoint and in the last line that $U$ and $I$ are linear. Thus,
        $P$ is self adjoint so that by problem 2.17, $P$ is an orthogonal projection. Therefore, we can write $U = 2P - I$ for some
        orthogonal projection $P$.\\

        Now, assume $U = 2P-I$ for some orthogonal projection $P$, we prove that $U$ is self-adjoint and unitary. For $h,g\in\mathscr{H}$:
        \begin{align*}
            \langle Uh, g\rangle &= \langle (2P-I)h, g\rangle \\
                &= \langle Ph, 2g\rangle - \langle Ih,g\rangle \\
                &= \langle h,P(2g)\rangle - \langle h,Ig\rangle \\
                &= \langle h,2Pg - Ig\rangle \\
                &= \langle h,Ug\rangle 
        \end{align*}
        where we used the fact that $P$ is self adjoint and linear in the third and fourth line respectively. Thus, $U$ is self-adjoint and
        so it is left to show $U$ is unitary, i.e. show $UU^* = I = U^*U$. Note, we showed $U$ is self adjoint, so it suffices to show
        $U^2 = I$. Note, for any $x\in\mathscr{H}$:
        \[ U^2(x) = U(2Px - Ix) = 2P(2Px - Ix) - I(2Px - Ix) = 4P^2x - 2P\circ Ix - I\circ 2Px + I^2x \]
        Since $I$ is the identity map, we have that $2P\circ Ix = 2Px = I\circ 2Px$. Also, $I^2 = I$ and since $P$ is the projection map,
        $P^2 = P$. Thus, we can simplify the equation above to:
        \[ U^2(x) = 4Px - 4Px + Ix = x \]
        so that $U^2 = I$ as wanted.
    \end{Answer}

    \newpage
    \begin{Exercise}
        In $c_{00}(\mathbb{C})$ with the sup norm, find a nested sequence of non-empty closed sets with diameters approaching zero that has
        an empty intersection. Sufficiently justify your findings. Explain why your solution does not contradict the Nested Set Theorem.
    \end{Exercise}

    \begin{Answer}
        Define,
        \begin{align*}
            D_1 &:= D_1((1,0,0,\hdots)) = \{x: \norm{x-(1,0,0,\hdots)}_{c_{0}}\leq 1\} \\
            D_2 &:= D_{1/4}((1,1/4,0,0,\hdots)) = \{x: \norm{x-(1,1/4,0,0,\hdots)}_{c_0}\leq 1/2\} \\
            D_3 &:= D_{1/9}((1,1/4,1/9,0,0,\hdots)) \\
            &\vdots
        \end{align*}
        so that $D_n$ is defined to the closed ball of radius $1/n^2$ whose center is 
        \[ (1,1/4,1/9,\hdots,1/n^2,0,0,\hdots) \]
        We claim that $D := \{D_1,D_2,D_3,\hdots\}$ is our desired sequence of non-empty closed sets with diameters approaching zero that has
        an empty intersection. It is clear that this collection has diameters which approach zero and also that each point in this
        collection is closed since each closed ball is closed. It is left to prove $\cap_n D_n = \emptyset$ and $D_{n+1}\subseteq D_n$. We prove
        the latter first.\\

        It is clear each $D_n$ is non-empty. Thus, choose $x\in D_{n+1}$ and write $x = (x_1,x_2,x_3,\hdots)$ for $x_i\in\mathbb{C}$.
        Let $C_n$ denote the center of $D_n$. Then, given $\norm{x-C_{n+1}}\leq 1/(n+1)^2$, we have to show $\norm{x-C_n}\leq 1/n^2$.
        Note, it suffices to prove that for all $k\in\mathbb{N}$, $|x_k - (C_n)_k|\leq 1/n^2$.
        Note, we have that for all $k\geq 1$, since $(C_{n+1})_{n+k}=0$, we have $|x_{n+k} - 0|\leq 1/(n+1)^2 < 1/n^2$
        and also, since $(C_n)_k = (C_{n+1})_k$ agree for all $k<n$, we also, have $|x_k - (C_n)_k|<1/n^2$. Thus, the only non-trivial case
        is $k = n$. Note, we have $|x_k - (C_{n+1})_k| < 1/(n+1)^2$ so that since $(C_{n+1})_k = 1/(n+1)^2$, we have
        $|x_k - 1/(n+1)^2| \leq 1/(n+1)^2$ and in particular, $|x_k|\leq 2/(n+1)^2\leq 2/n^2$. So that by the reverse triangle inequality, we have
        $|x_k - 1/n^2| \leq ||x_k|-1/n^2| \leq 1/n^2$ as wanted. Therefore, $|x_k-(C_n)_k|\leq 1/n^2$ so that $\norm{x - C_n}\leq 1/n^2$ as wanted.
        Therefore, $x\in D_n$ so that our collection $D$ is a nested sequence of closed sets. We now prove $\cap_n D_n = \emptyset$.\\

        Suppose instead that there exists $x\in c_{00}$ such that $x\in \cap_n D_n$. But $x\in c_{00}$, so in particular, $x$ only contains
        $i$-many non-zero terms for finite integer $i$. This is a contradiction since $D_{i+1}$ clearly contains elements which contain 
        \textit{at least} $i+1$ many non-zero terms. Thus, $\cap_n D_n = \emptyset$ as wanted.
    \end{Answer}

    \newpage
    \begin{Exercise}
        Let $J$ be the set of irrational numbers in $[0,1]$. Show that it is impossible to write $J = \cup_1^{\infty} F_n$ where for each
        $n\in\mathbb{N}$, $F_n$ is a non-empty, closed subset of $\mathbb{R}$.
    \end{Exercise}

    \begin{Answer}
        Suppose we exist in perfect world where if someone writes down something, then it must be true. Then consider the following:
        \begin{displayquote}
            We can write $J = \cup_n F_n$ where each $F_n$ is closed and non-empty.
        \end{displayquote}
        Thus, we can write $J = \cup_n F_n$ for non-empty closed $F_n$'s. 
        Note $[0,1]$ equipped with the Euclidean norm is complete. We have that if $\{q_1,q_2,\hdots\}$ is an enumeration of the rationals
        in $[0,1]$, we we can write:
        \[ [0,1] = J^c \cup J = \{q_1,q_2,\hdots\} \cup \bigcup_{n=1}^{\infty} F_n = \bigcup_{n=1}^{\infty} \{q_n\} \cup F_n \]
        Note, each $A_n := \{q_n\}\cup F_n$ is closed since $F_n$ and $\{q_n\}$ are closed. Furthermore, $A_n$ does not contain a non-trivial
        interval (e.g. not $(i,i)$ or $[i,i]$) since all non-trivial intervals contain infinitely many rationals and irrational numbers. By
        the Baire-Category theorem, at least one of the $A_n$'s has the proprety that $A_n^{\circ}\neq \emptyset$. Thus, there exists some
        non-trivial interval $(a,b)\subseteq[0,1]$ such that $(a,b)\subseteq A_n$. However, this is impossible as $A_n$ contains no non-trivial
        intervals. Therefore, we have a contradiction and made an assumption that was not true. This assumption is the fact the world is
        in fact, \textit{not} perfect, and we cannot write $J = \cup_n F_n$.
    \end{Answer}

    \newpage
    \begin{Exercise}
        Let $X$ be a Banach space and consider sequences $\{B_n\}$ and $\{A_n\}$ in $\mathscr{B}(X)$. Assume $A,B:X\to X$ are the pointwise
        limits of $\{A_n\}$ and $\{B_n\}$ respectively. Prove that the sequence $\{B_nA_n\}$ converges pointwise to $BA$.
    \end{Exercise}

    \begin{Answer}
        Choose any $x\in X$ and any $\epsilon>0$. We know $B\in\mathscr{B}(X)$ so that for some non-zero real number $M$, $\norm{B}\leq M$.
        
        We know $A_n\to A$ pointwise so for $\epsilon/[2M]>0$, we know there exists some $N_1>0$ such that if 
        $n>N_1$, then $\norm{A_nx - Ax}_X<\epsilon$. Similarly, we know $B_n\to B$ pointwise, thus for $\epsilon/2>0$, there exists $N_2$
        such that if $n>N_2$, then $\norm{B_n(Ax) - B(Ax)} < \epsilon/2$. Choose $N =$ max$\{N_1,N_2\}$. Then, for all $n>N$:
        \begin{align*}
            \norm{B_nA_nx - BAx}_X &= \norm{B_nA_nx - B_nAx + B_nAx - BAx} \\
                &= \norm{B_nA_nx - B_nAx} + \norm{B_nAx - BAx} \\
                &= \norm{B_n(A_nx) - B_n(Ax)} + \norm{B_n(Ax) - B(Ax)} \\
                &= \norm{B_n(A_nx - Ax)} + \norm{B_n(Ax) - B(Ax)} \\
                &\leq \norm{B_n}\norm{A_nx - Ax} + \epsilon/2 \\
                &< M(\epsilon/[2M]) + \epsilon/2 \\
                &= \epsilon
        \end{align*}
        so that $\norm{B_nA_nx - BAx}\to 0$. Thus, $B_nA_n\to BA$ as wanted. Note, the existence $B$ and $A$ are given by the fact that
        $X$ is Banach.
    \end{Answer}

    \newpage
    \begin{Exercise}
        Let $F$ be the set of \say{eventually zero} sequences, in the supremum norm; this means that a sequence $\{a_n\}\in\ell^{\infty}$ belongs
        to $F$ if there is an $N$ with $a_n = 0$ for all $n\geq N$. Define the linear maps $T_n:F\to\mathbb{C}$ by
        \[ T_n(\{a_k\}) = \sum_{k=1}^n a_k \]
        Show that each $T_n$ is linear and bounded and for any fixed sequence $x = \{a_k\}$ in $F$, $\sup\{|T_n(x)|:n=1,2,3,\hdots\}$ is finite.
        Is $\sup\{\norm{T_n}:n=1,2,3,\hdots\} < \infty$?
    \end{Exercise}

    \begin{Answer}
        Let $x,y\in F$ and $a,b$ scalars. Then, write $x = (x_1,\hdots)$ and $y = (y_1,\hdots)$. Note, by definition of vector addition in
        $\ell^p$, we have $ax + by = (ax_1 + by_1,ax_2+by_2,\hdots)$. Then, consider:
        \begin{align*}
            T_n(ax+by) &= \sum_{k=1}^{n}  (ax_k+by_k) \\
                &= \sum_{k=1}^{n} ax_k + \sum_{k=1}^{\infty} by_k \\
                &= a\sum_{k=1}^{n} x_k + b\sum_{k=1}^{\infty} y_k \\
                &= aT_n(x) + bT_n(y)
        \end{align*}
        so that each $T_n$ is linear. Each $T_n$ is bounded as seen by:
        \[ \norm{T_n} = \sup\{|T_nx|: \norm{x}_{F}=1\} = n \]
        since $\norm{x}_F = 1$ implies that the first $n$ terms of $x$ are \textit{at most} 1. Then clearly, the supremum is achieved if
        \textit{all} of the first $n$ terms are 1 which implies $|T_nx| = n$. This also implies that $\sup\{\norm{T_n}:n=1,2,\hdots\}$ is
        not finite.\\

        Now, fix any sequence $x = (a_k)$ in $F$. Note by definition of $T_n$, we have that $T_nx \leq T_{n+1}x$. Thus, it follows
        $\norm{T_n}\leq \norm{T_{n+1}}$. Therefore, $\sup\{\norm{T_n}:n=1,2,3,\hdots\}=\lim_{n\to\infty} T_n$. This just a limit of partial
        sums and we know this converges since $x\in\ell^{\infty}$. Therefore, $\sup\{\norm{T}:n=1,2,3,\hdots\}$ exists and is finite.
    \end{Answer}

    \begin{Exercise}
        Let $X$ and $Y$ be normed linear spaces and suppose $T:X\to Y$ is linear. Show that $T$ is continuous if $\phi\circ T$ is continuous
        for all $\phi\in Y^*$.
    \end{Exercise}
    \begin{Answer}
        By corollary 3.5, since $Tx$ for any $x\in X$ is an element of $Y$, we have that,
        \[ \norm{Tx}_Y = \sup\{|\phi(Tx)|: \phi\in Y^*,\norm{\phi}=1\} \]
        and that there exists some specific $\phi_0\in Y^*$ for which this supremum is attained so that,
        \[ \norm{Tx}_Y = |\phi_0(Tx)| = |(\phi_0 \circ T)x| \leq \norm{\phi_0\circ T}\norm{x}_X \]
        where we used the fact that composition is associative and $\phi_0\circ T$ is continuous and thus, bounded. Thus, $\norm{T}$
        is bounded by $\norm{\phi_0\circ T}$ so that $T$ is bounded and so, continuous.
    \end{Answer}

    \begin{Exercise}
        Let $X$ be a normed linear space and let $Y$ be a closed subspace of $X$. Given any $x_0\in X$ where $x_0\not\in Y$, prove that there is
        some $f\in X^*$ such that $f(Y) = 0$ and $f(x_0) = 1$.
    \end{Exercise}
    \begin{Answer}
        Note, we have that span$\{x_0\}\cap Y = \{0\}$. Since $Y$ is closed and $x_0\not\in Y$, we have that $x_0\in Y^c$ and $Y^c$ open so that
        by definition, there exists some ball around $x_0$ which does not intersect $Y$. Thus, $d(x_0,Y)>0$. Take some $\alpha x_0\in$ span$(Y)$
        such that $\alpha\in\mathbb{C}-\{0\}$, then $d(x_0,Y)>0$ implies,
        \begin{align*}
            0 < \inf\{|\alpha|\norm{x_0-z}: z\in Y\} &= \inf\{\norm{\alpha x_0 - \alpha z}:z/|\alpha|\in Y\} \\
                &= \inf\{\norm{\alpha x_0 - y}:y\in Y\}
        \end{align*}
        so that $d(\alpha x_0, Y)>0$. In particular, this implies span$\{x_0\}\cap Y = \{0\}$. Now, define a mapping 
        $f':$ span$\{x_0\}\cup Y\to \mathbb{C}$ by $x\in$ span$\{x_0\} \mapsto \alpha\norm{x}/\norm{x_0}$ (clearly $x_0\neq 0$ since $0\in Y$)
        and $x\not\in$ span$\{x_0\}\mapsto 0$. span$\{x_0\}\cap Y=0$ ensures this is a well defined mapping. We show $f$ is linear
        and bounded. Choose $x,y\in$ span$\{x_0\}\cup Y$ and $a,b$ scalars. Then, if $x,y\in$ span$\{x_0\}$, we have:
        \[ f'(ax) = a\norm{x}/\norm{x_0} = a(\norm{x}/\norm{x_0}) = af'(x) \]
        and since $x,y\in$ span$\{x_0\}$, we have for some $\alpha\in\mathbb{C}$, $x = \alpha y$. Thus, 
        \[ f'(x+y) = f'((\alpha+1)x) = (\alpha+1)f'(x) = \alpha f'(x) + f'(x) = f'(\alpha x) + f'(x) = f'(y) + f(x) \]
        so that $f'$ is linear for $x,y\in$ span$\{x_0\}$. If $y$ in $Y$, then either $y$ is not in span$\{x_0\}$ (assuming $x$, $y$ not 0) so that:
        $f'(by) = 0 bf'(y)$ and $f(x+y) = 0$ (since $x+y\not\in$ span$\{x_0\}$). The case is even more trivial if both $x,y\not\in$ span$\{x_0\}$. Thus,
        $f'$ is linear. Clearly, $f$ is bounded by $1/\norm{x_0}$ and $f'(Y) = 0$. Note, $f'(x_0) = \norm{x_0}/\norm{x_0} = 1$. We
        can apply the Hahn-Banach theorem to get a mapping $f:X\to\mathbb{C}$ such that $f\in X^*$ and $f$ restricted to 
        span$\{x_0\}\cup Y$ is equal to $f'$. Thus, $f(Y) = 0$ and $f(x_0) = 1$ as wanted.
    \end{Answer}

\end{document}
