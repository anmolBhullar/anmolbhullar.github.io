\documentclass{article}
\usepackage[utf8]{inputenc}
\usepackage{amsmath}
\usepackage{amssymb}
\usepackage{amsfonts}
\usepackage{amsthm}

\newtheorem{theorem}{Theorem}[section]
\newtheorem{corollary}{Corollary}[theorem]
\newtheorem{lemma}[theorem]{Lemma}
\newtheorem{example}[theorem]{Example}
\newtheorem{exercise}[theorem]{Exercise}

\title{Algebraic Topology Companion Notes}
\author{Anmol Bhullar}
\date{April 2018}

\begin{document}

\maketitle

Filling in some details or trying some proofs myself from James Munkres' \textit{Topology}.

\section*{\S 51 Homotopy of Paths}

\setcounter{section}{51}

\begin{lemma}
	The relations $\simeq$ and $\simeq_p$ are equivalence relations.
\end{lemma}
\begin{proof}
	First, we show $\simeq$ is an equivalence relation between homotopic continuous functions from $X\to Y$.\\
	We will prove reflexivity first. Let $f: X\to Y$ be continuous. We show $f$ is homotopic with itself. To do this, we have to show there exists a continuous mapping 
	$F:X\times I\to Y$ such that $F(x,0) = f(x)$ and $F(x,1)=f(x)$ for all $x\in X$. Thus, define $F$ to be given by $(x,t)\mapsto f(x)$. Since $f$ is continuous, it 
	follows that $F$ is then also continuous.\\
	Now, we show symmetry. Let $f$ and $g$ be any homotopic functions from $X$ to $Y$. We show $g\simeq f$. To do this,
	we have to show there exists a continuous function $F: X\times I\to Y$ (not to be confused with the one above, we are no longer using it) such that $F(x,0) = g(x)$ and
	$F(x,1) = f(x)$. Thus, define $F$ to be given by the mapping $(x,t) \mapsto G(x,1-t)$ where $G$ is the homotopy between $f$ and $g$. Thus, we immediately obtain that
	$F(x,0)=G(x,1)=g(x)$ and $F(x,1)=G(x,0)=f(x)$ as wanted. To see this is continuous, it suffices to see that the component functions are continuous. It is clear that
	$\pi_1(F(x,t))$ is continuous since $G$ is continuous. The fact that $\pi_2(F(x,t))$ is continuous follows from the fact that it is a composition of continuous functions
	(namely $(x,t)\mapsto (x,1-t)$ and $(x,t)\mapsto G(x,t)$).\\
	Now, we prove $\simeq$ is transitive to conclude that $\simeq$ is an equivalence relation. Let $f\simeq g$
	and $g\simeq h$ for homotopic functions $f,g:X\to Y$ and homotopic functions $g,h:X\to Y$. Let $F'$ and $F''$ be the homotopy of $f,g$ and $g,h$ respectively.
	Now, define $H:X\times [0,2]\to Y$ via $(x,t)\mapsto F'(x,t)$ if $t\leq 1$ and $(x,t)\mapsto F''(x,t-1)$ if $t\geq 1$. We see that $H$ is well defined since
	$H(x,1) = F'(x,1) = g(x) = F''(x,0)$. $H$ is also continuous since $H$ is continuous on $t<1$ via $F'$ and continuous on $t>1$ via $F''$. Thus, by the pasting lemma, we have
	that $H$ is continuous on $X\times [0,2]$. Using this as motivation, we define the homotopy $\delta: X\times I \to Y$ between $f,h$ via $(x,t) \mapsto F'(x,2t)$ if $t\leq 
	1/2$ and $(x,t)\mapsto F''(x,2t-1)$ if $t\geq 1/2$.\\
	
	Now, we show $\simeq_p$ is an equivalence relation. Let $f: I\to X$ be a continuous path from $x_0$ to $x_1$ and define $F$ to be the homotopy from the reflexive 
	proof of $\simeq$. Then, we only need to show the additional condition that $F(0,t) = x_0$ and $F(1,t) = x_1$. Note $F(x,t) = f(x)$ so $F(0,t) = f(0) = x_0$ and $F(1,t)=f(1)=x_1$
	as wanted. Thus, $f\simeq_p f$ showing reflexivity. Now, we show symmetry. Let $f,f':I\to X$ be two homotopic paths from $x_0$ to $x_1$. Define $F:I^2 \to X$ via
	$(s,t)\mapsto F'(s,1-t)$ where $F'$ is the path homotopy between $f$ and $f'$. Then from the symmetry proof of $\simeq$ above, we have only left to prove that
	$F(0,t) = x_0$ and $F(1,t) = x_1$. Note $F(0,t) = F'(0,1-t) = x_0$ and $F(1,t) = F'(1,1-t) = x_1$ as wanted. We now, only need show transitivity to conclude that $\simeq_p$ is
	an equivalence relation. Let $f,g:I\to X$ and $g,h:I\to X$ be path homotopies between $x_0$ and $x_1$. We show $f,h$ are path homotopic. Define $\delta$ to be a path
	homotopy between $f$ and $h$ the same way it was done in the transitive proof above. Thus, it is left to prove $\delta(0,t) = x_0$ and $\delta(1,t) = x_1$.
	If $t<1/2$, we have $\delta(0,t) = F'(0,2t) = x_0$ since $F'$ is a path homotopy between $f$ and $g$ which are paths between $x_0$ and $x_1$. Similarly, $\delta(1,t) = x_1$.
	We can repeat this for $t\geq 1$ to again, get the same result. Thus, $\delta$ is a path homotopy as wanted. Therefore, $\simeq_p$ is an equivalence relation as wanted.
\end{proof}

\begin{example}
	Let $f$ and $g$ be any two maps of a space $X$ into $\mathbb{R}^2$. It is easy to see that $f$ and $g$ are homotopic; the map
	\[ F(x,t) = (1-t)f(x) + tg(x) \]
	is a homotopy between them called the \textbf{straight line homotopy}.
\end{example}
\begin{proof}
	Note that $F(x,0) = (1-0)f(x) + 0g(x) = f(x)$ and $F(x,1) = (1-1)f(x) + 1g(x) = g(x)$ as wanted. $F$ is continuous because $f$ and $g$ are continuous functions on $X$, thus,
	so are $(1-t)f(x)$ and $tg(x)$ for all $t\in\mathbb{R}$. $F(x,t)$ is the sum of these two functions so, we immediately obtain that $F$ is a continuous function. Thus, $F$ is a homotopy
	between $f$ and $g$. 
\end{proof}

\subsubsection*{Exercise 1}

Show that if $h,h':X\to Y$ are homotopic and $k,k':Y\to Z$ are homotopic, then $k\circ h$ and $k'\circ h'$ are homotopic.
\begin{proof}
	We want to deform $k\circ h: X\to Z$ into $k' \circ h':X \to Z$ in a continuous manner. Let $H$ be the homotopy between $h$ and $h'$ and $K$ be the homotopy between $k$ and $k'$.
	Let,
	\[ G: (x,t) \mapsto K(H(x,t),t) \]
	If $t=0$, we have $K(H(x,0),0) = k(H(x,0)) = k\circ h (x)$. If $t=1$, we have $K(H(x,1),1) = k'(H(x,1)) = k' \circ h' (x)$. $G$ is a composition of continuous functions, so it is continuous.
	Thus, $k\circ h$ and $k'\circ h'$ are homotopic.
\end{proof}

\subsubsection*{Exercise 2}

Given spaces $X$ and $Y$, let $[X, Y]$ denote the set of homotopy \textit{classes} of maps of $X$ into $Y$.
\begin{enumerate}
	\item[(a)] Let $I = [0,1]$. Show that for any $X$, the set $[X, I]$ has a single element.
	\item[(b)] Show that if $Y$ is path connected, the set $[I, Y]$ has a single element.
\end{enumerate}
\begin{proof}
	For (a), it suffices to show that any continuous maps $f,f': X\to I$ is homotopic. Note $I$ is convex so the straight line homotopy is enough to show that $f$ and $f'$ are homotopic.
\end{proof}

\subsubsection*{Exercise 3}

A space $X$ is said to be \textbf{contractible} if the identity map $i_X: X\to X$ is nulhomotopic (able to be continuously deformed into a constant map).
\begin{enumerate}
	\item[(a)] Show that $I$ and $\mathbb{R}$ are contractible.
	\item[(b)] Show that a contractible space is path-connected.
	\item[(c)] Show that if $Y$ is contractible, then for any $X$, the set $[X,Y]$ has a single element.
	\item[(d)] Show that if $X$ is contractible and $Y$ is path connected, then $[X,Y]$ has a single element.
\end{enumerate}
\begin{proof}
	(a): Define $F: I\times I\to I$ to be given by $F(x,t) = (1-t)x$. This is the straight line homotopy between the identity map $i_I$ and the constant map $x\mapsto 0$.
	$\mathbb{R}$ is contractible for the same reason, simply apply the straight line homotopy between $i_{\mathbb{R}}$ and $x\mapsto 0$. Note, this is possible since both $I$ and 
	$\mathbb{R}$ are convex.\\
	
	(b): Let $X$ be a contractible space. Let $F$ be the homotopy between $i_X$ and some constant map $f$. Then, $\psi: t\mapsto F(x,t)$ is a continuous path with endpoints $x$ and $f(x)$.
	Similarly, $\phi: t\mapsto F(y,t)$ is a continuous path with endpoints $y$ and $f(y)$. Since $f$ is a constant map, we have $f(x)=f(y)$ and so we can construct a continuous path $\Phi: I\to X$
	using the pasting lemma with endpoints $x$ and $y$.\\
	
	(c): Let the identity map on $Y$ be homotopic to the constant map, mapping everything to $y_0\in Y$. Let $g: X\to Y$ be given by $x\mapsto y_0$ and $f: X\to Y$ be any continuous map.
	Let $H$ be the homotopy between $i_Y$ and the map $Y\mapsto y_0$. Define $F: X\times I \to Y$ be given by $F(x,t) = H(f(x),t)$. Then, $F$ is a homotopy since $H$ is and so $[X,Y]$ has
	a single element.\\
	
	(d): Let $H$ be the homotopy between the identity map on $X$ and the map $X\mapsto x_0\in X$. Then, $f\circ H: X\times I \to Y$ is a homotopy between $f$ and $X\mapsto f(x_0)$ since
	$f\circ H(x,0) = f(x)$ and $f\circ H(x,1) = f\circ x_0 = f(x_0)$. Taking $g: X\to Y$ to be another continuous map, we can see $g\circ H$ is a homotopy between $g$ and $X\mapsto g(x_0)$.
	$Y$ is path connected so there exists a path between $g(x_0)$ and $f(x_0)$. Through this, we get a homotopy between $f$ and $g$. Thus, $[X,Y]$ has a single element.
\end{proof}

Side note: One could think that you may not need contractibility to be able to answer (d) if we let our homotopy $H(x,t)$ be equal to the path that connects some continuous functions $f(x)$ and $g(x)$
in $Y$. However, this mapping is not continuous.

\section{The Fundamental Group}

\subsection*{Exercise 1}

A subset $A$ of $\mathbb{R}^n$ is said to be \textbf{star convex} if for some point $a_0$ of $A$, all the line segments joining $a_0$ to other points of $A$ lie in $A$.
\begin{enumerate}
	\item[(a)] Find a star convex set that is not convex
	\item[(b)] Show that if $A$ is star convex, $A$ is simply connected.
\end{enumerate}
\begin{proof}
	\item[(a)] Let $A$ be a star in $\mathbb{R}^2$ centered at $(0,0)$. This is clearly not convex just by taking any two points on any (distinct) "tips" of the star.
	\item[(b)] Let $a_0\in A$ be the point which is connected to any other point of $A$ via a straight line (lying entirely in $A$). 
	From this property of $a_0$, we see that $A$ is path connected. Now, we show that $|\pi_1 (A,a_0)| = 1$ and so the fundamental group based at any other point also has order 1 
	(and so it must be the trivial fundamental group). To see this, let $f$ and $g$ be any two paths based at $a_0$. 
	Define $F: I^2 \to A$ by $F(x,t) = [\phi_{f(x)} * \bar{\phi}_{g(x)}] (t)$ where $\phi_a: I\to A$ is the straight line connecting $a$ and $a_0$. Intuitively, $F$ deforms $f$ into $g$ by first
	taking $f(x)$ to $a_0$, then takes $a_0$ to $g(x)$. Thus, $f\simeq_p g$ and so $|\pi_1(A,a_0)| = 1$ as wanted.
\end{proof}

\subsection*{Exercise 2}

Let $\alpha$ be a path in $X$ from $x_0$ to $x_1$; let $\beta$ be a path in $X$ from $x_1$ to $x_2$. Show that if $\gamma = \alpha * \beta$, then $\hat{\gamma} = \hat{\beta}\circ \hat{\alpha}$.
\begin{proof}
	\begin{align*}
		\hat{\beta} \circ \hat{\alpha([f])} &= [\overline{\beta}] * ([\overline{\alpha}] * [f] * [\alpha]) * [\beta] \\
			&= ([\overline{\beta}] * [\overline{\alpha}]) * [f] * ([\alpha] * [\beta]) \\
			&= [\overline{\beta} * \overline{\alpha}] * [f] * [\alpha * \beta]
	\end{align*}
	Recall from group theory that $(a \cdot b)^{-1} = b^{-1} \cdot a^{-1}$ so:
	\begin{align*}
		\hat{\beta} \circ \hat{\alpha}([f]) = [\overline{\alpha * \beta}] * [f] * [\alpha * \beta] = \hat{\gamma}([f])
	\end{align*}
	as wanted.
\end{proof}

\subsection*{Exercise 3}

Let $x_0$ and $x_1$ be points of the path-connected space $X$. Show that $\pi_1(X,x_0)$ is abelian if and only if for every pair $\alpha$ and $\beta$ of paths from $x_0$ to $x_1$, we have
$\hat{\alpha} = \hat{\beta}$.
\begin{proof}
	Assume the fundamental group at $x_0$ is abelian. Consider:
	\begin{align*}
		\hat{\alpha}([f]) &= [\overline{\alpha}] * [f] * [\alpha] \\
			&= ([\overline{\beta}] * [\beta]) * [\overline{\alpha}] * [f] * [\alpha] * ([\overline{\beta}] * [\beta]) \\
			&= [\overline{\beta}] * ([f] * [\alpha] * [\overline{\beta}]) * ([\beta]) * [\overline{\alpha}]) * [\beta] \\
			&= [\overline{\beta}] * [f] * [\beta] = \hat{\beta}([f])
	\end{align*}
	where we have used the fact that $([f] * [\alpha] * [\overline{\beta}])$ is a path from $x_0$ to $x_0$ and so is $([\beta]) * [\overline{\alpha}])$ and so we can switch them using our
	abelian property. Now, assume instead that $\hat{\alpha} = \hat{\beta}$.\\
	
	We know that $\gamma := f * \alpha$ is a path from $x_0$ to $x_1$ and so $\hat{\gamma} = \hat{\alpha}$. Note $[\overline{\gamma}] = [\overline{f * \alpha}] = [\overline{\alpha}] * [\overline{f}]$
	and so,
	\begin{align*}
		\hat{\gamma}([f * g]) &= [\overline{\gamma}] * [f * g] * [\gamma] \\
			&= [\overline{\alpha}] * [\overline{f}] * [f] * [g] * [f] * [\alpha] \\
			&= [\overline{\alpha}] * [g] * [f] * [\alpha] \\
			&= [\overline{\alpha}] * [g * f] * [\alpha] = \hat{\alpha}([g * f]) = \hat{\gamma}([g * f])
	\end{align*}
	and so $\hat{\gamma}([f * g]) = \hat{\gamma}([g * f])$. Using cancellation, we get $[f * g] = [g * f]$ (multiply by $[\gamma]$ on the left and by $[\overline{\gamma}]$ on the right) 
	so that $\pi_1(X,x_0)$ is abelian.
\end{proof}

\subsection*{Exercise 4}

Let $A\subset X$; suppose that $r: A\to X$ is a continuous map such that $r(a) = a$ for each $a\in A$. (The map $r$ is called a retraction of $X$ onto $A$). If $a_0\in A$, show that
\[ r_*: \pi_1(X,a_0) \to \pi_1(A,a_0) \]
is surjective.
\begin{proof}
	Choose any $[f] \in \pi_1(A,a_0)$. Since $f$ (and any map, path homotopic to it) lies in $A$, we know that $r([f]) = [f]$ by definition of retraction. Thus, $[f]\in \pi_1(X,x_0)$ as wanted.
\end{proof}

\section{Covering Spaces}

\subsection*{A fiber over a point in a covering space is the discrete space}

Let $p: E\to B$ be a covering map. Let $b\in B$. Equip $p^{-1}(b)$ with the subspace topology. Then $p^{-1}(b)$ is precisely the discrete space.
\begin{proof}
	A subspace $p^{-1}(b)$ of $E$ is discrete if and only if for each $h\in p^{-1}(b)$, there exists an open set $V$ in $E$ such that $V\cap p^{-1}(b) = \{h\}$. Note $p$ is a covering
	map, so in particular, there exists some neighbourhood $U$ of $B$ that $p$ evenly covers. So, we partition $p^{-1}(U)$ into open disjoint sets and denote them by $V_{\alpha}$.
	Thus, $h\in p^{-1}(b)$ falls in some unique $V_{\alpha}$ and so $\{h\} \subset V_{\alpha} \cap p^{-1}(b)$. Note that $p$ is a covering map so $p|_{V_{\alpha}}: V_{\alpha} \to B$ 
	is a homeomorphism. In particular, this means that $\# (p|_{V_{\alpha}}^{-1})=1$ or equivalently, $\{h\} = V_{\alpha} \cap p^{-1}(b)$. Thus, $p^{-1}(b)$ is a discrete space.
\end{proof}

\end{document}
