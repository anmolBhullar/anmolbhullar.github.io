\documentclass{article}
\usepackage[utf8]{inputenc}
\usepackage [english]{babel}
\usepackage [autostyle, english = american]{csquotes}
\MakeOuterQuote{"}
\usepackage{amsmath}
\usepackage{amssymb}
\usepackage{amsfonts}
\usepackage{amsthm}

\title{Tutorial Notes}
\author{CSCB36 TUT001}
\date{April 2018}

\begin{document}

\maketitle

\section*{Tutorial \#2: Example of a Badly Written Proof}

\begin{enumerate}
	\item[Line 1:] Writing "$P(n) = f(n) = n$ for all $n\in\mathbb{N}$". This should be "For all $n\in\mathbb{N},$ $P(n): f(n) = n$". Note that we moved the quantifier statement to the beginning of
	the sentence.\textbf{(*)}
	\item[Line 2-4:] Don't use stack of equations. Also, these steps have no justification. Instead, these lines should be written as:
		\begin{align*}
			f(n) = f(2) &= 2^2 - 2 \\
				&= 2 \\
				&= n
		\end{align*}
	\item[Line 5:] "Therefore, $f(2)$ holds." should be "Therefore, $P(2)$ holds."
	\item[Line 6:] This is an off by one error. Rather than "Let $n>2$", we should have "Let $n\geq 2$" since the former shows that $P(2)$, $P(3)\implies P(4)$, $P(4)\implies P(5)$, $\hdots$
		In particular, it does not show $P(3)$ but the latter does.
	\item[Line 7:] Refer to this by induction hypothesis or \textbf{IH}.
	\item[Line 9:] This is using what we want to prove. Notice, it is written "$P(n+1) = (n+1)^2-2 = n+1$". Also, it is wrong to say $P(n+1) = (n+1)^2-2$ since $P$ is a predicate.
	\item[Line 11:] Indicate that we are using our \textbf{IH}.
	\item[Line 15:] Should say "$P(n+1)$ is true" instead of just "true".
	\item[Line 9-14:] This is again, a stack of equations. The logic of these lines is wrong.
		There is no way of fixing these lines (even if we get rid of the stack) since what we are trying to prove ($P(n+1)$) is demonstrably false.
\end{enumerate}

\section*{Tutorial \#3: Understanding "only" and "all" or "every" in statements}

Let,
\begin{align*}
\end{align*}

\end{document}
