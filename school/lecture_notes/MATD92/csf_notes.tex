\documentclass{article}
\usepackage[utf8]{inputenc}
\usepackage{amsmath}
\usepackage{amssymb}
\usepackage{amsfonts}
\usepackage{amsthm}

\newtheorem{theorem}{Theorem}

\title{CSF Notes}
\author{Anmol Bhullar}
\date{April 2018}

\begin{document}

\maketitle

\section{Basics}

Suppose $\{\Gamma_t\subset\mathbb{R}^2\}$ is one-parameter family of embedded (i.e. simple) curves. If
this family moves by curve shortening flow (CSF), by definition, it satisfies:
\begin{align} \tag{1.1}
    \partial_t (p) = \vec{\kappa}(p)
\end{align}
where $p$ is some point on $\Gamma_t$. In order to compute $\partial_t(p)$ and $\vec{\kappa}(p)$, first, we
parameterize $\Gamma_t$ by some parameteric equation $\phi_t:[0,2\pi]\to\mathbb{R}^2$. Suppose this is in
arc-length parameterization. Then, $\partial_t(p)$ is just computed by differentiating $\phi_t(x)$ 
with respect to $t$ ($x$ some element of domain) and $\vec{\kappa}(p)$ is computed by $\partial_x^2[\phi_t(x)]$.
Suppose $\phi_t$ is not given in arc-length parameterization. Then, if $\phi_t(s) = (x(s),y(s))$, we have that the
signed curvature is given by the formula:
\[ k(s) = \frac{x'y''-x''y'}{((x')^2+(y')^2)^{3/2}}\]
so that we can then compute $\vec{k}$ by evaluating $k(s)N(s)$ where $N$ denotes the normal vector at of $\alpha$ at 
$s$.

\section{The CSF equation is not strictly parabolic}

The curve shortening flow equation can be (erroneously) viewed as a heat equation by writing $\gamma$ in terms of its arclength parameter. Notice:
\[ \partial_t \gamma = \partial_s^2 \gamma \]
which is exactly the heat equation. However, the catch is that $\partial_s$ is not a partial derivative but is instead defined via the equation:
\[ \partial_s = v^{-1}\partial_x,\qquad v = \frac{\partial \gamma}{\partial x} \]
We can see the true nature of this equation by writing it in vector form. Write $\gamma = (\gamma^1, \gamma^2)$ and we would obtain:
\[ \begin{pmatrix} \partial_t \gamma^1 \\ \partial_t \gamma^2 \end{pmatrix} = |\partial_x \gamma|^{-4} \begin{pmatrix} (\partial_x \gamma^2)^2 & \partial_x\gamma^2 \partial_x\gamma^2 \\
	\partial_x\gamma^1\partial_x\gamma^2 & (\partial_x\gamma^1)^2 \end{pmatrix} \begin{pmatrix} \partial_x^2 \gamma^1 \\ \partial_x^2 \gamma^2 \end{pmatrix} \]
Note that the matrix (in front of the vector on the right side) is positive \textit{semi}definite and so the CSF equation is not strictly parabolic.

\subsection{Round shrinking circles}
We show that if $\Gamma_t = \partial B^2_{r(t)}\subset\mathbb{R}^2$, then (1.1) reduces to the ODE:
\[ \dot{r} = -1/r \]
and if we give it the initial value $r(0) = R$, then $r(t) = \sqrt{R^2 - 2t}$ for $t\in(-\infty,R^2/2)$.
\begin{proof}
    Let $r(t)$ be a $C^1$ function that gives a radius dependent on the parameter $t$.
    Assume $B^2_{r(t)} = \{x\in\mathbb{R}^2: |x|< r(t)\}$. Then $\partial B^2_{r(t)} = \{x\in\mathbb{R}^2:|x|=r(t)\}$.
    Fix $t$. We let $\partial B^2_{r(t)}$ be given by the parameterization $\phi_t: [0,2\pi]\to\mathbb{R}^2$,
    \[ \phi_t(s) = r(t)(\cos(s),\sin(s))\]
    Now, we compute $\partial_t(\phi_t(s))$. Consider:
    \begin{align*}
        \partial_t(\phi_t(s)) = \partial_t(r(t)(\cos(s),\sin(s))) = \dot{r}(t)(\cos(s),\sin(s))
    \end{align*}
    and also, we can compute the signed curvature:
    \begin{align*}
        k(s) &= \frac{r^2(t)[\cos'(s)\sin''(s)-\cos''(s)\sin'(s)]}{((r(t)\cos'(s))^2+(r(t)\sin'(s))^2)^{3/2}} \\
        	       &= \frac{r^2(t)[\sin^2(s) + \cos^2(s)]}{(r^2(t)[\sin^2(s) + \cos^2(s)])^{3/2}} = \frac{r^2(t)}{r^3(t)} = \frac{1}{r(t)}
    \end{align*}
    
    Now, we compute the unit normal. Note $v(s) = |\partial_s(\phi_t(s))| = r(t)$. So, the unit tangent vector $\mathbf{T}(s)$ is given by,
    \[ \mathbf{T}(s) = \partial_s(\phi_t(s))/v(s) = r(t)(-\sin(s),\cos(s))/r(t) = (-\sin(s),\cos(s))\]
    from this, we can compute the unit normal:
    \[ \kappa\mathbf{N} = \frac{\partial_s(T(s))}{\partial_s(v(s))} = (-\cos(s),-\sin(s))\cdot 1/r(t)   \]
    From this, we see that $\mathbf{N} = (-\cos(s),-\sin(s))$ so that (1.1) reduces down to:
    \[ \dot{r}(t)(\cos(s),\sin(s))  = [1/r(t)](-\cos(s),-\sin(s)) \]
    which of course, yields:
    \[ \dot{r}(t) = -1/r(t) \]
    as wanted. Now, set $R > 0$ and give the initial value data $r(0) = R$. We have $dr/dt = -1/r$ which we can rearrange to get $r dr = -1 dt$.
    Integrating both sides, we get $r^2/2 = -t + c$ for some constant $c$. If $r(0) = R$, then $c = R$. Thus, we get:
    \[ r(t) = \sqrt{2}\sqrt{R-t} \;\text{for}\;t\in (-\infty, R)\]
\end{proof}

\subsection{Grim Reaper Curve}

A self similar solution to the curve shortening flow is given by the grim reaper curve. This is given by the family
$\Gamma_t =$ graph$(\log \cos p)+t$ where $p\in(-\pi/2,\pi/2)$ and $t\in\mathbb{R}$.\\

Now, instead consider the family: 
\[ \Phi_t =\:\text{graph}(u_t(p)) = \{(p,u_t(p)): p\in\:\text{Dom}(u_t:U\subset\mathbb{R}\to\mathbb{R})\}. \]
We ask which equation does $u_t$ satisfy. We parameterize $\Phi_t$ via the map $\phi_t: U\to \mathbb{R}^2$ given by
the mapping $p\in U\mapsto (p,u_t(p))$. Immediately, we see that 
\[ \partial_t(\phi_t(p)) = (\partial_t(p),\partial_t(u_t(p))) = (0,\partial_t(u_t(p))) \]
Now, we calculate $\vec{k}$ by calculating $k\textbf{N}$. Note, 
\[ \nu(p) = |\partial_p(\phi_t(p))| = |(\partial_p(p),\partial_p(u_t(p)))| = \sqrt{1^2 + (\partial_p u_t(p))^2}\]
so that,
\[ \mathbf{T}(p) = \frac{\partial_p u_t(p)}{\nu(p)} = \frac{(1,\partial_p(u_t(p)))}{\sqrt{1^2 + (\partial_p u_t(p))^2}}\]
and so,
\begin{align*}
    \partial_p \mathbf{T}(p) &= \partial_p\Big{(}\frac{(1,\partial_p u_t(p))}{\sqrt{1+(\partial_p u_t(p))^2}}\Big{)} \\
    &= \frac{\partial_p (1,\partial_p u_t(p))}{\sqrt{1+(\partial_p u_t(p))^2}} - (1,\partial_p u_t(p))\partial_p[1+(\partial_p u_t(p))^2]^{-1/2} \\
    &= \frac{(0,\partial_p^2 u_t(p))}{\sqrt{1+(\partial_p u_t(p))^2}} - \frac{\partial_p u_t(p)[\partial_p^2 u_t(p)](1,\partial_p u_t(p))}{[1+(\partial_p u_t(p))^2]^{3/2}} \\
    &= \frac{[1+(\partial_p u_t(p))^2](0,\partial_p^2 u_t(p))}{[1+(\partial_p u_t(p))^2]^{3/2}} - \frac{\partial_p u_t(p)[\partial_p^2 u_t(p)](1,\partial_p u_t(p))}{[1+(\partial_p u_t(p))^2]^{3/2}} \\
    &= \frac{[\partial_p^2 u_t(p)](0,1+(\partial_p u_t(p))^2) - [\partial_p^2 u_t(p)](\partial_p u_t(p),(\partial_p u_t(p))^2}{[1+(\partial_p u_t(p))^2]^{3/2}} \\
    &= [\partial_p^2 u_t(p)]\frac{(0,1+(\partial_p u_t(p))^2) - (\partial_p u_t(p),(\partial_p u_t(p))^2}{[1+(\partial_p u_t(p))^2]^{3/2}}) \\
    &= [\partial_p^2 u_t(p)]\frac{(-\partial_p u_t(p),1)}{[1+(\partial_p u_t(p))^2]^{3/2}}
\end{align*}
Thus $\vec{\kappa}$ can be computed by:
\begin{align*}
    \kappa\mathbf{N} &= \frac{\partial_p T(p)}{\nu(p)} = \frac{1}{\nu(p)}\partial_p T(p) \\
        &= \frac{1}{\sqrt{1^2 + (\partial_p u_t(p))^2}}\frac{[\partial_p^2 u_t(p)](-\partial_p u_t(p),1)}{[1+(\partial_p u_t(p))^2]^{3/2}} \\
        &= \frac{[\partial_p^2 u_t(p)](-\partial_p u_t(p),1)}{[1+(\partial_p u_t(p))^2]^2}
\end{align*}
Note, that by computing the norm of $\vec{\kappa}$, we get:
\begin{align*}
    |\kappa\mathbf{N}| &= \frac{|\partial_p^2 u_t(p)}{(1+(\partial_p u_t(p))^2)^2}\big{|}(-\partial_p u_t(p),1)\big{|} \\
    &= \frac{|\partial_p^2 u_t(p)|}{(1+(\partial_p u_t(p))^2)^{3/2}}
\end{align*}
Thus, we obtain the following:
\[ \kappa = \frac{\partial_p^2 u_t(p)}{(1+(\partial_p u_t(p))^2)^{3/2}}\text{,}\qquad \mathbf{N} = \frac{(-\partial_p u_t(p),1)}{\sqrt{1+(\partial_p u_t(p))^2}}\]
Now, we use the equation: 
\[ \partial_t \phi_t(p)\cdot \mathbf{N} = \kappa \]
which after substitution, reduces down to:
\[ \frac{\partial_t u_t(p)}{(1+(\partial_p u_t(p))^2)^{1/2}} = \frac{\partial_p^2 u_t(p)}{(1+(\partial_p u_t(p))^2)^{3/2}} \]
Thus, we get:
\[ \partial_t u_t(p) = \frac{\partial_p^2 u_t(p)}{1+(\partial_p u_t(p))^2}\text{, or}\quad \partial_p^2 u_t(p) = \partial_t u_t(p)[1+(\partial_p u_t(p))^2] \]
If $\partial_t u_t(p) = 1$, we get the ODE:
\[ \partial_p^2 u_t(p) = 1+(\partial_p u_t(p))^2 \]
We attempt to solve this. First, we rewrite it as:
\[ y''(x) = 1  + (y'(x))^2 \]

\subsection{Evolution equation of length}

We derive the evolution equation of  $L(t) = \int_{\Gamma_t} ds$. Note if $\Gamma_t$ is parameterized by $\gamma(x,t): S^1 \times [0,T) \to \mathbb{R}^2$, then:
\[ L(t) = \int_{S^1} \langle \partial_x \gamma, \partial_x \gamma\rangle^{1/2}dx \]
We compute,
\begin{align*}
	\partial_t L(t) &= \partial_t \Big{(} \int_{S^1} \langle \partial_x \gamma, \partial_x \gamma\rangle^{1/2}dx \Big{)} \\
		&= \int_{S^1} \partial_t \langle \partial_x \gamma, \partial_x \gamma\rangle^{1/2}dx \\
		&= \int_{S^1} \frac{1}{2}\langle \partial_x \gamma, \partial_x \gamma\rangle^{-1/2} 
			[\langle \partial_t \partial_x \gamma, \partial_x \gamma\rangle + \langle \partial_x \gamma, \partial_t \partial_x \gamma\rangle] dx \\
		&= \int_{S^1} \frac{1}{2}|\partial_x \gamma|^{-1} 2\langle \partial_t \partial_x \gamma, \partial_x \gamma\rangle dx \\
		&= \int_{S^1} \langle \partial_t \partial_x \gamma, \frac{\partial_x \gamma}{|\partial_x \gamma|}\rangle dx \\
		&= \int_{S^1} \langle \partial_x \partial_t \gamma, T\rangle dx
\end{align*}
By definition of the curve shortening flow, we know that $\partial_t \gamma = \kappa N$. Thus,
\[ \partial_t L(t) = \int_{S^1} \langle \partial_x (\kappa N), T\rangle dx \]
By the Frenet-Serret formulas, we have $\partial_s N = -\kappa T$ and so by chain rule, we obtain $\partial_x N = -(\partial_x s)\kappa T$. Thus,
\[ \langle \partial_x (\kappa N), T\rangle = \langle (\partial_x \kappa)N, T\rangle + \langle \kappa \partial_x N, T\rangle = 0 + \kappa [(\partial_x s)\kappa T]\cdot T = -\kappa^2 \partial_x s \]
Writing instead $\langle \partial_x (\kappa N),T\rangle = -\kappa^2 \frac{ds}{dx}$, we easily obtain the final equation:
\[ \partial_t L(t) = \int_{\Gamma_t} -\kappa^2 ds \]

\subsection{Evolution equation of curvature}

We show that $\kappa_t = \kappa_{ss} + \kappa^3$. For convenience sake, set $|\partial_x \gamma| = 1$ and $\langle \partial_x^2 \gamma, T\rangle = 0$ at the point $(x,t)$.
Note $\langle \partial_x^2 \gamma, N\rangle = \langle \kappa N, N\rangle = \kappa$. Since $|\partial_x \gamma| = 1$, we can also say:
\[ \kappa = \frac{\langle \partial_x^2 \gamma, N\rangle}{|\partial_x \gamma|^2} \]
We evaluate $\kappa_t$ i.e. $\partial_t \kappa$.
\begin{align*}
	\partial_t \kappa &= \frac{\partial_t(\langle \partial_x^2 \gamma, N\rangle)}{|\partial_x \gamma|^2} + \frac{\langle \partial_x^2 \gamma, N\rangle}{\partial_t  |\partial_x \gamma|^2} \\
		&= \partial_t(\langle \partial_x^2 \gamma, N\rangle) + \partial_t (\langle \partial_x \gamma, \partial_x \gamma\rangle^{-1})\langle \partial_x^2 \gamma, N\rangle \\
		&= \partial_t(\langle \partial_x^2 \gamma, N\rangle) - \langle \partial_x \gamma, \partial_x \gamma\rangle^{-2}\partial_t (\langle \partial_x\gamma, \partial_x\gamma\rangle) \langle \partial_x^2 \gamma, N\rangle \\
		&= \partial_t(\langle \partial_x^2 \gamma, N\rangle) - 2|\partial_x \gamma|^{-4}\langle \partial_t \partial_x\gamma, \partial_x\gamma\rangle \langle \partial_x^2 \gamma, N\rangle \\
		&= \partial_t(\langle \partial_x^2 \gamma, N\rangle) - 2\langle \partial_t \partial_x\gamma, T\rangle \langle \partial_x^2 \gamma, N\rangle \\
		&= \langle \partial_t \partial_x^2 \gamma, N\rangle + \langle \partial_x^2 \gamma, \partial_t N\rangle - 2\langle \partial_t \partial_x\gamma, T\rangle \langle \partial_x^2 \gamma, N\rangle \\
		&= \langle \partial_x^2 \partial_t \gamma, N\rangle + \langle \partial_x^2 \gamma, \partial_t N\rangle - 2\langle \partial_x \partial_t\gamma, T\rangle \langle \partial_x^2 \gamma, N\rangle
\end{align*}
Now, substitute $\kappa N$ for $\partial_t \gamma$ and $\kappa N$ for $\partial_x^2 \gamma$ to get:
\begin{align*}
	\partial_t \kappa &= \langle \partial_x^2 (\kappa N), N\rangle + \langle \kappa N, \partial_t N\rangle - 2\langle \partial_x (\kappa N), T\rangle \langle \kappa N, N\rangle \\
		&= \partial_x^2 (\kappa)\langle N, N\rangle + \kappa\langle \partial_x^2 N, N\rangle + \kappa\langle N, \partial_t N\rangle - 2[\partial_x (\kappa)\langle N, T\rangle + 
			\kappa\langle \partial_x N, T\rangle]\kappa\langle N, N\rangle \\
		&= \partial_x^2 (\kappa) + \kappa\langle \partial_x(\partial_x N), N\rangle - 2\kappa^2\langle \partial_x N, T\rangle
\end{align*}
By the Frenet formulas, we know $\partial_x N = -\kappa T$. So we can continue our computation:
\begin{align*}
	\partial_t \kappa &= \partial_x^2 (\kappa) + \kappa\langle \partial_x(-\kappa T), N\rangle - 2\kappa^2\langle -\kappa T, T\rangle \\
		&= \partial_x^2 (\kappa) + \kappa[\langle \partial_x (-\kappa)T, N\rangle - \langle \kappa \partial_x T, N\rangle] + 2\kappa^3 \\
		&= \partial_x^2 (\kappa) - \kappa^2[\langle \partial_x T, N\rangle] + 2\kappa^3 \\
		&= \partial_x^2 (\kappa) - \kappa^3 + 2\kappa^3 \\
		&= \partial_x^2 (\kappa) + \kappa^3
\end{align*}
From our assumption in the beginning, we can regard $\partial_x^2 (\kappa)$ as $\partial_{ss} (\kappa)$. From this, we get:
\[ \kappa_t = \partial_{ss} \kappa + \kappa^3 \]
as wanted.

\subsection{Evolution equation of area}

Let $A(t)$ denote the area enclosed by $\Gamma_t$. We equate $\partial_t A(t)$.\\

Let $\Gamma_t$ be given by $(x,t) = F(u)$. By Green's theorem, we immediately obtain:
\[ A(t) = \frac{1}{2} \int_{0}^{2\pi} \big{(}x\frac{\partial y}{\partial u} - y\frac{\partial y}{\partial u}\big{)}du \]
We have $\partial F/\partial u = \partial (x,y) / \partial u = (\partial x/\partial u, \partial y/\partial u)$. Note 
\[ \langle (\partial x/\partial u, \partial y/\partial u),-(\partial y/\partial u, \partial x/\partial u)\rangle=0 \]
so these are orthogonal. Since $\partial F/\partial u$ is the tangent, we then obtain that $n := -(\partial y/\partial u, \partial x/\partial u)$ is proportional to the inward pointing unit normal.
In particular, $N = n/|n|$. Note $|n|$ is simply equal to the norm of $\partial_u F$ so we denote it by $v$. Furthermore,
\[ \langle n, F\rangle = -x \frac{\partial y}{\partial u} + y\frac{\partial x}{\partial u} \]
but $n = vN$ so, we simply write:
\[ \langle vN, F\rangle = -x \frac{\partial y}{\partial u} + y\frac{\partial x}{\partial u} \]
Thus,
\[ A(t) = \frac{-1}{2} \int_0^{2\pi} \langle F, vN\rangle du \]
Now, we can compute $\partial_t A(t)$. Consider:
\begin{align*}
	\partial_t A(t) &= \frac{-1}{2} \int_0^{2\pi} \langle \partial_t F, vN\rangle + \langle F, \partial_t vN\rangle du \\
		&= \frac{-1}{2} \int_0^{2\pi} \langle \partial_t F, vN\rangle + \langle F, \partial_t (v)N\rangle +  \langle F, v\partial_t N\rangle du
\end{align*}
We have by the curve shortening flow that $\partial_t F = \kappa N$ so $\langle \partial_t F, vN\rangle = \langle \kappa N, vN\rangle = \kappa v$. Now, we try to compute $\partial_t N$.
Note:
\[ 0 = \langle T, N\rangle \implies 0 = \partial_t \langle T, N\rangle = \langle \partial_t T, N\rangle + \langle T, \partial_t N\rangle \]
So, we first compute $\partial_t T$: $\partial_t T = \partial_t (\partial_s F) = \partial_{ts} F$. We prove the following equality: (recall $s$ depends on both $u$ \textbf{and} $t$)
\[ \partial_{ts} = \partial_{st} + \kappa^2 \partial_s \qquad(*)\]
Recall that $\partial_s$ is defined as $v^{-1}\partial_u$ and so,
\begin{align*}
	\partial_t\partial_s = \partial_t[v^{-1}\partial_u] = \partial_t(v^{-1})\partial_u + v^{-1}\partial_t(\partial_u)
\end{align*}
So, again we need to do a preliminary calculation. Find $\partial_t v$:
\begin{align*}
	\partial_t v^2 = \partial_t(\langle \partial_u F, \partial_u F\rangle) &= 2\langle \partial_{tu} F, \partial_u F\rangle\\
		&= 2\langle \partial_{u}(\partial_t F), \partial_u F\rangle \\
		&= 2\langle \partial_u(\kappa N), vT\rangle \\
		&= 2\langle \partial_u(\kappa)N + (\partial_u N)\kappa, vT\rangle \\
		&= 2\langle -\kappa^2 vT, vT\rangle = -2v^2 \kappa^2
\end{align*}
where we have \textbf{to show $\partial_u \kappa = 0$?} Thus, $\partial_t v = -v\kappa^2$. So, we can continue with (*):
\begin{align*}
	\partial_t(v^{-1})\partial_u + v^{-1}\partial_t(\partial_u) &= v^{-2}v\kappa^2 \partial_u + v^{-1}\partial_u\partial_t \\
		&= \kappa^2 v^{-1} \partial_u  + v^{-1}\partial_u\partial_t \\
		&= \kappa^2\partial_s + \partial_s\partial_t
\end{align*}
Thus, $\partial_t\partial_s = \partial_t[v^{-1}\partial_u] = \partial_t(v^{-1})\partial_u + v^{-1}\partial_t(\partial_u)$ as wanted. So,
\begin{align*}
	\partial_t T = \partial_t(\partial_s F) &= \partial_s(\partial_t F) + \kappa^2 \partial_s F \\
		&= \partial_s(\kappa N) + \kappa^2 T \\
		&= N\partial_s \kappa + \kappa\partial_s N + \kappa^2 T \\
		&= N\partial_s \kappa -  \kappa^2 T + \kappa^2 T \\
		&= N\partial_s \kappa = (\partial_s \kappa)T
\end{align*}
Now, we compute $\partial_t N$:
\begin{align*}
	0 = \partial_t \langle T, N\rangle &= \langle \partial_t T, N\rangle + \langle T, \partial_t N\rangle \\
		&= \langle N\partial_s\kappa, N\rangle + \langle T,\partial_t N\rangle
\end{align*}
So, $\langle T, \partial_t N\rangle = \partial_s \kappa$. Now, $T$ and $\partial_t N$ are parallel so $\partial_t N = -T \partial_s \kappa$. Thus,
\[ \langle F, v\partial_t N\rangle = \langle F, -vT\partial_s \kappa\rangle = \langle F, -T\partial_u \kappa\rangle \]
and $v\kappa^2 N = \kappa v\kappa N = \kappa\partial_u T$. So, we currently have:
\[ \partial_t A = -\frac{1}{2} \int_0^{2\pi} [v\kappa - \langle F, \kappa\partial_u T\rangle + \langle F, -T \partial_u \kappa\rangle] du \]
Note $\partial_u \langle F,\kappa T\rangle + \langle F, -(\partial_u \kappa)T\rangle = \langle \partial_u F, \kappa T\rangle + \langle F, \kappa \partial_u T\rangle$ and
 \[ \int_0^{2\pi} \partial_u \langle F, \kappa T\rangle = \langle F(2\pi), \kappa(2\pi)T(2\pi) \rangle - \langle F(0), \kappa(0)T(0) \rangle = 0 \]
So,
\[ \int_0^{2\pi} \langle F,-(\partial_u \kappa)T\rangle = \int_0^{2\pi} [\langle \partial_u F, \kappa T\rangle + \langle F, \kappa \partial_u T\rangle]du \]
Thus:
\begin{align*}
	\partial_t A &= -\frac{1}{2} \int_0^{2\pi} [v\kappa - \langle F, \kappa\partial_u T\rangle + \langle \partial_u F, \kappa T\rangle + \langle F, 
        \kappa \partial_u T\rangle] du \\
		&= -\int_0^{2\pi} [v\kappa + \langle \partial_u F, \kappa T\rangle] du \\
		&= -\int_0^{2\pi} [v\kappa] du = -2\pi
\end{align*}
since the total curvature of a simple closed curve is $2\pi$. Thus, $\partial_t A = -2\pi$ and so $A(t) = A(0) -2\pi t$ as wanted.$\hfill\blacksquare$
\newpage

\subsection{Things to do:}
\begin{enumerate}
    \item Prove: If $\Gamma_{-1}$ satisfies $\kappa + (1/2)\langle \gamma, N\rangle = 0$, then $\Gamma_t := \sqrt{-t}\Gamma_{-1}$ moves under the curve shortening flow.
    \item Prove: If $\Gamma_t := \sqrt{-t}\Gamma_{-1}$ $(t < 0)$ moves under the curve shortening flow, then $\Gamma_{-1}$ satisfies 
        $\kappa + (1/2)\langle\gamma,N\rangle = 0$.
    \item Show the parabolic rescaling of curves moving under the curve shortening flow still move under the curve shortening flow.
    \item Work out the details to Huisken's Monotonicity formula.
    \item Prove convexity is preserved under the curve shortening flow.
    \item Prove if two curves move under the curve shortening flow and are disjoint at initial time, then they stay disjoint under the flow.
\end{enumerate}

\section{Huisken's Monotonicity Formula}

\subsection{Curvature under uniform scaling}

First, we show how curvature changes under uniform scaling.

Say I have some smooth curve $\Gamma$ in $\mathbb{R}^2$, I am trying to figure out what happens to its curvature when we scale it by some multiple (say any $t>0$). 

First, I assume I have some parameterization $\gamma$ of $\Gamma$. Then, I can calculate $\kappa (\gamma(x))$ by $||\partial_s^2 \gamma(s)||$ where $s$ is 
some arc-length parameter. This is all good.

Now, let's say I want to find the curvature of the curve $t\Gamma$. Then,
I can parameterize this via $t\gamma$. But the old arc-length parameter $s$ does not give us the unit tangent vector any more since I have the factor $t>0$. 
To remedy this, we simply divide by $t$ so:
\[ \mathbf{T} = \frac{\partial_s(t\gamma)}{t}= \frac{t(\partial_s \gamma)}{t} = \partial_s \gamma \]
which we know is a unit vector since $\partial_s \gamma$ is. To find $\kappa \vec{N}$, we then need to compute 
\[ \frac{\partial_s\mathbf{T}}{t} = \frac{\partial_s \partial_s \gamma}{t} = \frac{\kappa\vec{N}}{t} \]
so that,
\[ \kappa = \frac{\partial_s^2 \gamma}{t} \]
We can also apply this line of reasoning to $\vec{N}$. In order to keep $\vec{N}$ a unit vector, we must divide by the scaling factor so if $\mathbf{N}$ is the
unit vector of $\gamma$, then $\vec{N} = \mathbf{N}/t$ is the unit vector of $t\gamma$.

\subsection{Characterizing self-similar shrinking solutions}

We study curves which are of the form $\Gamma_t = \sqrt{-t}\Gamma_{-1}$ where $(t < 0)$ and $\Gamma_{-1}$ is a smooth closed (and not necessairly embedded) 
curve in $\mathbb{R}^2$.

We first show that if $\Gamma_{-1}$ makes the equation $\kappa + (1/2)\langle \gamma,\vec{N}\rangle = 0$ hold, then $\{\Gamma_t\}_t$ moves under CSF. \\

We can start by calculating the curvature of $\Gamma_t$. Consider:
\[ \kappa(\gamma_t(x)) = \kappa(\sqrt{-t}\gamma_{-1}(x)) = \frac{\kappa(\gamma_{-1}(x))}{\sqrt{-t}} \]
Note $\Gamma_{-1}$ has the following assumed property:
\[ \kappa(\gamma_{-1}(x)) + (1/2)\langle \gamma_{-1}(x), \vec{N}(\gamma_{-1}(x))\rangle = 0 \]
so we can write:
\begin{align} 
	\kappa(\gamma_t(x)) = \frac{\langle \gamma_{-1}(x), \vec{N}(\gamma_{-1}(x))\rangle}{-2t}
\end{align}
We also have $\vec{N}(\gamma_t(x)) = \vec{N}(\gamma_{-1}(x))/\sqrt{-t}$ and
\begin{align*}
	\partial_t(\gamma_t(x)) &= \partial_t(\sqrt{-t})\gamma_{-1}(x) + \sqrt{-t}\partial_t \gamma_{-1}(x) \\
		&= \frac{-\gamma_{-1}(x)}{2\sqrt{-t}}
\end{align*}
so:
\begin{align*}
	\langle \partial_t \gamma_t(x), \vec{N}(\gamma_{t}(x))\rangle &= \langle \frac{-\gamma_{-1}(x)}{2\sqrt{-t}}, \frac{\vec{N}(\gamma_{-1}(x))}{\sqrt{-t}} \rangle \\
		&= \frac{\langle \gamma_{-1}(x), \vec{N}(\gamma_{-1}(x)) \rangle}{2t} \\
		&= -\kappa(\gamma_t(x))
\end{align*}
which makes sense since we have negative (i.e. backwards) time.

Now, we show the converse is true: If $\{\Gamma_t\}_t$ move under the curve shortening flow, then $\Gamma_{-1}$ satisfies
\[ \kappa(\gamma_{-1}(x)) + (1/2)\langle \gamma_{-1}(x), N(\gamma_{-1}(x))\rangle = 0\] 
Note, we still have:
\[ \langle \partial_t \gamma_t(x), \vec{N}(\gamma_{t}(x))\rangle = \frac{\langle \gamma_{-1}(x), \vec{N}(\gamma_{-1}(x)) \rangle}{2t} \]
and by assumption, we also have:
\[ \langle \partial_t \gamma_t(x), \vec{N}(\gamma_{t}(x))\rangle = -\kappa(\gamma_t(x)) \]
which implies that we have:
\[ \kappa(\gamma_t(x)) + \frac{\langle \gamma_{-1}(x), \vec{N}(\gamma_{-1}(x)) \rangle}{2t} = 0 \]
Plugging in $t=-1$ and cancelling the negatives (from the curvature and the denominator), we get:
\[ \kappa(\gamma_{-1}(x)) + \frac{\langle \gamma_{-1}(x), \vec{N}(\gamma_{-1}(x)) \rangle}{2} = 0 \]
as wanted.\\

This characterizes self-similar shrinking solutions of the curve shortening flow. It is much harder to prove (but it is true) that for embedded curves,
the line and the circle are the only examples of such curves.

\subsection{The Monotonicity Formula}

We want to consider blowup sequences $\lambda \Gamma$ as $\lambda\to\infty$. The length formula previously seen will not help us here since we know
length$(\lambda\Gamma) = \lambda$length$(\Gamma)$. This is where the monotonicity formula comes into play.\\

Let $\{\Gamma_t \subseteq \mathbb{R}^2\}$ be a CSF and let $X_0 = (x_0,t_0)$ be a point in space-time, and let
\begin{align}
	\rho_{X_0}(x,t) = (4\pi (t_0 - t))^{-1/2}e^{-|x-x_0|^2/4(t_0-t)}\qquad (t < t_0)
\end{align}
be the one dimensional backwards heat kernel centered at $X_0$.

\begin{theorem}[Huisken's Monotonicity Formula]
	\begin{align}
		\frac{d}{dt} \int_{\Gamma_t} \rho_{X_0}ds = -\int_{\Gamma_t} \Big{|}\kappa + \frac{\langle \kappa, N\rangle}{2(t_0-t)}\Big{|}^2\rho_{X_0}ds
	\end{align}
\end{theorem}
As seen before, the quantity within the absolute signs exactly characterizes the selfsimilarly shrinking solutions.
\begin{proof}
	Recall that the tangential gradient is the gradient without the normal component. So, the tangential gradient of $\rho$ is:
	\begin{align} \triangledown \rho - \langle \triangledown \rho, N \rangle N \end{align}
	We can write that this is equal to $\partial_s \rho$ since we are only moving along the tangential part or as $\triangledown_T \rho$ ($T =$ tangent
	unit vector of $\Gamma$) i.e.
	\begin{align} 
		\triangledown_T \rho = \partial_s \rho \equiv (\triangledown_T \rho)^T = \triangledown \rho - \langle \triangledown \rho, N\rangle N
	\end{align}
	Following this notation, we would have that $\partial_s^2$ represents the `intrinsic Laplacian' i.e. gives only the second derivative along the curve.
	So by definition,
	\begin{align} 
		\partial_s^2 \rho &\equiv \langle T, \triangledown_T \triangledown_T \rho\rangle  \\
			&= \langle T, \triangledown_T (\triangledown \rho - \langle \triangledown \rho, N\rangle N)\rangle \\
            &= \langle T, \triangledown_T \triangledown \rho\rangle - \langle T, \triangledown_T \langle \triangledown \rho, N\rangle N\rangle \rangle \\
            &= \langle T, \triangledown_T \triangledown \rho\rangle - 2\kappa \langle N, \triangledown \rho\rangle
	\end{align}
    We have $\rho = \rho_0$ which is just equal to $(4\pi t)^{-1/2}e^{-|x|^2/4t}$ so
\end{proof}

\subsection{Avoidance Principle}

Two curves which follow CSF and are disjoint will stay disjoint under the flow.
\begin{proof}
    Suppose we have curves $T$ and $\overline{T}$ which intersect at a single point (and this is the first occurence of intersection). 
    If at their intial curves, the curves did not intersect and one
    of the curves did not enclose the other, then they will never touch. So, suppose at the initial curves, one curve encolsed the other. Suppose
    $T$ encloses $\overline{T}$. Since $T$ encloses $\overline{T}$, we have that $\vec{\kappa_{T}}\leq \vec{\kappa_{\overline{T}}}$. If 
    $\vec{\kappa_{T}} < \vec{\kappa_{\overline{T}}}$, then using the curve shortening flow equation, we can conclude that they must have intersected
    before which is a contradiction. So, we have $\vec{\kappa_T} = \vec{\kappa_{\overline{T}}}$. (Now, I think I have to use the maximum principle
    here somewhere to conclude another contradiction.) Thus, curves $T$ and $\overline{T}$ cannot exist and so two curves which are initially disjoint
    will stay disjoint under the curve shortening flow.
\end{proof}

\end{document}
